%% This file is part of Enblend.
%% Licence details can be found in the file COPYING.


\section[Local Contrast Weighting]{\label{sec:local-contrast-weighting}%
  \genidx{weighting!local contrast}%
  \gensee{local contrast}{weighting, local contrast}%
  Local Contrast Weighting}

Local contrast weighting favors pixels inside a high contrast neighborhood.  The notion of
``high contrast'' is defined either by two different criteria or by a blend of both:

\begin{itemize}
\item
  The standard deviation (\acronym{SDev}) of all the pixels in the local analysis window is
  large.  See \sectionName~\fullref{sec:standard-deviation}.

\item
  The Laplacian-of-Gaussian (\acronym{LoG}) has a large magnitude.  See
  \sectionName~\fullref{sec:laplacian-of-gaussian}.

\item
  If the \acronym{LoG} magnitude is below a given threshold, use \acronym{SDev} data, otherwise
  stick with \acronym{LoG}.  See \sectionName~\fullref{sec:blend-sdev-and-log}.
\end{itemize}

\App{} converts every \acronym{RGB} image to grayscale before it determines its contrast.
Option~\flexipageref{\option{--gray-projector}}{opt:gray-projector} controls the projector
function.  Depending on the subject, one of several grayscale projectors may yield the best
black-and-white contrast for image fusion.

In the following sections we describe each algorithm in detail.


\subsection[Standard Deviation]{\label{sec:standard-deviation}%
  \genidx{weighting!contrast using standard deviation}%
  \gensee{contrast weighting using standard deviation}{weighting, contrast using standard deviation}%
  Standard Deviation}

\genidx{window!local-analysis}%
\gensee{local analysis window}{window, local-analysis}%
The pixel under consideration~\textsf{C} sits exactly in the center of a square, the so-called
local analysis window.  It always has an uneven edge length.  The user sets the size with
option~\option{--contrast-window-size}.  \figureName~\ref{fig:local-analysis-window} shows two
windows with different sizes.


\begin{figure}[htbp]
  \begin{maxipage}
    \centering
    \includeimage{local-analysis-window}
  \end{maxipage}

  \caption[Local analysis window]{\label{fig:local-analysis-window}Examples of local analysis
    windows for the sizes~3 and~5.  ``\textsf{C}'' marks the center where the pixel gets the
    weight.  ``\textsf{N}'' denote neighboring pixels, which all contribute to the weight.}
\end{figure}


During the analysis, \App{} scans the local analysis window across all rows and all
columns\footnote{In the current implementation a \code{floor(contrast-window-size / 2)}-wide
  border around the images remains unprocessed and gets a weight of zero.}\ of each of the input
images to compute the contrast weight of every pixel.

\begin{optionsummary}
\item[--contrast-weight] \sectionName~\fullref{opt:contrast-weight}
\item[--contrast-window-size] \sectionName~\fullref{opt:contrast-window-size}
\item[--gray-projector] \sectionName~\fullref{opt:gray-projector}
\item[--hard-mask] \sectionName~\fullref{opt:hard-mask}
\end{optionsummary}


\subsubsection[Statistical Moments]{\label{sec:statistical-moments}%
  \genidx{statistical moments}%
  Statistical Moments}

\genidx{probability function}%
\genidx{random variable}%
We start with the probability function~$w$ of the random variable~$X$:
\[
    w: x \rightarrow p(\{\omega: X(\omega) = x\}).
\]
\noindent It associates a probability~$p$ with each of the $n$ different possible
outcomes~$\omega$ of the random variable~$X$.

\genidx{expectation value}%
\genidx{moment!first}%
\gensee{first moment}{moment, first}%
Based on $w$, we define the expectation value or ``First Moment'' of the random variable~$X$:
\[
    \Ex X := \sum_{i = 1}^n x_i w(x_i).
\]

\genidx{variance}%
\genidx{moment!second}%
\gensee{second moment}{moment, second}%
Using the definition of the expectation value, we define the
variance, or ``Second Moment'' as
\[
    \Var X := \mbox{\Ex}\left( (X - \Ex X)^2 \right),
\]
\genidx{standard deviation}%
\noindent and the standard deviation as
\[
    \sigma \, X := \sqrt{\Var X}.
\]

Obviously, the variance of $X$ is the expectation value of the squared deviation from the
expectation value of $X$ itself.  Note that the variance's dimension is $X$'s dimension squared;
the standard deviation rectifies the dimension to make it comparable with $X$ itself again.


\subsubsection[Estimators]{\label{sec:estimators}%
  \genidx{estimators}%
  Estimators}

In \App{}, we assume that $X$ follows a uniform probability function~$w(x) = \mbox{const.}$ That
is, all pixel values in the local analysis window are considered to be equally probable.  Thus,
the expectation value and the variance can be estimated from the pixel values like this
\[
    \mbox{\Ex } X := \frac{1}{n} \sum_{i = 1}^n x_i.
\]
\noindent In other words: the expectation value is the arithmetic mean of the lightness of all
pixels in the local analysis window.  Analogously, the variance becomes
\[
    \mbox{\Var } X :=
    \frac{1}{n - 1} \, \mbox{\Ex}\left( (X - \mbox{\Ex } X)^2 \right).
\]


\subsection[Laplacian of Gaussian]{\label{sec:laplacian-of-gaussian}%
  \genidx{weighting!contrast using \propername{Laplacian}-of-\propername{Gaussian}}%
  \gensee{contrast weighting using \propername{Laplacian}-of-\propername{Gaussian}}%
         {weighting, contrast using \propername{Laplacian}-of-\propername{Gaussian}}%
  Laplacian of Gaussian}

\genidx{Laplacian-of-Gaussian@\propername{Laplacian}-of-\propername{Gaussian}}%
\gensee{Log@\acronym{LoG}}{\propername{Laplacian}-of-\propername{Gaussian}}%
\begin{sloppypar}
  The \propername{Laplacian}-of-\propername{Gaussian} (\acronym{LoG}) is an operator to detect
  edges in an image.  Sometimes the \acronym{LoG}-operator is also called
  \propername{Marr}-\propername{Hildreth} operator.  A
  \propername{Laplacian}-of-\propername{Gaussian} operator,
  \uref{\hciiwrvigracommonfilters}{\code{vigra::laplacianOfGaussian}} is part of the
  package~\uref{\hciiwrvigra}{\acronym{VIGRA}} that \App{} is built upon and is used for edge
  detection if option~\option{--contrast-edge-scale} is non-zero and
  \option{--contrast-min-curvature} equal to or less than zero.
\end{sloppypar}

Let the \propername{Gaussian} function be
\[
    g(x, y) =
    \frac{1}{2 \pi \sigma^2} \,
    {\exp\left(-\frac{x^2 + y^2}{2 \sigma^2}\right)}
\]
\noindent The parameter~$\sigma$, the argument of option~\option{--contrast-edge-scale}, is the
length scale on which edges are detected by $g(x, y)$.  We apply the \propername{Laplacian}
operator in \propername{Cartesian} coordinates
\[
    \bigtriangleup \equiv \nabla \cdot \nabla =
    \frac{\partial^2}{\partial x^2} + \frac{\partial^2}{\partial y^2}
\]
\noindent to $g(x, y)$, to arrive at a continuous representation of the two-dimensional filter
kernel
\begin{equation}\label{equ:laplacian-of-gaussian}
    k(x, y) = \frac{\xi^2 - 1}{\pi \sigma^4} \exp(-\xi^2),
\end{equation}
\noindent where we have used the dimensionless distance~$\xi$ from the origin
\[
    \xi^2 = \frac{x^2 + y^2}{2 \sigma^2}.
\]

\App{} uses a discrete approximation of $k$ in the convolution with the image.  The operator is
radially symmetric with respect to the origin, which is why we can easily plot it in
\figureName~\ref{fig:laplacian-of-gaussian}, setting $R = \sqrt{x^2 + y^2}$.


\begin{figure}[htbp]
  \ifreferencemanual\begin{maxipage}\fi
  \centering
  \includeimage{laplacian-of-gaussian}
  \ifreferencemanual\end{maxipage}\fi

  \caption[\propername{Laplacian}-of-\propername{Gaussian}]{\label{fig:laplacian-of-gaussian}%
    Plot of the \propername{Laplacian}-of-\propername{Gaussian} function~$k(R)$,
    \equationabbr~\ref{equ:laplacian-of-gaussian}, for $\sigma = 0.5$, using $R = \sqrt{x^2 +
      y^2}$.}
\end{figure}


\noindent See also \uref{\hipriilog}{\acronym{HIPR2}:
  \propername{Laplacian}-of-\propername{Gaussian}}.

Sometimes the \acronym{LoG} is plagued by noise in the input images.  After all, it is a
numerical approximation of the second derivative and deriving always ``roughens'' a function.
The (normalized) mask files relentlessly disclose such problems.  Use
option~\option{--contrast-min-curvature} with a \emph{negative} argument~\metavar{CURVATURE} to
suppress all edges with a curvature below $-\metavar{CURVATURE}$ (which is a positive value).
Check the effects with the mask files and particularly the hard-mask files
(\filename{\val{val:default-hard-mask-template}}) if using option~\option{--hard-mask}.

To indicate the \metavar{CURVATURE} in relative terms, which is particularly comprehensible for
humans, append a percent sign (\sample{\%}).  Try minimum curvatures starting from $-0.5\%$ to
$-3\%$.

\begin{optionsummary}
\item[--contrast-edge-scale] \sectionName~\fullref{opt:contrast-edge-scale}
\item[--contrast-min-curvature] \sectionName~\fullref{opt:contrast-min-curvature}
\item[--contrast-weight] \sectionName~\fullref{opt:contrast-weight}
\item[--hard-mask] \sectionName~\fullref{opt:hard-mask}
\end{optionsummary}


\subsection[Blend \acronym{SDev} and \acronym{LoG}]%
{\label{sec:blend-sdev-and-log}%
\genidx{weighting!contrast using a blend of methods}%
\gensee{contrast weighting using a blend of methods}{weighting, contrast using a blend of methods}%
Blend Standard Deviation and \propername{Laplacian}\hyp{}of\hyp{}\propername{Gaussian}}

\App{} can team the standard deviation computation and Laplacian of Gaussian to deliver the best
of both methods.  Use a \emph{positive} argument~\metavar{CURVATURE} with
option~\option{--contrast-min-curvature} to combine both algorithms.  In this mode of operation
\App{} computes the \acronym{SDev}-weight and the \acronym{LoG}-weight, then uses the
\acronym{LoG} to decide whether to go with that value or prefer the \acronym{SDev} data.  If the
\acronym{LoG} is greater than \metavar{CURVATURE} \App{} uses the weight delivered by the
\acronym{LoG}, otherwise the \acronym{SDev}-weight is rescaled such that its maximum is equal to
\metavar{CURVATURE}, and the scaled \acronym{SDev} is used as weight.

This technique merges the two edge detection methods where they are best.  The \acronym{LoG}
excels with clear edges and cannot be fooled by strong but smooth gradients.  However, it is bad
at detecting faint edges and it is susceptible to noise.  The \acronym{SDev} on the other hand
shines with even the most marginal edges, and resists noise quite well.  Its weakness is that is
is easily deceived by strong and smooth gradients.  Tuning \metavar{CURVATURE} the user can pick
the best threshold for a given set of images.

\begin{optionsummary}
\item[--contrast-edge-scale] \sectionName~\fullref{opt:contrast-edge-scale}
\item[--contrast-min-curvature] \sectionName~\fullref{opt:contrast-min-curvature}
\item[--contrast-weight] \sectionName~\fullref{opt:contrast-weight}
\item[--contrast-window-size] \sectionName~\fullref{opt:contrast-window-size}
\item[--gray-projector] \sectionName~\fullref{opt:gray-projector}
\item[--hard-mask] \sectionName~\fullref{opt:hard-mask}
\end{optionsummary}


\subsection[Scaling and Choice of Mode]{\label{sec:scaling-and-choice}%
  \genidx{scaling of parameters}
  \genidx{mode of operation (\acronym{SDev}, \acronym{LoG}, \dots)}%
  Scaling and Choice of Mode}

Experience has shown that neither the parameters \metavar{EDGESCALE} and \metavar{CURVATURE} nor
the mode of operation (\acronym{SDev}-only, \acronym{LoG}-only, or a blend of both) scales to
different image sizes.  In practice, this means that if you start with a set of reduced size
images, say $2808 \times 1872$~pixels, carefully optimize \metavar{EDGESCALE},
\metavar{CURVATURE} and so on, and find \acronym{LoG}-only the best mode, and then switch to the
original resolution of $5616 \times 3744$~pixels, multiplying (or dividing) the parameters by
four and sticking to \acronym{LoG}-only might \emph{not} result in the best fused image.  For
best quality, perform the parameter optimization and the search for the most appropriate mode at
the final resolution.


%%% Local Variables:
%%% fill-column: 96
%%% End:
