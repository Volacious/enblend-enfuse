%% This file is part of Enblend.
%% Licence details can be found in the file COPYING.


\subsection[Program Flow Charts]{Program Flow And Option Settings
  \label{sec:program-flow-and-option-settings}
  \genidx[\rangebeginlocation]{program flow}
  \genidx{options!program flow}}

Figure~\ref{fig:internal-enblend-flow} depicts \App's internal work
flow\genidx{program flow!internal} and shows what option influences
which part.  \App{} works incrementally.  Thus, no matter whether
starting with the first image or the result of blending any previous
images, it loads the next image.  After blending it the result again
is a single image, which serves as base for the next input image, this
is the next iteration.

\begin{figure}[htbp]
  \begin{maxipage}
    \centering
    \includeimage{internal-enblend-flow}

    \caption[Internal work flow]{\App's internal work flow for the
      ``next'' image.  The defaults are indicated like \sample{[default]}.
      The Optimizer Chain is complicated enough to warrant its own
      chart in
      Figure~\ref{fig:internal-optimizer-chain}.}\label{fig:internal-enblend-flow}
  \end{maxipage}
\end{figure}

\begin{geeknote}
  Figure~\ref{fig:internal-enblend-flow} somewhat simplifies the
  program flow.

  \begin{itemize}
  \item
    The Graph Cut algorithm needs an initial ``guess'' of the seam
    line.  It gets it by running an \acronym{NFT}.

  \item
    If the overlap between the previous image and the next image is
    too small, the ``Scale down mask'' step is skipped and \App{}
    works with the mask in its original size (``fine mask'') no matter
    what the command-line options specify.
  \end{itemize}
\end{geeknote}

\begin{figure}[htbp]
  \begin{maxipage}
    \centering
    \includeimage{internal-optimizer-chain}

    \caption[Optimizer chain]{A closer look the Optimizer Chain of
      Figure~\ref{fig:internal-enblend-flow}.}\label{fig:internal-optimizer-chain}
  \end{maxipage}
\end{figure}

\genidx[\rangeendlocation]{program flow}
