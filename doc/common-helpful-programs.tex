%% This file is part of Enblend.
%% Licence details can be found in the file COPYING.


\chapter[Helpful Programs\commonpart]{\label{sec:helpful-programs}%
  \genidx[\rangebeginlocation]{helpful programs}%
  Helpful Programs And Libraries\commonpart}

Several programs and libraries have proven helpful when working with \App{} or \OtherApp.


\section[Raw Image Conversion]{\label{sec:raw-image-conversion}%
  \genidx{helpful programs!raw image conversion}%
  Raw Image Conversion}

\begin{itemize}
  \label{app:darktable}
\item
  \appidx{Darktable}\uref{\darktableorg}{\application{Darktable}} is an open-source photography
  workflow application and raw-image developer.

  \label{prg:dcraw}
\item
  \prgidx{dcraw}\uref{\cybercomnetdcraw}{\application{DCRaw}} is a universal raw-converter
  written by \propername{David Coffin}.

  \label{app:rawtherapee}
\item
  \appidx{RawTherapee}\uref{\rawtherapeecom}{\application{RawTherapee}} is powerful open-source
  raw converter for Win\asterisk, \mbox{\acronym{MacOS}} and Linux.

  \label{app:ufraw}
\item
  \appidx{UFRaw}\uref{\ufrawsourceforgenet}{\application{UFRaw}} is a raw-converter written by
  \propername{Udi Fuchs} and based on \application{DCRaw} (see above).  It adds a \acronym{GUI}
  (\prgidx{ufraw}\command{ufraw}), versatile batch processing
  (\prgidx{ufraw-batch}\command{ufraw-batch}), and some additional features such as cropping,
  noise reduction with wavelets, and automatic lens-error correction.
\end{itemize}


\section[Image Alignment and Rendering]{\label{sec:image-alignment-and-rendering}%
  \genidx{helpful programs!image alignment}%
  \genidx{helpful programs!image rendering}%
  Image Alignment and Rendering}

\begin{itemize}
  \label{app:hugin}
\item
  \appidx{Hugin}\uref{\huginsourceforgenet}{\application{Hugin}} is a \acronym{GUI} that aligns
  and stitches images.

  It comes with several command-line tools, like for example \prgidx{nona
    \textrm{(Hugin)}}\command{nona} to stitch panorama images,
  \command{align\_image\_stack}\prgidx{align\_image\_stack \textrm{(Hugin)}} to align
  overlapping images for \acronym{HDR} or create focus stacks, and \prgidx{fulla
    \textrm{(Hugin)}}\command{fulla} to correct lens errors.

  \label{app:panotools}
\item
  \appidx{PanoTools}\uref{\panotoolssourceforgenet}{PanoTools} the successor of
  \propername{Helmut Dersch's} \uref{\allinoneeedersch}{original PanoTools} offers a set of
  command-line driven applications to create panoramas.  Most notable are \prgidx{PTOptimizer
    \textrm{(PanoTools)}}\application{PTOptimizer} for control point optimization and
  \prgidx{PTmender \textrm{(PanoTools)}}\application{PTmender}, an image stitcher.
\end{itemize}


\section[Image Manipulation]{\label{sec:image-manipulation}%
  \genidx{helpful programs!image manipulation}%
  Image Manipulation}

\begin{itemize}
  \label{app:cinepaint}
\item
  \appidx{CinePaint}\uref{\cinepaintorg}{\application{CinePaint}} is a branch of an early
  \application{Gimp} forked off at version~1.0.4.  It sports much less features than the current
  \application{Gimp}, but offers 8~bit, 16~bit and~32~bit color channels, \acronym{HDR} (for
  example floating-point \acronym{TIFF}, and \acronym{OpenEXR}), and a tightly integrated color
  management system.

  \label{app:gimp}
\item
  \appidx{Gimp}\uref{\gimporg}{\application{The Gimp}} is a general purpose image manipulation
  program.  At the time of this writing it is still limited to images with only 8~bits per
  channel.

  \label{app:gmic}
\item
  \prgidx{gmic}\uref{\gmicsourceforgenet}{\application{G'Mic}} is an open and full-featured
  framework for image processing, providing several different user interfaces to convert,
  manipulate, filter, and visualize generic image datasets.

  \label{app:imagemagick}\label{app:graphicsmagick}
\item
  Both \appidx{ImageMagick}\uref{\imagemagickorg}{\application{ImageMagick}} and
  \appidx{GraphicsMagick}\uref{\graphicsmagickorg}{\application{GraphicsMagick}} are
  general\hyp{}purpose command\hyp{}line controlled image\hyp{}manipulation programs, for
  example, \prgidx{convert \textrm{(ImageMagick)}}\command{convert}, \prgidx{convert
    \textrm{(ImageMagick)}}\command{display}, \prgidx{convert
    \textrm{(ImageMagick)}}\command{identify}, and \prgidx{convert
    \textrm{(ImageMagick)}}\command{montage}.  \application{GraphicsMagick} bundles most
  \application{ImageMagick} invocations in the single dispatcher call to \prgidx{gm
    \textrm{(GraphicsMagick)}}\command{gm}.
\end{itemize}


\section[High Dynamic Range]{\label{sec:high-dynamic-range}%
  \genidx{helpful programs!High Dynamic Range}%
  \genidx{helpful programs!\acronym{HDR}}%
  High Dynamic Range}

\begin{itemize}
  \label{lib:openexr}
\item
  \genidx{OpenEXR@\acronym{OpenEXR}}\uref{\openexrcom}{OpenEXR} offers libraries and some
  programs to work with the \genidx{EXR@\acronym{EXR}}\acronym{EXR}
  \genidx{HDR@\acronym{HDR}}\acronym{HDR} format, for example the \acronym{EXR} display utility
  \prgidx{exrdisplay \textrm{(OpenEXR)}}\command{exrdisplay}.

  \label{app:psftools}
\item
  \prgidx{PFSTools}\uref{\pfstoolssourceforgenet}{PFSTools} read, write, modify, and tonemap
  high-dynamic range images.
\end{itemize}


\section[Libraries]{\label{sec:helpful-libraries}%
  \genidx{helpful programs!libraries}%
  \genidx{helpful libraries}%
  Major Libraries}

\begin{itemize}
  \label{lib:jpeg}
\item
  \genidx{LibJPEG}\uref{\ijgorg}{LibJPEG} is a library for handling the
  \genidx{JPEG@\acronym{JPEG}}\acronym{JPEG}
  (\gensee{JFIF@\acronym{JFIF}}{\acronym{JPEG}}\acronym{JFIF}) image format.

  \label{lib:png}
\item
  \genidx{LibPNG}\uref{\libpngorg}{LibPNG} is a library that handles the Portable Network
  Graphics (\genidx{PNG@\acronym{PNG}}\acronym{PNG}) image format.

  \label{lib:tiff}
\item
  \genidx{LibTiff}\uref{\remotesensingorglibtiff}{LibTIFF} offers a library and utility programs
  to manipulate the ubiquitous Tagged Image File Format, \genidx{TIFF@\acronym{TIFF}}\acronym{TIFF}.

  The nifty \prgidx{tiffinfo \textrm{(libtiff)}}\command{tiffinfo} command in the LibTIFF
  distribution quickly inquires the most important properties of \acronym{TIFF} files.
\end{itemize}


\section[Meta-Data Handling]{\label{sec:meta-data-handling}%
  \genidx{helpful programs!meta-data handling}%
  Meta-Data Handling}

\begin{itemize}
  \label{app:exiftool}
\item
  \prgidx{exiftool}\uref{\snophyqueensucaexiftool}{EXIFTool} reads and writes
  \genidx{EXIF@\acronym{EXIF}}\acronym{EXIF} meta data.  In particular it copies meta-data from
  one image to another.

  \label{app:littlecms}
\item
  \appidx{LittleCMS}\uref{\littlecmscom}{LittleCMS} is the color-management library used by
  \application{Hugin}, \application{DCRaw}, \application{UFRaw}, \App{}, and \OtherApp.  It
  supplies some binaries, too.  \prgidx{tificc \textrm{(LittleCMS)}}\command{tificc}, an
  \genidx{ICC@\acronym{ICC}}\acronym{ICC} color profile applier, is of particular interest.
\end{itemize}


\section[Camera Firmware Extension]{\label{sec:camera-firmware-extension}%
  \genidx{helpful programs!camera firmware}%
  Camera Firmware Extension}

\begin{itemize}
  \label{app:magiclantern}
\item
  \appidx{Magic Lantern}\uref{\magiclanternfm}{Magic Lantern} is a software add-on that runs
  from the \acronym{SD}~(Secure Digital) or \acronym{CF}~(Compact Flash) card and adds new
  features to cameras of a certain Japanese brand of cameras as for example

  \begin{itemize}
  \item
    \genidx{dual-ISO@dual-\acronym{ISO}}Dual-\acronym{ISO} (more precisely: simultaneous dual
    sensor speed); this operation mode may in fact obviate the need for \application{Enfuse}.

  \item
    Focus stacking

  \item
    \genidx{bracketing!HDR@\acronym{HDR}}%
    \gensee{HDR bracketing@\acronym{HDR}-bracketing}{bracketing, \acronym{HDR}}%
    \acronym{HDR}-bracketing
  \end{itemize}
\end{itemize}

\genidx[\rangeendlocation]{helpful programs}


%%% Local Variables:
%%% fill-column: 96
%%% End:
