%% This file is part of Enblend.
%% Licence details can be found in the file COPYING.


\chapter[Helpful Programs\commonpart]{Helpful Programs And Libraries\commonpart
  \label{sec:helpful-programs}
  \genidx[\rangebeginlocation]{helpful programs}}

Several programs and libraries have proven helpful when working with
\application{Enblend} and \application{Enfuse}.


\section[Raw Image Conversion]{Raw Image Conversion
  \label{sec:raw-image-conversion}
  \genidx{helpful programs!raw image conversion}}

\begin{itemize}
\item
  \uref{\darktableorg}{\application{Darktable}}\appidx{Darktable} is
  an open source photography workflow application and raw developer.

\item
  \uref{\cybercomnetdcraw}{\application{DCRaw}}\prgidx{dcraw} is a
  universal raw-converter written by \propername{David Coffin}.

\item
  \uref{\rawtherapeecom}{\application{RawTherapee}}\appidx{RawTherapee}
  is powerful 64-bit open source raw converter for Win\asterisk, Mac
  OS~X and Linux.

\item
  \uref{\ufrawsourceforgenet}{\application{UFRaw}} is a raw converter
  written by \propername{Udi Fuchs} and based on \application{DCRaw} (see
  right above).  It adds a \acronym{GUI}
  (\command{ufraw}\prgidx{ufraw}), versatile batch processing
  (\command{ufraw-batch}\prgidx{ufraw-batch}), and some additional
  features such as cropping, noise reduction with wavelets, and
  automatic lens error correction.
\end{itemize}


\section[Image Alignment and Rendering]{Image Alignment and Rendering
  \label{sec:image-alignment-and-rendering}
  \genidx{helpful programs!image alignment}
  \genidx{helpful programs!image rendering}}

\begin{itemize}
\item\label{app:hugin}
  \uref{\huginsourceforgenet}{\application{Hugin}}\appidx{Hugin} is
  a \acronym{GUI} that aligns and stitches images.

  It comes with several command line tools, like
  \command{nona}\prgidx{nona \textrm{(Hugin)}} to stitch panorama
  images, \command{align\_image\_stack}\prgidx{align\_image\_stack
    \textrm{(Hugin)}} to align overlapping images for \acronym{HDR} or
  create focus stacks, and \command{fulla}\prgidx{fulla
    \textrm{(Hugin)}} to correct lens errors.

\item\label{app:panotools} \uref{\panotoolssourceforgenet}{PanoTools}
  the successor of \propername{Helmut Dersch's}
  \uref{\allinoneeedersch}{original PanoTools} offers a set of
  command-line driven applications to create panoramas.  Most notable
  are \application{PTOptimizer}\prgidx{PTOptimizer
    \textrm{(PanoTools)}} and
  \application{PTmender}.\prgidx{PTmender \textrm{(PanoTools)}}
\end{itemize}


\section[Image Manipulation]{Image Manipulation
  \label{sec:image-manipulation}
  \genidx{helpful programs!image manipulation}}

\begin{itemize}
\item\label{app:cinepaint}
  \uref{\cinepaintorg}{\application{CinePaint}}\appidx{CinePaint} is
  a branch of an early \application{Gimp} forked off at version~1.0.4.
  It sports much less features than the current \application{Gimp},
  but offers 8~bit, 16~bit and~32~bit color channels, \acronym{HDR}
  (for example floating-point \acronym{TIFF}, and \acronym{OpenEXR}),
  and a tightly integrated color management system.

\item\label{app:gimp} \uref{\gimporg}{\application{The
    Gimp}}\appidx{Gimp} is a general purpose image manipulation
  program.  At the time of this writing it is still limited to images
  with only 8~bits per channel.

\item\label{app:gmic}
  \uref{\gmicsourceforgenet}{\application{G'Mic}}\prgidx{gmic} is
  an open and full-featured framework for image processing, providing
  several different user interfaces to convert, manipulate, filter,
  and visualize generic image datasets.

\item\label{app:imagemagick} Both
  \uref{\imagemagickorg}{ImageMagick}\appidx{ImageMagick} and
  \uref{\graphicsmagickorg}{GraphicsMagick}\appidx{GraphicsMagick} are
  general\hyp{}purpose command\hyp{}line controlled image manipulation
  programs, for example, \command{convert},\prgidx{convert
    \textrm{(ImageMagick)}} \command{display},\prgidx{display
    \textrm{(ImageMagick)}} \command{identify},\prgidx{identify
    \textrm{(ImageMagick)}} and \command{montage}.\prgidx{montage
    \textrm{(ImageMagick)}} GraphicsMagick bundles most ImageMagick
  invocations in the single dispatcher call to \command{gm}.\prgidx{gm
    \textrm{(GraphicsMagick)}}
\end{itemize}


\section[High Dynamic Range]{High Dynamic Range
  \label{sec:high-dynamic-range}
  \genidx{helpful programs!High Dynamic Range}
  \genidx{helpful programs!\acronym{HDR}}}

\begin{itemize}
\item\label{lib:openexr} \uref{\openexrcom}{OpenEXR} offers libraries
  and some programs to work with the \acronym{EXR} \acronym{HDR}
  format, for example the \acronym{EXR} display utility
  \command{exrdisplay}.\prgidx{exrdisplay \textrm{(OpenEXR)}}

\item\label{app:psftools}
  \uref{\pfstoolssourceforgenet}{PFSTools} read, write,
  modify, and tonemap high-dynamic range images.
\end{itemize}


\section[Libraries]{Major Libraries
  \label{sec:helpful-libraries}
  \genidx{helpful programs!libraries}
  \genidx{helpful libraries}}

\begin{itemize}
\item\label{lib:jpeg} \uref{\ijgorg}{LibJPEG}\genidx{LibJPEG} is a
  library for handling the \acronym{JPEG} (\acronym{JFIF}) image
  format.

\item\label{lib:png} \uref{\libpngorg}{LibPNG}\genidx{LibPNG} is a
  library that handles the Portable Network Graphics (\acronym{PNG})
  image format.

\item\label{lib:tiff}
  \uref{\remotesensingorglibtiff}{LibTIFF}\genidx{LibTiff} offers a
  library and utility programs to manipulate the ubiquitous Tagged
  Image File Format, \acronym{TIFF}.

  The nifty \command{tiffinfo}\prgidx{tiffinfo \textrm{(libtiff)}}
  command in the LibTIFF distribution quickly inquires the most
  important properties of \acronym{TIFF} files.
\end{itemize}


\section[Meta-Data Handling]{Meta-Data Handling
  \label{sec:meta-data-handling}
  \genidx{helpful programs!meta-data handling}}

\begin{itemize}
\item\label{app:exiftool}
  \uref{\snophyqueensucaexiftool}{EXIFTool}\prgidx{exiftool} reads
  and writes \acronym{EXIF} meta data.  In particular it copies
  meta-data from one image to another.

\item\label{app:littlecms}
  \uref{\littlecmscom}{LittleCMS}\prgidx{tifficc
    \textrm{(LittleCMS)}} is the color management library used by
  \application{Hugin}, \application{DCRaw}, \application{UFRaw},
  \application{Enblend}, and \application{Enfuse}.  It supplies some
  binaries, too.  \command{tifficc}, an \acronym{ICC} color profile
  applier, is of particular interest.
\end{itemize}


\section[Camera Firmware Extension]{Camera Firmware Extension
  \label{sec:camera-firmware-extension}
  \genidx{helpful programs!camera firmware}}

\begin{itemize}
\item\label{app:magiclantern} \uref{\magiclanternfm}{Magic
  Lantern}\appidx{Magic Lantern} is a software add-on that runs from
  the \acronym{SD}~(Secure Digital) or \acronym{CF}~(Compact Flash)
  card and adds new features to cameras of a certain Japanese brand as
  for example

  \begin{itemize}
  \item
    Dual-\acronym{ISO} (more precisely: simultaneous dual sensor
    speed); this operation mode may in fact obviate the need for
    \application{Enfuse}.

  \item
    Focus stacking

  \item
    \acronym{HDR}-bracketing
  \end{itemize}
\end{itemize}

\genidx[\rangeendlocation]{helpful programs}
