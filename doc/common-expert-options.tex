%% This file is part of Enblend.
%% Licence details can be found in the file COPYING.


\subsection[Expert Options]{\label{sec:expert-options}%
  \genidx[\rangebeginlocation]{expert options}%
  \genidx{options!expert}%
  Expert Options}% Not a \commonpart!


%%% IMPLEMENTATION NOTE: This file is included in both manuals, but
%%% rendered differently by conditional inclusion; Enblend and Enfuse
%%% have different sets of ``Expert Options''.  Hence, the name prefix
%%% `common-', but no `\commonpart' sign.


\ifenblend
  Control inner workings of \App{} and in particular the interpretation of images.
\fi

\ifenfuse
  Control inner workings of \App{} and the reading\slash writing of weight masks.
\fi


\begin{codelist}
  \label{opt:fallback-profile}%
  \optidx[\defininglocation]{--fallback-profile}%
\item[--fallback-profile=\metavar{PROFILE-FILENAME}]\itemend
  Use the \acronym{ICC} profile in \metavar{PROFILE-FILENAME} instead of the default
  \acronym{sRGB}.\genidx{profile!fallback}\gensee{fallback profile}{profile, fallback} This
  option only is effective if the input images come \emph{without} color profiles \emph{and}
  blending is not performed in the trivial luminance interval\genidx{luminance interval!trivial}
  or \genidx{RGB-cube@\acronym{RGB}-cube}\acronym{RGB}-cube.

  Compare option~\flexipageref{\option{--blend-colorspace}}{opt:blend-colorspace} and
  \chapterName~\fullref{sec:color-profiles} on color profiles.


  \label{opt:layer-selector}%
  \optidx[\defininglocation]{--layer-selector}%
  \genidx{layer selection}%
\item[--layer-selector=\metavar{ALGORITHM}]\itemend
  Override the standard layer selector algorithm~\sample{\val{val:layer-selector}}.

  \App{} offers the following algorithms:

  \begin{codelist}
    \genidx{layer selection!all layers}%
  \item[all-layers]\itemend
    Select all layers in all images.

    \genidx{layer selection!first layer}%
  \item[first-layer]\itemend
    Select only first layer in each multi-layer image.  For single-layer images this is the same
    as \sample{all-layers}.

    \genidx{layer selection!last layer}
  \item[last-layer]\itemend
    Select only last layer in each multi-layer image.  For single-layer images this is the same
    as \sample{all-layers}.

    \genidx{layer selection!largest-layer}
  \item[largest-layer]\itemend
    Select largest layer in each multi-layer image, where the ``largeness'', this is the size is
    defined by the product of the layer width and its height.  The channel width of the layer is
    ignored.  For single-layer images this is the same as \sample{all-layers}.

    \genidx{layer selection!no layer}
  \item[no-layer]\itemend
    Do not select any layer in any image.

    This algorithm is useful to temporarily exclude some images in response files.
  \end{codelist}


\ifenfuse
  \label{opt:load-masks}%
  \optidx[\defininglocation]{--load-masks}%
  \item[\itempar{--load-masks~\textrm{(\oldstylefirst~form)}
      \\ --load-masks=\metavar{SOFT-MASK-TEMPLATE}~\textrm{(\oldstylesecond~form)}
      \\ --load-masks=\metavar{SOFT-MASK-TEMPLATE}:\feasiblebreak
      \metavar{HARD-MASK-TEMPLATE}~\textrm{(\oldstylethird~form)}}]\itemend
    \genidx{mask!loading}\gensee{load mask}{mask, loading}Load masks from images instead of
    computing them.

    The masks must be grayscale images.

    \begin{sloppypar}
      First form: Load all soft-weight masks from files that were previously saved with
      option~\option{--save-masks}.  If option~\option{--hard-mask} is effective only load hard
      masks.  The respective defaults are \genidx{mask!filename template}\gensee{filename
        template}{mask, filename template}\mbox{\code{\val{val:default-soft-mask-template}}} and
      \mbox{\code{\val{val:default-hard-mask-template}}}. In the second form,
      \metavar{SOFT\hyp{}MASK\hyp{}TEMPLATE} defines the names of the soft-mask files.  In the
      third form, \metavar{HARD\hyp{}MASK\hyp{}TEMPLATE} additionally defines the names of the
      hard-mask files.  See option~\option{--save-masks} below for the description of mask
      templates.
    \end{sloppypar}

    Options~\option{--load-masks} and~\option{--save-masks} are mutually exclusive.
\fi


  \label{opt:parameter}%
  \optidx[\defininglocation]{--parameter}%
\item[--parameter=\metavar{KEY}\optional{=\metavar{VALUE}}\optional{:\dots}]\itemend
  Set a \metavar{KEY}-\metavar{VALUE} pair, where \metavar{VALUE} is optional.  This option is
  cumulative.  Separate multiple pairs with the usual numeric delimiters.

  This option has the negated form \optidx[\defininglocation]{--no-parameter}%
  \sample{--no-parameter}, which takes one or more \metavar{KEY}s and removes them from the list
  of defined parameters.  The special key~\sample{*} deletes all parameters at once.

  Parameters allow the developers to change the internal workings of \App{} without the need to
  recompile or relink.


\ifenblend
    \label{opt:pre-assemble}%
    \optidx[\defininglocation]{--pre-assemble}%
    \shoptidx{-a}{--pre-assemble}%
  \item[\itempar{-a \\ --pre-assemble}]\itemend
    \genidx{preassemble}\gensee{assemble}{preasselble}Pre-assemble non-overlapping images before
    each blending iteration.

    \genidx{blending!sequential}\gensee{sequential blending}{blending, sequential}This overrides
    the default behavior which is to blend the images sequentially in the order given on the
    command line.  \App{} will use fewer blending iterations, but it will do more work in each
    iteration.

    This option has the negated form \optidx[\defininglocation]{--no-pre-assemble}%
    \sample{--no-pre-assemble}, which restores the default.
\fi


  \label{opt:prefer-gpu}%
  \optidx[\defininglocation]{--prefer-gpu}%
\item[--prefer-gpu=\optional{\metavar{PLATFORM}:}\metavar{DEVICE}
  \restrictednote{\acronym{OpenCL}-enabled versions only.}]\itemend

  Direct \App{} towards a particular \genidx{OpenCL}\acronym{OpenCL}
  \genidx{OpenCL!device}\metavar{DEVICE} on the first autodetected
  \genidx{OpenCL!platform}\metavar{PLATFORM} or directly to the given
  \metavar{PLATFORM}\slash\metavar{DEVICE} combination.  Use the numbers of platform and device
  found either with

  \begin{terminal}
    \$ \app{} --verbose --version
  \end{terminal}
  or
  \begin{terminal}
    \$ \app{} --show-gpu-info
  \end{terminal}

  Note that this option only selects \acronym{GPU}-devices; it does not \emph{activate} any, use
  option~\flexipageref{\option{--gpu}}{opt:gpu} for that.

  When told to employ the \acronym{GPU} with \sample{--gpu}, by default \App{} uses the first
  device on the first autodetected platform it finds via queries of the \acronym{OpenCL}
  subsystem, where neither the device, nor the platform may be the ones the user wants.  Usually
  she will select the device with the highest performance, maximum possible number of
  work-items, and largest associated memory.


\ifenblend
    \label{opt:x}%
    \optidx[\defininglocation]{-x}%
  \item[-x]
    \genidx{result!checkpoint}Checkpoint\genidx{checkpoint results} partial results to the output
    file after each blending step.
\fi


\ifenfuse
    \label{opt:save-masks}%
    \optidx[\defininglocation]{--save-masks}%
  \item[\itempar{--save-masks~\textrm{(\oldstylefirst~form)}
      \\ --save-masks=\metavar{SOFT-MASK-TEMPLATE}~\textrm{(\oldstylesecond~form)}
      \\ --save-masks=\metavar{SOFT-MASK-TEMPLATE}:\feasiblebreak
      \metavar{HARD-MASK-TEMPLATE}~\textrm{(\oldstylethird~form)}}]\itemend
    \genidx{mask!save}\gensee{save mask}{mask, save}Save the generated weight masks to image
    files.

    \begin{sloppypar}
      First form: Save all soft-weight masks in files.  If option~\option{--hard-mask} is
      effective also save the hard masks.  The defaults are \genidx{mask!filename
        template}\gensee{filename template}{mask, filename
        template}\code{\val{val:default-soft-mask-template}} and
      \code{\val{val:default-hard-mask-template}}. In the second form,
      \metavar{SOFT\hyp{}MASK\hyp{}TEMPLATE} defines the names of the soft-mask files.  In the
      third form, \metavar{HARD\hyp{}MASK\hyp{}TEMPLATE} additionally defines the names of the
      hard-mask files.
    \end{sloppypar}

    \genidx{save mask!only}\gensee{only save mask}{save mask only}\App{} will stop after saving
    all masks unless option~\option{--output} is given, too.  With both options given, this is,
    \sample{--save-masks} and \sample{--output}, \App{} saves all masks and then proceeds to
    fuse the output image.

    Both \metavar{SOFT\hyp{}MASK\hyp{}TEMPLATE} and \metavar{HARD\hyp{}MASK\hyp{}TEMPLATE}
    define templates that are expanded for each mask file.  In a template a percent sign
    (\sample{\%}) introduces a variable part.  All other characters are copied literally.
    Lowercase letters refer to the name of the respective input file, whereas uppercase ones
    refer to the name of the output file.  \tableName~\fullref{tab:mask-template-characters}
    lists all variables.

    A fancy mask filename template could look like \sample{\%D/mask-\%02n-\%f.tif}.  It puts the
    mask files into the same directory as the output file (\sample{\%D}), generates a two-digit
    index (\sample{\%02n}) to keep the mask files nicely sorted, and decorates the mask filename
    with the name of the associated input file (\sample{\%f}) for easy recognition.

    Options~\option{--load-masks} and~\option{--save-masks} are mutually exclusive.
\fi
\end{codelist}


\ifenfuse
  %% This file is part of Enblend.
%% Licence details can be found in the file COPYING.


\begin{table}
  \centering
  \begin{tabular}{cp{.8\linewidth}}
    \hline
    \multicolumn{1}{c|}{Format} & \multicolumn{1}{c}{Interpretation} \\
    \hline\extraheadingsep
    \genidx{mask!template character!\%@\sample{\%}}\code{\%\%} &
    Produces a literal \sample{\%}-sign. \\


    \genidx{mask!template character!i@\sample{i}}\code{\%i} & Expands to the index of the mask
    file starting at zero.  \sample{\%i} allows for setting a pad character or a width
    specification:

    \begin{literal}
      \% \metavar{PAD} \metavar{WIDTH} i
    \end{literal}

    \metavar{PAD} is either \sample{0} or any punctuation character; the default pad character
    is \sample{0}.  \metavar{WIDTH} is an integer specifying the minimum width of the number.
    The default is the smallest width given the number of input images, this is 1 for
    2--9~images, 2 for 10--99~images, 3 for 100--999~images, and so on.

    Examples: \sample{\%i}, \sample{\%02i}, or \sample{\%\_4i}. \\


    \genidx{mask!template character!n@\sample{n}}\code{\%n} & Expands to the number of the mask
    file starting at one.  Otherwise it behaves identically to \sample{\%i}, including pad
    character and width specification. \\


    \genidx{mask!template character!p@\sample{p}}\code{\%p} & This is the full name (path,
    filename, and extension) of the input file associated with the mask.

    Example: If the input file is called \filename{/home/luser/snap/img.jpg}, \sample{\%p}
    expands to \filename{/home/luser/snap/img.jpg}, or shorter: \sample{\%p} \result{}
    \filename{/home/luser/snap/img.jpg}. \\


    \genidx{mask!template character!P@\sample{P}}\code{\%P} & This is the full name of the
    output file. \\


    \genidx{mask!template character!d@\sample{d}}\code{\%d} & Is replaced with the directory
    part of the associated input file.

    Example (cont.): \sample{\%d} \result{} \filename{/home/luser/snap}. \\


    \genidx{mask!template character!D@\sample{D}}\code{\%D} & Is replaced with the directory
    part of the output file. \\


    \genidx{mask!template character!b@\sample{b}}\code{\%b} & Is replaced with the non-directory
    part (often called ``basename'') of the associated input file.

    Example (cont.): \sample{\%b} \result{} \filename{img.jpg}. \\


    \genidx{mask!template character!B@\sample{B}}\code{\%B} & Is replaced with the non-directory
    part of the output file. \\


    \genidx{mask!template character!f@\sample{f}}\code{\%f} & Is replaced with the filename
    without path and extension of the associated input file.

    Example (cont.): \sample{\%f} \result{} \filename{img}. \\


    \genidx{mask!template character!F@\sample{F}}\code{\%F} & Is replaced with the filename
    without path and extension of the output file. \\


    \genidx{mask!template character!e@\sample{e}}\code{\%e} & Is replaced with the extension
    (including the leading dot) of the associated input file.

    Example (cont.): \sample{\%e} \result{} \filename{.jpg}. \\


    \genidx{mask!template character!E@\sample{E}}\code{\%E} & Is replaced with the extension of
    the output file.
  \end{tabular}

  \caption[Mask template characters]{\label{tab:mask-template-characters}%
    \genidx[\summarylocation]{mask!template character}%
    Special format characters to control the generation of mask filenames.  Uppercase letters
    refer to the output filename and lowercase ones to the input files.}
\end{table}


%%% Local Variables:
%%% fill-column: 96
%%% End:

\fi

\genidx[\rangeendlocation]{expert options}


%%% Local Variables:
%%% fill-column: 96
%%% End:
