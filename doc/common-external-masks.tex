%% This file is part of Enblend.
%% Licence details can be found in the file COPYING.


\section[External Masks]{\label{sec:external-masks}%
  \genidx{external mask}%
  \genidx{mask!external}%
  External Masks}

In the usual workflow \application{Enblend} and \application{Enfuse} generate the blending and
fusing masks according to the command-line options and the input images and then they
immediately use these masks for blending or fusing the output image.

Sometimes more control over the masks is needed or wanted.  To this end, both applications
provide the option pair \option{--load-masks} and \option{--save-masks}.  See
\chapterName~\fullref{sec:invocation}, for detailed explanations of both options.  With the help
of these options the processing can be broken up into two steps:

\begin{enumerate}
  \optidx{--save-masks}
  \optidx{--output}
\item
  Save masks with \option{--save-masks}.  Generate masks and save them into image files.

  Avoid option~\option{--output} unless the blended or fused image at this point is necessary.

  \optidx{--load-masks}
\item
  Load masks with \option{--load-masks}.  Load masks from files and then blend or fuse the final
  image with the help of the loaded masks.
\end{enumerate}

In between these two steps the user may apply whatever transformation to the mask files, as long
as their geometries and offsets remain the same.  Thus the ``Combine Images'' box of
\figureName~\ref{fig:photographic-workflow} becomes three activities as is depicted in
\figureName~\ref{fig:external-mask-workflow}.

\begin{figure}
  \begin{maxipage}
    \centering
    \includeimage{external-mask-workflow}
  \end{maxipage}

  \caption[External mask workflow]{\label{fig:external-mask-workflow}%
    \genidx{workflow!external mask}
    Workflow for externally modified masks.}
\end{figure}

To further optimize this kind of workflow, both \application{Enblend} and \application{Enfuse}
stop after mask generation if option~\option{--save-masks} is given, but \emph{no output file}
is specified with the \option{--output} option.  This way the time for pyramid generation,
blending, fusing, and writing the final image to disk is saved, as well as no output image gets
generated.

Note that options~\option{--save-masks} and \option{--load-masks} cannot be used simultaneously.


%%% Local Variables:
%%% fill-column: 96
%%% End:
