%% This file is part of Enblend.
%% Licence details can be found in the file COPYING.


\chapter[Understanding Masks\commonpart]{\label{sec:understanding-masks}%
  \genidx{mask}%
  \genidx[\rangebeginlocation]{understanding masks}%
  Understanding Masks\commonpart}

\genidx{mask!binary}%
\gensee{binary mask}{mask, binary}%
\genidx{mask!weight}%
A binary mask indicates for every pixel of an image if this pixel must be considered in further
processing, or ignored.  For a weight mask, the value of the mask determines how much the pixel
contributes, zero again meaning ``no contribution''.

Masks arise in two places: as part of the input files and as separate files, showing the actual
pixel weights prior to image blendung or fusion.  We shall explore both occurrences in the next
sections.


\section[Masks In Input Files]{\label{sec:masks-in-input-files}%
  \genidx{mask!input files}%
  \gensee{input mask}{mask, input files}%
  Masks In Input Files}

Each of the input files for \App{} and \OtherApp{} can contain its own mask.  Both applications
interpret them as binary masks no matter how many bits per image pixel they contain.

Use \application{ImageMagick}'s \prgidx{identify \textrm{(ImageMagick)}}\command{identify} (see
\exampleName~\ref{ex:using-identify}) or, for \acronym{TIFF} files only, \prgidx{tiffinfo
  \textrm{(libtiff)}}\command{tiff\/info} (see \exampleName~\ref{ex:using-tiffinfo}) to inquire
quickly whether a file contains a mask.  \appendixName~\fullref{sec:helpful-programs} shows
where to find these programs on the web.

\begin{exemplar}
  \begin{terminal}
    \$ identify -version \\
    Version: ImageMagick 6.7.7-10 2014-03-08 Q16 http://www.imagemagick.org \\
    Copyright: Copyright (C) 1999-2012 ImageMagick Studio LLC \\
    Features: OpenMP \\
    ~ \\
    \$ identify -format "\%f \%m \%wx\%h \%r \%q-bit" image-0000.tif \\
    image-0000.tif TIFF 917x1187 DirectClass sRGB Matte 16-bit \\
    ~~~~~~~~~~~~~~~~~~~~~~~~~~~~~~~~~~~~~~~~~~~~~~\caret\caret\caret\caret\caret \\
    ~~~~~~~~~~~~~~~~~~~~~~~~~~~~~~~~~~~~~~~~~~~~~~\textit{mask}
  \end{terminal}

  \caption[Using \command{identify}.]%
          {\label{ex:using-identify}%
            Using \command{identify} to find out about the mask in \filename{image-0000.tif}.
            \sample{Matte} indicates the existence of a mask.}
\end{exemplar}

\begin{exemplar}
  \begin{terminal}
    \$ tiffinfo \\
    LIBTIFF, Version 4.0.2 \\
    Copyright (c) 1988-1996 Sam Leffler \\
    Copyright (c) 1991-1996 Silicon Graphics, Inc. \\
    \dots \\
    ~ \\
    \$ tiffinfo image-0000.tif \\
    TIFF Directory at off set 0x3a8182 (3834242) \\
    ~~Subfile Type: (0 = 0x0) \\
    ~~Image Width: 917 Image Length: 1187 \\
    ~~Resolution: 150, 150 pixels/inch \\
    ~~Position: 0, 0 \\
    ~~Bits/Sample: 8 \\
    ~~Sample Format: unsigned integer \\
    ~~Compression Scheme: PackBits \\
    ~~Photometric Interpretation: RGB color \\
    ~~Extra Samples: 1<unassoc-alpha>~~~~~~~~~~~~~~~\textit{mask} \\
    ~~Orientation: row 0 top, col 0 lhs \\
    ~~Samples/Pixel: 4~~~~~~~~~~~~~~~~~~~~~~~~~~~~~~\textit{R, G, B, and mask} \\
    ~~Rows/Strip: 285 \\
    ~~Planar Configuration: single image plane \\
    ~~ImageFullWidth: 3000 \\
    ~~ImageFullLength: 1187
  \end{terminal}

  \caption[Using \command{tiff\/info}.]%
          {\label{ex:using-tiffinfo}%
            Using \command{tiff\/info} to find out about the mask in \filename{image-0000.tif}.
            Here the line \sample{Extra Samples} indicates one extra sample per pixel, which is
            interpreted as an unassociated alpha-channel; \App{} and \OtherApp{} interpret this
            as mask.  The second hint \command{tiff\/info} gives is in \sample{Samples/Pixel},
            where -- for a \acronym{RGB}-image -- the $4 = 3 + 1$ tells about the extra
            channel.}
\end{exemplar}

\genidx{matte}%
\genidx{extra samples}%
\genidx{alpha}%
The ``Matte'' part of the image class and the ``Extra Samples'' line tell us that the file
features a mask.  Also, many interactive image manipulation programs show the mask as a separate
channel, sometimes called ``alpha''.  There, the white (high mask value) parts of the mask
enable pixels and black (low mask value) parts suppress them.

The multitude of terms all describing the concept of a mask is confusing.

\begin{description}
  \genidx{mask}%
\item[Mask]\itemend
  A mask defines a selection of pixels.  A value of zero represents an unselected pixel.  The
  maximum value (``white'') represents a selected pixel and the values between zero and the
  maximum are partially selected pixels.  See \uref{\gimpsavvy}{Gimp-Savy.}

  \genidx{channel!alpha}%
  \gensee{alpha channel}{channel, alpha}%
\item[Alpha Channel]\itemend
  The alpha channel stores the transpacency value for each pixel, typically in the range from
  zero to one.  A value of zero means the pixel is completely transparent, thus does not
  contribute to the image.  A value of one on the other hand means the pixel is completely
  opaque.

  \genidx{matte}%
\item[Matte]\itemend
  The notion ``matte'' as used by \application{ImageMagick} refers to an inverted alpha channel,
  more precisely: $1 - \mbox{alpha}$.  See
  \uref{\imagemagickorgusagechannelstrans}{\application{ImageMagick}} for further explanations.
\end{description}

\App{} and \OtherApp{} only consider pixels that have an associated mask value other than zero.
If an input image does not have an alpha channel, \application{Enblend} warns and assumes a mask
of all non-zero values, that is, it will use every pixel of the input image for fusion.

\prgidx{nona \textrm{(Hugin)}}%
Stitchers like \command{nona} add a mask to their output images.

Sometimes it is helpful to manually modify a mask before fusion.  For example to suppress
unwanted objects (insects and cars come into mind) that moved across the scene during the
exposures.  If the masks of all input images are black at a certain position, the output image
will have a hole in that position.


\section[Weight Mask Files]{\label{sec:weight-mask-files}%
  \genidx{mask!weight}%
  \gensee{weight!mask}{mask, weight}%
  Weight Mask Files}

\fixme{Show some weight masks and explain them.}

\genidx[\rangeendlocation]{understanding masks}


%%% Local Variables:
%%% fill-column: 96
%%% End:
