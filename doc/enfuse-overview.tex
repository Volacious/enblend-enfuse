%% This file is part of Enblend.
%% Licence details can be found in the file COPYING.


\chapter[Overview]{Overview
  \label{sec:overview}
  \genidx[\rangebeginlocation]{overview}}

\begin{sloppypar}
  \App{} merges overlapping images using the
  \propername{Mertens}\genidx{Mertens@\propername{Mertens, Tom}}-%
  \propername{Kautz}\genidx{Kautz@\propername{Kautz, Jan}}-%
  \propername{Van Reeth}\genidx{Reeth@\propername{Van Reeth, Frank}}
  exposure fusion\genidx{exposure fusion} algorithm.\footnotemark{}
  This is a quick way for example to blend differently exposed images
  into a nice output image, without producing intermediate
  high-dynamic range\genidx{high dynamic range}
  (\acronym{HDR}\gensee{HDR@\acronym{HDR}}{high dynamic range}) images
  that are then tonemapped to a viewable image.  This simplified
  process often works much better than tonemapping algorithms.
\end{sloppypar}
\footnotetext{\propername{Tom Mertens}, \propername{Jan Kautz}, and
  \propername{Frank van Reeth}, ``Exposure Fusion'', Proceedings of
  the 15$^{\mathrm{th}}$ Pacific Conference on Computer Graphics and
  Applications, pages~382--390.}

\App{} can also be used to build extended
depth-of-field\genidx{depth-of-field}
(\acronym{DoF}\gensee{DoF@\acronym{DoF}}{depth-of-field}) images by
blending a focus stack.

The idea is that pixels in the input images are weighted according to
qualities such as, for example, proper exposure, good local contrast,
or high saturation.  These weights determine how much a given pixel
will contribute to the final image.

A \propername{Burt}\genidx{Burt@\propername{Burt, Peter J.}}-%
\propername{Adelson}\genidx{Adelson@\propername{Adelson, Edward H.}}
multiresolution spline\genidx{multiresolution spline}%
\genidx{Burt-Adelson@\propername{Burt}-\propername{Adelson}
  multiresolution spline} blender\footnotemark{} is used to combine
the images according to the weights.  The multiresolution blending
ensures that transitions between regions where different images
contribute are difficult to spot.%
\footnotetext{\propername{Peter J. Burt} and \propername{Edward
    H. Adelson}, ``A Multiresolution Spline With Application to Image
  Mosaics'', \acronym{ACM} Transactions on Graphics,
  \abbreviation{Vol}.~2, \abbreviation{No}.~4, October~1983,
  pages~217--236.}

\App{} uses up to four criteria to judge the quality of a pixel:

\begin{description}
  \genidx{weighting!exposure}%
  \gensee{exposure weighting}{weighting, exposure}%
\item[Exposure]\itemend The exposure criteria favors pixels with
  luminance close to the middle of the range.  These pixels are
  considered better exposed than those with high or low luminance
  levels.

  \genidx{weighting!saturation}%
  \gensee{saturation weighting}{weighting, saturation}%
\item[Saturation]\itemend The saturation criteria favors
  highly-saturated pixels.  Note that saturation is only defined for
  color pixels.

  \genidx{weighting!local contrast}%
  \gensee{contrast weighting}{weighting, local contrast}
  \gensee{local contrast weighting}{weighting, local contrast}
\item[Local Contrast]\itemend The contrast criteria favors pixels
  inside a high-contrast neighborhood.  \App{} can use standard
  deviation, \propername{Laplacian} magnitude, or a blend of both as
  local contrast measure.

  \genidx{weighting!local entropy}%
  \gensee{entropy weighting}{weighting, local entropy}%
  \gensee{local entropy weighting}{weighting, local entropy}%
\item[Local Entropy]\itemend The entropy criteria prefers pixels
  inside a high-entropy neighborhood.  In addition, \App{} allows the
  user to mitigate the problem of noisy images when using entropy
  weighting by setting a black threshold.
\end{description}

\noindent See Table~\ref{tab:default-weights} for the default weights
of these criteria.

For the concept of pixel weighting, and details on the different
weighting functions, see Chapter~\ref{sec:weighting-functions}.

Adjust how much importance is given to each criterion by setting the
weight parameters on the command line.  For example, if you set
\begin{literal}
  --exposure-weight=1.0 --saturation-weight=0.5
\end{literal}
\App{} will favor well-exposed pixels over highly-saturated pixels
when blending the source images.  The effect of these parameters on
the final result will not always be clear in advance.  The quality of
the result is subject to your artistic interpretation.  Playing with
the weights may or may not give a more pleasing result.  The authors
encourage the users to experiment, perhaps using
down-sized\footnote{Down-sizing (also called ``down-sampling'') with a
  good interpolator reduces noise, which might not be desired to judge
  the image quality of the original-size image.  Cropping might be an
  alternative, though.} or cropped images for speed.

\App{} expects but does not require each input image to have an alpha
channel\genidx{channel!alpha}\gensee{alpha channel}{channel, alpha}.
By setting the alpha values of pixels to zero, users can manually
remove those pixels from consideration when blending.  If an input
image lacks an alpha channel, \App{} will issue a warning and continue
assuming all pixels should contribute to the final output.  Any alpha
value other than zero is interpreted as ``this pixel should contribute
to the final image''.

\App{} reads all layers of multi-layer images%
\gensee{multi-layer image}{image, multi-layer}%
\genidx{image!multi-layer},
like, for example, multi\hyp{}directory%
\gensee{multi-directory \acronym{TIFF}}{\acronym{TIFF}, multi-directory}%
\genidx{TIFF@\acronym{TIFF}!multi-directory}
\acronym{TIFF} images\footnote{Use utilities like, e.g.,
  \command{tiffcopy}\prgidx{tiffcopy} and
  \command{tiffsplit}\prgidx{tiffsplit} of LibTIFF to manipulate
  multi-directory \acronym{TIFF} images.  See
  Appendix~\fullref{sec:helpful-programs}.}.  The input images are
processed in the order they appear on the command line.  Multi-layer
images are processed from the first layer to the last before \App{}
considers the next image on the command line.

Find out more about \App{} on its
\uref{\sourceforgenet}{SourceForge}\genidx{SourceForge}
\uref{\enblendsourceforgenet}{web page}.

\genidx[\rangeendlocation]{overview}
