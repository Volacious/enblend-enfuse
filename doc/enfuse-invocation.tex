%% This file is part of Enblend.
%% Licence details can be found in the file COPYING.


\chapter[Invocation]{\label{sec:invocation}%
  \genidx[\rangebeginlocation]{invocation}%
  \genidx{command-line}%
  Invocation}

%HEVEA\cutdef{section}


\noindent Fuse the sequence of images \metavar{INPUT}\dots{} into a single \metavar{IMAGE}.

\smallskip

\code{\app} \optional{\metavar{OPTIONS}} \optional{\option{--output=}\metavar{IMAGE}}
\metavar{INPUT}\dots

\smallskip

\genidx{filename!literal}%
\gensee{literal filename}{filename, literal}%
\genidx{file!response}%
\gensee{response file}{file, response}%
\noindent \metavar{INPUT} images are either specified literally or via so-called response files
(see \sectionName~\ref{sec:response-files}).  The latter are an alternative to specifying image
filenames on the command line.  If omitted, the name of the output \metavar{IMAGE} defaults to
\filename{\val{val:default-output-filename}}.

%% This file is part of Enblend.
%% Licence details can be found in the file COPYING.


\section[Image Requirements]{\label{sec:image-requirements}%
  \genidx{requirements!image}%
  \gensee{image requirements}{requirements, image}%
  Input Image Requirements}

All input images for \App{} must comply with the following requirements.

\begin{itemize}
\item
  The images overlap.

\item
  The images agree on their number of bits-per-channel, i.e., their ``depth'':
  \begin{itemize}
  \item
    \code{uint8},
  \item
    \code{uint16},
  \item
    \code{float},
  \item
    etc.
  \end{itemize}

  See option~\option{--depth} below for an explanation of different (output) depths.

\item
  \App{} understands the images' filename extensions as well as their file formats.

  You can check the supported extensions and formats by calling \App{} with
  option~\option{--show-image-formats}.
\end{itemize}

Moreover, there are some good practices, which are not enforced by the application, but almost
certainly deliver superior results.

\begin{itemize}
\item
  Either all files lack an \acronym{ICC} profile, or all images are supplied with the
  \emph{same} \acronym{ICC} profile.

\item
  If the images' meta-data contains resolution information (``\acronym{DPI}''), it is the same
  for all pictures.
\end{itemize}


%%% Local Variables:
%%% fill-column: 96
%%% End:

%% This file is part of Enblend.
%% Licence details can be found in the file COPYING.


\section[Command-Line Options]{\label{sec:options}%
  \genidx[\rangebeginlocation]{options}%
  Command-Line Options}

In this section we group the options as the command-line help
\begin{terminal}
  \$ \app{} --help
\end{terminal}
does and sort them alphabetically within their groups.  For an alphabetic list of \emph{all}
options consult the \flexipageref{Option Index}{sec:option-index}.

\appcmd{} accepts arguments to any option in uppercase as well as in lowercase letters.  For
example, \sample{deflate}, \sample{Deflate} and \sample{DEFLATE} as arguments to the
\option{--compression}~option described below all instruct \appcmd{} to use the
\propername{Deflate} compression scheme.  This manual denotes all arguments in lowercase for
consistency.

%% This file is part of Enblend.
%% Licence details can be found in the file COPYING.


\subsection[Common Options\commonpart]{\label{sec:common-options}%
  \genidx[\rangebeginlocation]{common options}%
  \genidx{options!common}%
  Common Options\commonpart}

Common options control some overall features of \App.  They are called ``common'' because they
are used most often.  However, in fact, \App{} and \OtherApp{} do have these options in common.

\begin{codelist}
  \label{opt:compression}%
  \optidx[\defininglocation]{--compression}%
  \genidx{output!file compression}%
  \genidx{compression}%
\item[--compression=\metavar{COMPRESSION}]\itemend
  Write a compressed output file.  The default is not to compress the output image.

  Depending on the output file format, \App{} accepts different values for
  \metavar{COMPRESSION}.

  \begin{description}
    \genidx{compression!\acronym{JPEG}}%
    \gensee{JPEG@\acronym{JPEG} compression}{compression}%
  \item[\acronym{JPEG} format.]\itemend
    The compression either is a literal integer or a keyword\hyp{}option combination.

    \begin{codelist}
      \gensee{JPEG@\acronym{JPEG} quality level}{compression}%
    \item[\metavar{LEVEL}]\itemend
      Set \acronym{JPEG} quality~\metavar{LEVEL}, where \metavar{LEVEL} is an integer that
      ranges from 0--100.

    \item[jpeg\optional{:\metavar{LEVEL}}]\itemend
      Same as above; without the optional argument just switch on standard \acronym{JPEG}
      compression.

      \genidx{compression!arithmetic \acronym{JPEG}}%
      \gensee{arithmetic \acronym{JPEG} compression}{compression}%
    \item[jpeg-arith\optional{:\metavar{LEVEL}}]\itemend
      Switch on arithmetic \acronym{JPEG} compression.  With optional argument set the
      arithmetic compression~\metavar{LEVEL}, where \metavar{LEVEL} is an integer that ranges
      from 0--100.
    \end{codelist}

  \item[\acronym{TIF} format.]\itemend
    Here, \metavar{COMPRESSION} is one of the keywords:

    \begin{codelist}
    \item[none]\itemend Do not compress.  This is the default.

      \genidx{compression!deflate}%
      \gensee{deflate compression}{compression}%
    \item[deflate]\itemend
      Use the \propername{Deflate} compression scheme also called
      \acronym{ZIP}-in-\acronym{TIFF}.  \propername{Deflate} is a lossless data compression
      algorithm that uses a combination of the \acronym{LZ77} algorithm and \propername{Huffman}
      coding.

      \genidx{compression!\acronym{JPEG} of \acronym{TIFF}}%
    \item[jpeg\optional{:\metavar{LEVEL}}]\itemend
      Use \acronym{JPEG} compression.  With optional argument set the
      compression~\metavar{LEVEL}, where \metavar{LEVEL} is an integer that ranges from 0--100.

      \genidx{compression!\acronym{LZW}}%
      \gensee{LZW@\acronym{LZW} compression}{compression}%
    \item[lzw]\itemend
      Use \propername{Lempel}-\propername{Ziv}-\propername{Welch} (\acronym{LZW}) adaptive
      compression scheme.  \acronym{LZW} compression is lossless.

      \genidx{compression!packbits}%
      \gensee{packbits compression}{compression}%
    \item[packbits]\itemend
      Use \propername{PackBits} compression scheme.  \propername{PackBits} is a particular
      variant of run-length compression; it is lossless.
    \end{codelist}

  \item[Any other format.]\itemend
    Other formats do not accept a \metavar{COMPRESSION} setting.  However, the underlying
    \uref{\hciiwrvigra}{\acronym{VIGRA}} library automatically compresses \filename{png}-files
    with the \propername{Deflate} method.  (\acronym{VIGRA} is the image manipulaton library
    upon which \App{} is based.)
  \end{description}


  \label{opt:gpu}%
  \optidx[\defininglocation]{--gpu}%
  \genidx{graphics processing unit}%
  \gensee{GPU@\acronym{GPU}}{graphics processing unit}%
  \genidx{central processing unit}%
  \gensee{CPU@\acronym{CPU}}{central processing unit}%
  \genidx{OpenCL@\acronym{OpenCL}}%
\item[--gpu \restrictednote{\acronym{OpenCL}-enabled versions
    only.}]\itemend
  Employ one of the graphics processing units (\acronym{GPU}s) to perform computing instead of
  the central processing units (\acronym{CPU}s) alone.  \App{} must have been compiled with
  support for \acronym{OpenCL} access to the \acronym{GPU} for this feature.

  Depending on the input images, the options passed to \App{}, and the relative performance of
  the \acronym{CPU}s to the \acronym{GPU}s this option may or may not increase performance.

  This option enables \acronym{GPU} processing on the selected \acronym{GPU}s.  To choose a
  particular \acronym{GPU} or override \App's default choice use
  \flexipageref{\sample{--prefer-gpu}}{opt:prefer-gpu}.

  Negate this option with \sample{--no-gpu}\optidx[\defininglocation]{--no-gpu}.


  \label{opt:levels}%
  \optidx[\defininglocation]{--levels}%
  \shoptidx{-l}{--levels}%
\item[\itempar{-l \metavar{LEVELS} \\ --levels=\metavar{LEVELS}}]\itemend
  Use at most this many \genidx{levels!pyramid}\gensee{pyramid levels}{levels,
    pyramid}\metavar{LEVELS} for pyramid\footnotemark{} blending if \metavar{LEVELS} is
  positive, or reduce the maximum number of levels used by $-\metavar{LEVELS}$ if
  \metavar{LEVELS} is negative; \sample{auto} or \sample{automatic} restore the default, which
  is to use the maximum possible number of levels for each overlapping region.

  \footnotetext{As \genidx{Jackson, Daniel@\propername{Jackson, Daniel}}\propername{Dr.~Daniel
      Jackson} correctly \ahref{\stargatewikiathetomb}{noted}, actually, it is not a pyramid:
    ``Ziggaurat, it's a \ahref{\wikipediaziggaurat}{Ziggaurat}.''}% FIXME

  The number of levels used in a pyramid controls the balance between local and global image
  features (contrast, saturation,~\dots) in the blended region.  Fewer levels emphasize local
  features and suppress global ones.  The more levels a pyramid has, the more global features
  will be taken into account.

  \begin{geeknote}
    As a guideline, remember that each new level works on a linear scale twice as large as the
    previous one.  So, the zeroth layer, the original image, obviously defines the image at
    single-pixel scale, the first level works at two-pixel scale, and generally, the $n$-th
    level contains image data at $2^n$-pixel scale.  This is the reason why an image of
    $\mbox{width} \times \mbox{height}$~pixels cannot be deconstructed into a pyramid of more
    than $\lfloor \log_2(\min(\mbox{width}, \mbox{height})) \rfloor$ levels.

    If too few levels are used, ``halos'' around regions of strong local feature variation can
    show up.  On the other hand, if too many levels are used, the image might contain too much
    global features.  Usually, the latter is not a problem, but is highly desired.  This is the
    reason, why the default is to use as many levels as is possible given the size of the
    overlap regions.  \App{} may still use a smaller number of levels if the geometry of the
    overlap region demands.
  \end{geeknote}

  Positive values of \metavar{LEVELS} limit the maximum number of pyramid levels.  Depending on
  the size and geometry of the overlap regions this may or may not influence any pyramid.
  Negative values of \metavar{LEVELS} reduce the number of pyramid levels below the maximum no
  matter what the actual maximum is and thus always influence all pyramids.  Use \sample{auto}
  or \sample{automatic} as \metavar{LEVELS} to restore the automatic calculation of the maximum
  number of levels.

  The valid range of the absolute value of \metavar{LEVELS} is \val{val:minimum-pyramid-levels}
  to \val{val:maximum-pyramid-levels}.

  \label{opt:output}%
  \optidx[\defininglocation]{--output}%
  \shoptidx{-o}{--output}%
  \genidx{filename!output}%
  \gensee{output filename}{filename, output}%
  \gensee{output filename!default}{filename, output, default}%
\item[\itempar{-o \metavar{FILE} \\ --output=\metavar{FILE}}]\itemend
  Place \appdoes{}ed output image in \metavar{FILE}.  If \sample{--output} is omitted, \App{}
  writes the resulting image to \genidx{filename!output!default}%
  \gensee{default output filename}{filename, output, default}%
  \filename{\val{val:default-output-filename}}.


  \label{opt:verbose}%
  \optidx[\defininglocation]{--verbose}%
  \shoptidx{-v}{--verbose}%
  \genidx{level!verbosity}%
  \gensee{verbosity level}{level, verbosity}%
\item[\itempar{-v \optional{\metavar{LEVEL}} \\ --verbose\optional{=\metavar{LEVEL}}}]\itemend
  Without an argument, increase the verbosity of progress reporting.  Giving more
  \option{--verbose}~options will make \App{} more verbose; see
  \sectionName~\fullref{sec:finding-out-details} for an examplary output.  Directly set a
  verbosity level with a non-negative integral~\metavar{LEVEL}.
  \tableName~\ref{tab:verbosity-levels} shows the messages available at a particular
  \metavar{LEVEL}.

  \begin{table}[htbp]
    \centering
    \begin{tabular}{cp{.75\linewidth}}
      \hline
      \multicolumn{1}{c|}{Level} & \multicolumn{1}{c}{Messages} \\
      \hline\extraheadingsep
      0 & only warnings and errors \\
      1 & reading and writing of images \\
      2 & mask generation, pyramid, and blending \\
      3 & reading of response files, color conversions \\
      4 & image sizes, bounding boxes and intersection sizes \\
      5 & \restrictednote{\application{Enblend} only.} detailed
      information on the optimizer runs \\
      6 & estimations of required memory in selected processing steps \\
    \end{tabular}

    \caption[Verbosity levels]{\label{tab:verbosity-levels}Verbosity levels of \app{}; each
      level includes all messages of the lower levels.}
  \end{table}

  The default verbosity level of \App{} is
  \val{val:default-verbosity-level}.
\end{codelist}

\genidx[\rangeendlocation]{common options}


%%% Local Variables:
%%% fill-column: 96
%%% End:

%% This file is part of Enblend.
%% Licence details can be found in the file COPYING.


\subsection[Advanced Options\commonpart]{\label{sec:advanced-options}%
  \genidx[\rangebeginlocation]{advanced options}%
  \genidx{options!advanced}%
  Advanced Options\commonpart}

Advanced options control e.g.\ the channel depth, color model, and the cropping of the output
image.

\begin{codelist}
  \label{opt:blend-colorspace}%
  \optidx[\defininglocation]{--blend-colorspace}%
  \genidx{colorspace!blend}%
  \gensee{blend colorspace}{colorspace, blend}%
  \genidx{color appearance model}%
\item[--blend-colorspace=\metavar{COLORSPACE}]\itemend
  Force blending in selected \metavar{COLORSPACE}.  Given well matched images this option should
  not change the output image much.  However, if \App{} must blend vastly different colors (as
  e.g.\ anti-colors) the resulting image heavily depends on the \metavar{COLORSPACE}.

  Usually, \App{} chooses defaults depending on the input images:

  \begin{itemize}
    \genidx{profile!\acronym{ICC}}%
    \gensee{ICC@\acronym{ICC} profile}{profile, \acronym{ICC}}%
    \genidx{colorspace!\acronym{CIELUV}}%
  \item
    For grayscale or color input images \emph{with} \acronym{ICC}~profiles the default is to use
    \acronym{CIELUV}~colorspace.

    \genidx{color cube!\acronym{RGB}}%
    \gensee{RGB@\acronym{RGB} color cube}{color cube, \acronym{RGB}}%
  \item
    Images \emph{without} color profiles and floating-point images are blended in the trivial
    luminance interval (grayscale) or \acronym{RGB}-color cube by default.
  \end{itemize}

  On the order of fast to slow computation, \App{} supports the following blend colorspaces.

  \begin{description}
  \item[\itempar{\code{identity} \\ \code{id} \\ \code{unit}}]\itemend
    Compute blended colors in a na\"ive way sidestepping any dedicated colorspace.
    \begin{itemize}
      \genidx{luminance interval!trivial}%
    \item
      Use trivial, 1-dimensional luminance interval (see
      \equationabbr~\fullref{equ:trivial-luminance-blend}) for grayscale images and

      \genidx{color cube!\acronym{RGB}}%
      \genidx{sRGB@\acronym{sRGB}}%
    \item
      for color images utilize 3-dimensional \acronym{RGB}-cube (see
      \equationabbr~\fullref{equ:trivial-rgb-blend}) spanned by the input \acronym{ICC}~profile
      or \acronym{sRGB} if no profiles are present.  In the latter case, consider passing
      option~\flexipageref{\option{--fallback-profile}}{opt:fallback-profile} to force a
      different profile than \acronym{sRGB} upon all input images.
    \end{itemize}

    \genidx{colorspace!\acronym{CIEL*a*b*}}%
    \gensee{CIEL*a*b*@\acronym{CIEL*a*b*} colorspace}{colorspace, \acronym{CIEL*a*b*}}%
  \item[\itempar{\code{lab} \\ \code{cielab} \\ \code{lstar} \\ \code{l-star}}]\itemend
    Blend pixels in the \acronym{CIEL*a*b*} colorspace.

    \genidx{colorspace!\acronym{CIEL*u*v*}}%
    \gensee{CIEL*u*v*@\acronym{CIEL*u*v*} colorspace}{colorspace, \acronym{CIEL*u*v*}}%
  \item[\itempar{\code{luv} \\ \code{cieluv}}]\itemend
    Blend pixels in the \acronym{CIEL*u*v*} colorspace.

    \genidx{colorspace!\acronym{CIECAM02}}%
    \gensee{CIECAM02@\acronym{CIECAM02} colorspace}{colorspace, \acronym{CIECAM02}}%
  \item[\itempar{\code{ciecam} \\ \code{ciecam02} \\ \code{jch}}]\itemend
    Blend pixels in the \acronym{CIECAM02} colorspace.
  \end{description}

  \ifenblend
    \begin{restrictedmaterial}{\application{Enblend} only.}
      \genidx{optimizer!seam-line}Please keep in mind that by using different blend colorspaces,
      blending may not only change the colors of the output image, but \application{Enblend} may
      choose different seam line routes as some seam-line optimizers are guided by image
      differences, which are different when viewed in different colorspaces.
    \end{restrictedmaterial}
  \fi


  \label{opt-ciecam}%
  \optidx{--ciecam}%
  \shoptidx{-c}{--ciecam}%
\item[\itempar{-c \\ --ciecam}]\itemend
  Deprecated.  Use \sample{--blend-colorspace=ciecam} instead.  To emulate the negated
  option~\option{--no-ciecam}\optidx{--no-ciecam} use \sample{--blend-colorspace=identity}.

  \label{opt:depth}%
  \optidx[\defininglocation]{--depth}%
  \shoptidx{-d}{--depth}%
  \genidx{bits per channel}%
  \gensee{channel!width}{channel, depth}%
  \genidx{channel!depth}%
\item[\itempar{-d \metavar{DEPTH} \\ --depth=\metavar{DEPTH}}]\itemend
  Force the number of bits per channel and the numeric format of the output image, this is, the
  \metavar{DEPTH}.  The number of bits per channel is also known as ``channel width'' or
  ``channel depth''.

  \genidx{requantization}\App{} always uses a smart way to change the channel depth to assure
  highest image quality at the expense of memory, whether requantization is implicit because of
  the output format or explicit through option~\option{--depth}.

  \begin{itemize}
  \item
    If the output-channel depth is larger than the input-channel depth of the input images, the
    input images' channels are widened to the output channel depth immediately after loading,
    that is, as soon as possible.  \App{} then performs all blending operations at the
    output-channel depth, thereby preserving minute color details which can appear in the
    blending areas.

  \item
    If the output-channel depth is smaller than the input-channel depth of the input images, the
    output image's channels are narrowed only right before it is written to the output
    \metavar{FILE}, that is, as late as possible.  Thus the data benefits from the wider input
    channels for the longest time.
  \end{itemize}

  All \metavar{DEPTH} specifications are valid in lowercase as well as uppercase letters.  For
  integer format, use

  \begin{description}
  \item[\itempar{\code{8} \\ \code{uint8}}]\itemend
    Unsigned 8~bit; range: $0\dots255$

  \item[\code{int16}]\itemend
    Signed 16~bit; range: $-32768\dots32767$

  \item[\itempar{\code{16} \\ \code{uint16}}]\itemend
    Unsigned 16~bit; range: $0\dots65535$

  \item[\code{int32}]\itemend
    Signed 32~bit; range: $-2147483648\dots2147483647$

  \item[\itempar{\code{32} \\ \code{uint32}}]\itemend
    Unsigned 32~bit; range: $0\dots4294967295$
  \end{description}

  %% Minimum positive normalized value: 2^(2 - 2^k)
  %% Epsilon: 2^(1 - n)
  %% Maximum finite value: (1 - 2^(-n)) * 2^(2^k)
  For floating-point format, use

  \begin{description}
    \genidx{IEEE754@\acronym{IEEE754}!single precision float}%
    \gensee{single precision float (\acronym{IEEE754})}{\acronym{IEEE754}, single precision float}%
  \item[\itempar{\code{r32} \\ \code{real32} \\ \code{float}}]\itemend
    %% IEEE single: 32 bits, n = 24, k = 32 - n - 1 = 7
    \acronym{IEEE754} single precision floating-point, 32~bit wide, 24~bit significant;

    \begin{itemize}
    \item
      Minimum normalized value: \semilog{1.2}{-38}
    \item
      Epsilon: \semilog{1.2}{-7}
    \item
      Maximum finite value: \semilog{3.4}{38}
    \end{itemize}

    \genidx{IEEE754@\acronym{IEEE754}!double precision float}%
    \gensee{double precision float (\acronym{IEEE754})}{\acronym{IEEE754}, double precision float}%
  \item[\itempar{\code{r64} \\ \code{real64} \\ \code{double}}]\itemend
    %% IEEE double: 64 bits, n = 53, k = 64 - n - 1 = 10
    \acronym{IEEE754} double precision floating-point, 64~bit wide, 53~bit significant;

    \begin{itemize}
    \item
      Minimum normalized value: \semilog{2.2}{-308}
    \item
      Epsilon: \semilog{2.2}{-16}
    \item
      Maximum finite value: \semilog{1.8}{308}
    \end{itemize}
  \end{description}

  If the requested \metavar{DEPTH} is not supported by the output file format, \App{} warns and
  chooses the \metavar{DEPTH} that matches best.

  \restrictednote{Versions with \acronym{OpenEXR} read\slash write support only.}
  \noindent The \genidx{OpenEXR@\acronym{OpenEXR}!data format}\acronym{OpenEXR} data format is
  treated as \acronym{IEEE754}~float internally.  Externally, on disk, \acronym{OpenEXR} data is
  represented by ``half'' precision floating-point numbers.

  %% ILM half: 16 bits, n = 10, k = 16 - n - 1 = 5
  \genidx{OpenEXR@\acronym{OpenEXR}!half precision
    float}\uref{\openexrcomfeatures}{\acronym{OpenEXR}} \gensee{half precision float
    (\acronym{OpenEXR})}{\acronym{OpenEXR}, half precision float}half precision floating-point,
  16~bit wide, 10~bit significant;

  \begin{itemize}
  \item
    Minimum normalized value: \semilog{9.3}{-10}
  \item
    Epsilon: \semilog{2.0}{-3}
  \item
    Maximum finite value: \semilog{4.3}{9}
  \end{itemize}


  \label{opt:f}%
  \optidx[\defininglocation]{-f}%
\item[-f \metavar{WIDTH}x\metavar{HEIGHT}%
  \optional{+x\metavar{XOFFSET}+y\metavar{YOFFSET}}]\itemend
  Ensure that the minimum \genidx{size!canvas}``canvas'' \genidx{output image!set size}size of
  the output image is at least \metavar{WIDTH}\classictimes\metavar{HEIGHT}.  Optionally specify
  the \metavar{XOFFSET} and \metavar{YOFFSET} of the canvas, too.

  This option only is useful when the input images are cropped \acronym{TIFF} files, such as
  those produced by \prgidx{nona \textrm{(Hugin)}}\command{nona}.

  Note that option~\option{-f} neither rescales the output image, nor shrinks the canvas size
  below the minimum size occupied by the union of all input images.


  \label{opt:g}%
  \optidx[\defininglocation]{-g}%
  \genidx{alpha channel!associated}%
  \gensee{associated alpha channel}{alpha channel, associated}%
  \gensee{unassociated alpha channel}{alpha channel, associated}%
\item[-g]
  Save alpha channel as ``associated''. See the
  \uref{\awaresystemsbeextrasamples}{\acronym{TIFF} documentation} for an explanation.

  \appidx{Gimp}\application{The Gimp} before version~2.0 and
  \appidx{Cinepaint}\application{CinePaint} (see \appendixName~\fullref{sec:helpful-programs})
  exhibit unusual behavior when loading images with unassociated alpha channels.  Use
  option~\option{-g} to work around this problem.  With this flag \App{} will create the output
  image with the ``associated alpha tag'' set, even though the image is really unassociated
  alpha.


  \label{opt:wrap}%
  \optidx[\defininglocation]{--wrap}%
  \shoptidx{-w}{--wrap}%
  \genidx{wrap around}%
\item[\itempar{-w \optional{\metavar{MODE}} \\ --wrap\optional{=\metavar{MODE}}}]\itemend
  Blend around the boundaries of the panorama, or ``wrap around''.

  As this option significantly increases memory usage and computation time only use it, if the
  panorama will be

  \begin{itemize}
  \item
    consulted for any kind measurement, this is, all boundaries must match as accurately as
    possible, or

  \item
    printed out and the boundaries glued together, or

    \genidx{virtual reality}%
    \gensee{VR@\acronym{VR}}{virtual reality}%
  \item
    fed into a virtual reality~(\acronym{VR}) generator, which creates a seamless environment.
  \end{itemize}

  \noindent Otherwise, always avoid this option!

  With this option \App{} treats the set of input images (panorama) of width~$w$ and height~$h$
  as an infinite data structure, where each pixel~$P(x, y)$ of the input images represents the
  set of pixels~$S_P(x, y)$.

  \begin{geeknote}
    Solid-state physicists will be reminded of the
    \ahref{\wikipediabornvonkarman}%
          {\genidx{Born@\propername{Born, Max}}\propername{Born}-%
            \genidx{Karman@\propername{von~K\'arm\'an, Theodore}}\propername{von~K\'arm\'an}
            boundary condition}.
  \end{geeknote}

  \metavar{MODE} takes the following values:

  \begin{codelist}
  \item[\itempar{none \\ open}]\itemend
    This is a ``no-op''; it has the same effect as not giving \sample{--wrap} at all.  The set
    of input images is considered open at its boundaries.

  \item[horizontal]\itemend
    Wrap around horizontally:
    \[
    S_P(x, y) = \{P(x + m w, y): m \in Z\}.
    \]

    This is useful for \genidx{panorama!360\angulardegree!horizontal}%
    \gensee{360@360\angulardegree{}!horizontal panorama}{panorama, 360\angulardegree}
    360\angulardegree{} horizontal panoramas as it eliminates the left and right borders.

  \item[vertical]\itemend
    Wrap around vertically:
    \[
    S_P(x, y) = \{P(x, y + n h): n \in Z\}.
    \]

    This is useful for 360\angulardegree\genidx{panorama!360\angulardegree!vertical}%
    \gensee{360@360\angulardegree{}!vertical panorama}{panorama, 360\angulardegree}
    vertical panoramas as it eliminates the top and bottom borders.

  \item[\itempar{both \\ horizontal+vertical
      \\ vertical+horizontal}]\itemend
    Wrap around both horizontally and vertically:
    \[
    S_P(x, y) = \{P(x + m w, y + n h): m, n \in Z\}.
    \]

    In this mode, both left and right borders, as well as top and bottom borders, are
    eliminated.
  \end{codelist}

  Specifying \sample{--wrap} without \metavar{MODE} selects horizontal
  wrapping.
\end{codelist}

\genidx[\rangeendlocation]{advanced options}


%%% Local Variables:
%%% fill-column: 96
%%% End:

%% This file is part of Enblend.
%% Licence details can be found in the file COPYING.


\subsection[Fusion Options]{Fusion Options
  \label{sec:fusion-options}
  \genidx[\rangebeginlocation]{fusion options}
  \genidx{options!fusion}}

Fusion options define the proportion to which each input image's pixel
contributes to the output image.

\begin{codelist}
  \label{opt:contrast-weight}%
  \optidx[\defininglocation]{--contrast-weight}%
\item[--contrast-weight=\metavar{WEIGHT}]\itemend Sets the relative
  \metavar{WEIGHT} of high local-contrast pixels.%
  \genidx{weight!local contrast}%
  \gensee{contrast weight}{weight, local contrast}%
  \gensee{local contrast weight}{weight, local contrast}

  Valid range: $\val{val:minimum-weight-contrast} \leq
  \metavar{WEIGHT} \leq \val{val:maximum-weight-contrast}$, default:
  \val{val:default-weight-contrast}.

  See Sections~\ref{sec:local-contrast-weighting} and
  \ref{sec:expert-options}, option~\option{--contrast-window-size}.


  \label{opt:entropy-weight}%
  \optidx[\defininglocation]{--entropy-weight}%
\item[--entropy-weight=\metavar{WEIGHT}]\itemend Sets the relative
  \metavar{WEIGHT} of high local entropy pixels.%
  \genidx{weight!entropy}\gensee{entropy weight}{weight, entropy}

  Valid range: $\val{val:minimum-weight-entropy} \leq \metavar{WEIGHT}
  \leq \val{val:maximum-weight-entropy}$, default:
  \val{val:default-weight-entropy}.

  See Sections~\ref{sec:expert-options} and
  \ref{sec:local-entropy-weighting},
  options~\option{--entropy-window-size} and
  \option{--entropy-cutoff}.


  \label{opt:exposure-optimum}%
  \optidx[\defininglocation]{--exposure-optimum}%
  \genidx{exposure!optimum}%
  \gensee{optimum exposure}{exposure, optimum}%
\item[--exposure-optimum=\metavar{OPTIMUM}]\itemend Determine at what
  normalized exposure value the \metavar{OPTIMUM} exposure of the
  input images is.  This is, set the position of the maximum of the
  exposure weight curve.  Use this option to fine-tune exposure
  weighting.

  Valid range: $\val{val:minimum-exposure-optimum} \leq
  \metavar{OPTIMUM} \leq \val{val:maximum-exposure-optimum}$, default:
  \val{val:default-exposure-optimum}.


  \label{opt:exposure-weight}%
  \optidx[\defininglocation]{--exposure-weight}%
  \genidx{exposure!weight}%
  \gensee{weight!exposure}{exposure, weight}%
\item[--exposure-weight=\metavar{WEIGHT}]\itemend Set the relative
  \metavar{WEIGHT} of the ``well-exposedness'' criterion as defined by
  the chosen exposure weight function (see
  option~\option{--exposure-weight-function} below).  Increasing this
  weight relative to the others will make well-exposed pixels
  contribute more to the final output.

  Valid range: $\val{val:minimum-weight-exposure} \leq
  \metavar{WEIGHT} \leq \val{val:maximum-weight-exposure}$, default:
  \val{val:default-weight-exposure}.

  See Section~\fullref{sec:exposure-weighting}.


  \label{opt:exposure-width}%
  \optidx[\defininglocation]{--exposure-width}%
  \genidx{exposure!width}%
  \gensee{width of exposure weight curve}{exposure, width}%
\item[--exposure-width=\metavar{WIDTH}]\itemend Set the characteristic
  \metavar{WIDTH} (\acronym{FWHM}\genidx{FWHM@\acronym{FWHM}}) of the
  exposure weight function.  Low numbers give less weight to pixels
  that are far from the user-defined optimum
  (option~\option{--exposure-optimum}) and vice versa.  Use this
  option to fine-tune exposure weighting (See
  Section~\fullref{sec:exposure-weighting}).

  Valid range: $\metavar{WIDTH} > \val{val:minimum-exposure-width}$,
  default: \val{val:default-exposure-width}.


  \label{opt:hard-mask}%
  \optidx[\defininglocation]{--hard-mask}%
  \genidx{mask!hard}%
  \gensee{hard mask}{mask, hard}%
\item[--hard-mask]\itemend Force hard blend masks on the finest scale.
  This is the opposite flag of option~\option{--soft-mask}.

  This blending mode avoids averaging of fine details (only) at the
  expense of increasing the noise.  However it considerably improves
  the sharpness of focus stacks.  Blending with hard masks has only
  proven useful with focus stacks.

  See also Section~\ref{sec:fusion-options} and
  options~\option{--contrast-weight} as well as
  ~\option{--contrast-window-size} above.


  \label{opt:saturation-weight}%
  \optidx[\defininglocation]{--saturation-weight}%
  \genidx{saturation weight}%
  \gensee{weight!saturation}{saturation weight}%
\item[--saturation-weight=\metavar{WEIGHT}]\itemend Set the relative
  \metavar{WEIGHT} of high-saturation pixels.  Increasing this weight
  makes pixels with high saturation contribute more to the final
  output.

  Valid range: $\val{val:minimum-weight-saturation} \leq
  \metavar{WEIGHT} \leq \val{val:maximum-weight-saturation}$, default:
  \val{val:default-weight-saturation}.

  Saturation weighting is only defined for color images; see
  Section~\ref{sec:saturation-weighting}.


  \label{opt:soft-mask}%
  \optidx[\defininglocation]{--soft-mask}%
  \genidx{mask!soft}%
  \gensee{soft mask}{mask, soft}%
\item[--soft-mask]\itemend Consider all masks when fusing.  This is
  the default.
\end{codelist}

\genidx[\rangeendlocation]{fusion options}

%% This file is part of Enblend.
%% Licence details can be found in the file COPYING.


\subsection[Expert Options]{\label{sec:expert-options}%
  \genidx[\rangebeginlocation]{expert options}%
  \genidx{options!expert}%
  Expert Options}% Not a \commonpart!


%%% IMPLEMENTATION NOTE: This file is included in both manuals, but
%%% rendered differently by conditional inclusion; Enblend and Enfuse
%%% have different sets of ``Expert Options''.  Hence, the name prefix
%%% `common-', but no `\commonpart' sign.


\ifenblend
  Control inner workings of \App{} and in particular the interpretation of images.
\fi

\ifenfuse
  Control inner workings of \App{} and the reading\slash writing of weight masks.
\fi


\begin{codelist}
  \label{opt:fallback-profile}%
  \optidx[\defininglocation]{--fallback-profile}%
\item[--fallback-profile=\metavar{PROFILE-FILENAME}]\itemend
  Use the \acronym{ICC} profile in \metavar{PROFILE-FILENAME} instead of the default
  \acronym{sRGB}.\genidx{profile!fallback}\gensee{fallback profile}{profile, fallback} This
  option only is effective if the input images come \emph{without} color profiles \emph{and}
  blending is not performed in the trivial luminance interval\genidx{luminance interval!trivial}
  or \genidx{RGB-cube@\acronym{RGB}-cube}\acronym{RGB}-cube.

  Compare option~\flexipageref{\option{--blend-colorspace}}{opt:blend-colorspace} and
  \chapterName~\fullref{sec:color-profiles} on color profiles.


  \label{opt:layer-selector}%
  \optidx[\defininglocation]{--layer-selector}%
  \genidx{layer selection}%
\item[--layer-selector=\metavar{ALGORITHM}]\itemend
  Override the standard layer selector algorithm~\sample{\val{val:layer-selector}}.

  \App{} offers the following algorithms:

  \begin{codelist}
    \genidx{layer selection!all layers}%
  \item[all-layers]\itemend
    Select all layers in all images.

    \genidx{layer selection!first layer}%
  \item[first-layer]\itemend
    Select only first layer in each multi-layer image.  For single-layer images this is the same
    as \sample{all-layers}.

    \genidx{layer selection!last layer}
  \item[last-layer]\itemend
    Select only last layer in each multi-layer image.  For single-layer images this is the same
    as \sample{all-layers}.

    \genidx{layer selection!largest-layer}
  \item[largest-layer]\itemend
    Select largest layer in each multi-layer image, where the ``largeness'', this is the size is
    defined by the product of the layer width and its height.  The channel width of the layer is
    ignored.  For single-layer images this is the same as \sample{all-layers}.

    \genidx{layer selection!no layer}
  \item[no-layer]\itemend
    Do not select any layer in any image.

    This algorithm is useful to temporarily exclude some images in response files.
  \end{codelist}


\ifenfuse
  \label{opt:load-masks}%
  \optidx[\defininglocation]{--load-masks}%
  \item[\itempar{--load-masks~\textrm{(\oldstylefirst~form)}
      \\ --load-masks=\metavar{SOFT-MASK-TEMPLATE}~\textrm{(\oldstylesecond~form)}
      \\ --load-masks=\metavar{SOFT-MASK-TEMPLATE}:\feasiblebreak
      \metavar{HARD-MASK-TEMPLATE}~\textrm{(\oldstylethird~form)}}]\itemend
    \genidx{mask!loading}\gensee{load mask}{mask, loading}Load masks from images instead of
    computing them.

    The masks must be grayscale images.

    \begin{sloppypar}
      First form: Load all soft-weight masks from files that were previously saved with
      option~\option{--save-masks}.  If option~\option{--hard-mask} is effective only load hard
      masks.  The respective defaults are \genidx{mask!filename template}\gensee{filename
        template}{mask, filename template}\mbox{\code{\val{val:default-soft-mask-template}}} and
      \mbox{\code{\val{val:default-hard-mask-template}}}. In the second form,
      \metavar{SOFT\hyp{}MASK\hyp{}TEMPLATE} defines the names of the soft-mask files.  In the
      third form, \metavar{HARD\hyp{}MASK\hyp{}TEMPLATE} additionally defines the names of the
      hard-mask files.  See option~\option{--save-masks} below for the description of mask
      templates.
    \end{sloppypar}

    Options~\option{--load-masks} and~\option{--save-masks} are mutually exclusive.
\fi


  \label{opt:parameter}%
  \optidx[\defininglocation]{--parameter}%
\item[--parameter=\metavar{KEY}\optional{=\metavar{VALUE}}\optional{:\dots}]\itemend
  Set a \metavar{KEY}-\metavar{VALUE} pair, where \metavar{VALUE} is optional.  This option is
  cumulative.  Separate multiple pairs with the usual numeric delimiters.

  This option has the negated form \optidx[\defininglocation]{--no-parameter}%
  \sample{--no-parameter}, which takes one or more \metavar{KEY}s and removes them from the list
  of defined parameters.  The special key~\sample{*} deletes all parameters at once.

  Parameters allow the developers to change the internal workings of \App{} without the need to
  recompile or relink.


\ifenblend
    \label{opt:pre-assemble}%
    \optidx[\defininglocation]{--pre-assemble}%
    \shoptidx{-a}{--pre-assemble}%
  \item[\itempar{-a \\ --pre-assemble}]\itemend
    \genidx{preassemble}\gensee{assemble}{preasselble}Pre-assemble non-overlapping images before
    each blending iteration.

    \genidx{blending!sequential}\gensee{sequential blending}{blending, sequential}This overrides
    the default behavior which is to blend the images sequentially in the order given on the
    command line.  \App{} will use fewer blending iterations, but it will do more work in each
    iteration.

    This option has the negated form \optidx[\defininglocation]{--no-pre-assemble}%
    \sample{--no-pre-assemble}, which restores the default.
\fi


  \label{opt:prefer-gpu}%
  \optidx[\defininglocation]{--prefer-gpu}%
\item[--prefer-gpu=\optional{\metavar{PLATFORM}:}\metavar{DEVICE}
  \restrictednote{\acronym{OpenCL}-enabled versions only.}]\itemend

  Direct \App{} towards a particular \genidx{OpenCL}\acronym{OpenCL}
  \genidx{OpenCL!device}\metavar{DEVICE} on the first autodetected
  \genidx{OpenCL!platform}\metavar{PLATFORM} or directly to the given
  \metavar{PLATFORM}\slash\metavar{DEVICE} combination.  Use the numbers of platform and device
  found either with

  \begin{terminal}
    \$ \app{} --verbose --version
  \end{terminal}
  or
  \begin{terminal}
    \$ \app{} --show-gpu-info
  \end{terminal}

  Note that this option only selects \acronym{GPU}-devices; it does not \emph{activate} any, use
  option~\flexipageref{\option{--gpu}}{opt:gpu} for that.

  When told to employ the \acronym{GPU} with \sample{--gpu}, by default \App{} uses the first
  device on the first autodetected platform it finds via queries of the \acronym{OpenCL}
  subsystem, where neither the device, nor the platform may be the ones the user wants.  Usually
  she will select the device with the highest performance, maximum possible number of
  work-items, and largest associated memory.


\ifenblend
    \label{opt:x}%
    \optidx[\defininglocation]{-x}%
  \item[-x]
    \genidx{result!checkpoint}Checkpoint\genidx{checkpoint results} partial results to the output
    file after each blending step.
\fi


\ifenfuse
    \label{opt:save-masks}%
    \optidx[\defininglocation]{--save-masks}%
  \item[\itempar{--save-masks~\textrm{(\oldstylefirst~form)}
      \\ --save-masks=\metavar{SOFT-MASK-TEMPLATE}~\textrm{(\oldstylesecond~form)}
      \\ --save-masks=\metavar{SOFT-MASK-TEMPLATE}:\feasiblebreak
      \metavar{HARD-MASK-TEMPLATE}~\textrm{(\oldstylethird~form)}}]\itemend
    \genidx{mask!save}\gensee{save mask}{mask, save}Save the generated weight masks to image
    files.

    \begin{sloppypar}
      First form: Save all soft-weight masks in files.  If option~\option{--hard-mask} is
      effective also save the hard masks.  The defaults are \genidx{mask!filename
        template}\gensee{filename template}{mask, filename
        template}\code{\val{val:default-soft-mask-template}} and
      \code{\val{val:default-hard-mask-template}}. In the second form,
      \metavar{SOFT\hyp{}MASK\hyp{}TEMPLATE} defines the names of the soft-mask files.  In the
      third form, \metavar{HARD\hyp{}MASK\hyp{}TEMPLATE} additionally defines the names of the
      hard-mask files.
    \end{sloppypar}

    \genidx{save mask!only}\gensee{only save mask}{save mask only}\App{} will stop after saving
    all masks unless option~\option{--output} is given, too.  With both options given, this is,
    \sample{--save-masks} and \sample{--output}, \App{} saves all masks and then proceeds to
    fuse the output image.

    Both \metavar{SOFT\hyp{}MASK\hyp{}TEMPLATE} and \metavar{HARD\hyp{}MASK\hyp{}TEMPLATE}
    define templates that are expanded for each mask file.  In a template a percent sign
    (\sample{\%}) introduces a variable part.  All other characters are copied literally.
    Lowercase letters refer to the name of the respective input file, whereas uppercase ones
    refer to the name of the output file.  \tableName~\fullref{tab:mask-template-characters}
    lists all variables.

    A fancy mask filename template could look like \sample{\%D/mask-\%02n-\%f.tif}.  It puts the
    mask files into the same directory as the output file (\sample{\%D}), generates a two-digit
    index (\sample{\%02n}) to keep the mask files nicely sorted, and decorates the mask filename
    with the name of the associated input file (\sample{\%f}) for easy recognition.

    Options~\option{--load-masks} and~\option{--save-masks} are mutually exclusive.
\fi
\end{codelist}


\ifenfuse
  %% This file is part of Enblend.
%% Licence details can be found in the file COPYING.


\begin{table}
  \centering
  \begin{tabular}{cp{.8\linewidth}}
    \hline
    \multicolumn{1}{c|}{Format} & \multicolumn{1}{c}{Interpretation} \\
    \hline\extraheadingsep
    \genidx{mask!template character!\%@\sample{\%}}\code{\%\%} &
    Produces a literal \sample{\%}-sign. \\


    \genidx{mask!template character!i@\sample{i}}\code{\%i} & Expands to the index of the mask
    file starting at zero.  \sample{\%i} allows for setting a pad character or a width
    specification:

    \begin{literal}
      \% \metavar{PAD} \metavar{WIDTH} i
    \end{literal}

    \metavar{PAD} is either \sample{0} or any punctuation character; the default pad character
    is \sample{0}.  \metavar{WIDTH} is an integer specifying the minimum width of the number.
    The default is the smallest width given the number of input images, this is 1 for
    2--9~images, 2 for 10--99~images, 3 for 100--999~images, and so on.

    Examples: \sample{\%i}, \sample{\%02i}, or \sample{\%\_4i}. \\


    \genidx{mask!template character!n@\sample{n}}\code{\%n} & Expands to the number of the mask
    file starting at one.  Otherwise it behaves identically to \sample{\%i}, including pad
    character and width specification. \\


    \genidx{mask!template character!p@\sample{p}}\code{\%p} & This is the full name (path,
    filename, and extension) of the input file associated with the mask.

    Example: If the input file is called \filename{/home/luser/snap/img.jpg}, \sample{\%p}
    expands to \filename{/home/luser/snap/img.jpg}, or shorter: \sample{\%p} \result{}
    \filename{/home/luser/snap/img.jpg}. \\


    \genidx{mask!template character!P@\sample{P}}\code{\%P} & This is the full name of the
    output file. \\


    \genidx{mask!template character!d@\sample{d}}\code{\%d} & Is replaced with the directory
    part of the associated input file.

    Example (cont.): \sample{\%d} \result{} \filename{/home/luser/snap}. \\


    \genidx{mask!template character!D@\sample{D}}\code{\%D} & Is replaced with the directory
    part of the output file. \\


    \genidx{mask!template character!b@\sample{b}}\code{\%b} & Is replaced with the non-directory
    part (often called ``basename'') of the associated input file.

    Example (cont.): \sample{\%b} \result{} \filename{img.jpg}. \\


    \genidx{mask!template character!B@\sample{B}}\code{\%B} & Is replaced with the non-directory
    part of the output file. \\


    \genidx{mask!template character!f@\sample{f}}\code{\%f} & Is replaced with the filename
    without path and extension of the associated input file.

    Example (cont.): \sample{\%f} \result{} \filename{img}. \\


    \genidx{mask!template character!F@\sample{F}}\code{\%F} & Is replaced with the filename
    without path and extension of the output file. \\


    \genidx{mask!template character!e@\sample{e}}\code{\%e} & Is replaced with the extension
    (including the leading dot) of the associated input file.

    Example (cont.): \sample{\%e} \result{} \filename{.jpg}. \\


    \genidx{mask!template character!E@\sample{E}}\code{\%E} & Is replaced with the extension of
    the output file.
  \end{tabular}

  \caption[Mask template characters]{\label{tab:mask-template-characters}%
    \genidx[\summarylocation]{mask!template character}%
    Special format characters to control the generation of mask filenames.  Uppercase letters
    refer to the output filename and lowercase ones to the input files.}
\end{table}


%%% Local Variables:
%%% fill-column: 96
%%% End:

\fi

\genidx[\rangeendlocation]{expert options}


%%% Local Variables:
%%% fill-column: 96
%%% End:

%% This file is part of Enblend.
%% Licence details can be found in the file COPYING.


\subsection[Expert Fusion Options]{\label{sec:expert-fusion-options}%
  \genidx[\rangebeginlocation]{expert fusion options}%
  \genidx{options!fusion for experts}%
  Expert Fusion Options}

Expert fusion options control details of contrast-weight algorithms and they set ultimate
cutoffs for entropy and exposure fusion.

\begin{codelist}
  \label{opt:contrast-edge-scale}%
  \optidx[\defininglocation]{--contrast-edge-scale}%
  \genidx{Laplacian-of-Gaussian@\propername{Laplacian}-of-\propername{Gaussian}}%
  \gensee{LoG@\acronym{LoG}}{\propername{Laplacian}-of-\propername{Gaussian}}%
\item[\itempar{--contrast-edge-scale=\metavar{EDGE-SCALE}\optional{:\feasiblebreak
      \metavar{LCE-SCALE}:\feasiblebreak\metavar{LCE-FACTOR}}}]\itemend
  A non-zero value for \metavar{EDGE-SCALE} switches on the
  \propername{Laplacian}\hyp{}of\hyp{}\propername{Gaussian} (\acronym{LoG}) edge detection
  algorithm.  \metavar{EDGE-SCALE} is the radius of the \propername{Gaussian} used in the search
  for edges.  Default: \val{val:default-edge-scale}~pixels.

  \genidx{contrast!local enhancement}%
  \gensee{local contrast enhancement}{contrast, local enhancement}%
  \gensee{LCE@\acronym{LCE}}{local contrast enhancement}%
  A positive \metavar{LCE-SCALE} turns on local contrast enhancement (\acronym{LCE}) before the
  \acronym{LoG} edge detection.  \metavar{LCE-SCALE} is the radius of the \propername{Gaussian}
  used in the enhancement step, \metavar{LCE-FACTOR} is the weight factor (``strength'').
  \App{} calculates the \metavar{enhanced} values of the \metavar{original} ones with

  \begin{adalisting}
enhanced :=
    (1 + LCE-FACTOR) * original -
    LCE-FACTOR * GaussianSmooth(original,
                                LCE-SCALE).
  \end{adalisting}

  \metavar{LCE-SCALE} defaults to \val{val:default-lce-scale}~pixels and \metavar{LCE-FACTOR}
  defaults to \val{val:default-lce-factor}.  Append \sample{\%} to \metavar{LCE-SCALE} to
  specify the radius as a percentage of \metavar{EDGE-SCALE}.  Append \sample{\%} to
  \metavar{LCE-FACTOR} to specify the weight as a percentage.


  \label{opt:contrast-min-curvature}%
  \optidx[\defininglocation]{--contrast-min-curvature}%
  \genidx{contrast!minimum curvature}%
\item[--contrast-min-curvature=\metavar{CURVATURE}]\itemend
  Define the minimum \metavar{CURVATURE} for the \acronym{LoG} edge detection.  Default:
  \val{val:default-minimum-curvature}.  Append a \sample{\%} to specify the minimum curvature
  relative to maximum pixel value in the source image (for example 255 or~65535).

  A positive value makes \App{} use the local contrast data (controlled with
  option~\flexipageref{\option{--contrast-window-size}}{opt:contrast-window-size}) for
  curvatures less than \metavar{CURVATURE} and \acronym{LoG} data for values above it.

  A negative value truncates all curvatures less than $-\metavar{CURVATURE}$ to zero.  Values
  above \metavar{CURVATURE} are left unchanged.  This effectively suppresses weak edges.


  \label{opt:contrast-window-size}%
  \optidx[\defininglocation]{--contrast-window-size}%
  \genidx{contrast!window size}%
  \gensee{window size!contrast}{contrast, window size}%
\item[--contrast-window-size=\metavar{SIZE}]\itemend
  Set the window \metavar{SIZE} for local contrast analysis.  The window will be a square of
  $\metavar{SIZE} \times \metavar{SIZE}$~pixels.  If given an even \metavar{SIZE}, \App{} will
  automatically use the next odd number.

  For contrast analysis \metavar{SIZE} values larger than 5~pixels might result in a blurry
  composite image.  Values of 3 and~5~pixels have given good results on focus stacks.

  Valid range: $\metavar{SIZE} \geq \val{val:minimum-contrast-window-size}$, default:
  \val{val:default-contrast-window-size}~pixels.

  See also \sectionName~\ref{sec:fusion-options}, options
  \flexipageref{\option{--contrast-weight}}{opt:contrast-weight}~and
  \flexipageref{\option{--hard-mask}}{opt:hard-mask} below.


  \label{opt:entropy-cutoff}%
  \optidx[\defininglocation]{--entropy-cutoff}%
  \genidx{entropy!cutoff}%
  \gensee{cutoff entropy}{entropy, cutoff}%
\item[--entropy-cutoff=\metavar{LOWER-CUTOFF}\optional{:\metavar{UPPER-CUTOFF}}]\itemend
  Define a cutoff function~$Y'$ for the normalized luminance~$Y$ by \metavar{LOWER-CUTOFF} and
  \metavar{UPPER-CUTOFF}, which gets applied (only) before the local-entropy calculation.
  \metavar{LOWER-CUTOFF} is the value below which pixels are mapped to pure black when
  calculating the local entropy of the pixel's surroundings.  Optionally also define the
  \metavar{UPPER-CUTOFF} value above which pixels are mapped to pure white.

  \begin{equation}\label{equ:luminance-entropy-cutoff}
  Y' = \left\{
  \begin{array}{ll}
    0 & \quad \mbox{for }Y \le \metavar{LOWER-CUTOFF}, \\
    1 & \quad \mbox{for }Y \ge \metavar{UPPER-CUTOFF}\mbox{ and} \\
    Y & \quad \mbox{otherwise.}
  \end{array}
  \right.
  \end{equation}

  Also see \sectionName~\fullref{sec:local-entropy-weighting} for an explanation of local
  entropy.  \figureName~\fullref{fig:entropy-cutoff} shows an example for the luminance mapping.

  Note that the entropy cutoff does not apply directly to the local-entropy~$H$ of a pixel or
  its weight~$w_H$, but the luminance image that get fed into the local-entropy weight
  calculation.  However, assigning \emph{constant} values to extreme shadows or highlights
  in general decreases their local entropy, thereby reducing the pixels' weights.

  For color images \metavar{LOWER-CUTOFF} and \metavar{UPPER-CUTOFF} are applied separately and
  independently to each channel.

  Append a \sample{\%}-sign to specify the cutoff relative to maximum pixel value in the source
  image (for example 255 or~65535).  Negative \metavar{UPPER-CUTOFF} values indicate the maximum
  minus the absolute \metavar{UPPER-CUTOFF} value; the same holds for negative percentages.

  Defaults: \val{val:default-entropy-lower-cutoff} for \metavar{LOWER-CUTOFF} and
  \val{val:default-entropy-upper-cutoff} for \metavar{UPPER-CUTOFF}, that is, all pixels' values
  are taken into account.

  \begin{figure}
    \begin{maxipage}
      \centering
      \includeimage{entropy-cutoff}
    \end{maxipage}

    \caption[Entropy cutoff function]{\label{fig:entropy-cutoff}%
      Modified lightness~$Y'$, \equationabbr~\ref{equ:luminance-entropy-cutoff}, for
      $\metavar{LOWER-CUTOFF} = 5\%$ and $\metavar{UPPER-CUTOFF} = 90\%$, which are rather
      extreme values.}
  \end{figure}

  \begin{geeknote}
    Note that a high \metavar{LOWER-CUTOFF} value lightens the resulting image, as dark and
    presumably noisy pixels are averaged with \emph{equal} weights.  With
    $\metavar{LOWER-CUTOFF} = 0$, the default, on the other hand, ``noise'' might be interpreted
    as high entropy and the noisy pixels get a high weight, which in turn renders the resulting
    image darker.  Analogously, a low \metavar{UPPER-CUTOFF} darkens the output image.
  \end{geeknote}

  \label{opt:entropy-window-size}%
  \optidx[\defininglocation]{--entropy-window-size}%
  \genidx{entropy!window size}%
  \gensee{window size!entropy}{entropy, window size}%
\item[--entropy-window-size=\metavar{SIZE}]\itemend
  Window \metavar{SIZE} for local entropy analysis.  The window will be a square of
  $\metavar{SIZE} \times \metavar{SIZE}$~pixels.

  In the entropy calculation \metavar{SIZE} values of 3 to~7 yield an acceptable compromise of
  the locality of the information and the significance of the local entropy value itself.

  Valid range: $\metavar{SIZE} \geq \val{val:minimum-entropy-window-size}$, default:
  \val{val:default-entropy-window-size}~pixels.

  If given an even \metavar{SIZE} \App{} will automatically use the next-larger odd number.


  \label{opt:exposure-cutoff}%
  \optidx[\defininglocation]{--exposure-cutoff}%
  \genidx{exposure!cutoff}%
  \gensee{cutoff!exposure}{exposure, cutoff}%
\item[\itempar{--exposure-cutoff=\metavar{LOWER-CUTOFF}\optional{:\feasiblebreak
      \metavar{UPPER\hyp{}CUTOFF}\optional{:\feasiblebreak
        \metavar{LOWER\hyp{}PROJECTOR}:\feasiblebreak
        \metavar{UPPER\hyp{}PROJECTOR}}}}]\itemend
  Define an exposure-cutoff function by the luminances \metavar{LOWER-CUTOFF} and
  \metavar{UPPER-CUTOFF}.  Pixels below the lower or above the upper cutoff get a weight of
  exactly zero irrespective of the active exposure-weight function.

  For color images the values of \metavar{LOWER-CUTOFF} and \metavar{UPPER-CUTOFF} refer to the
  gray-scale projection as selected with \metavar{LOWER\hyp{}PROJECTOR} and
  \metavar{UPPER\hyp{}PROJECTOR}.  This is similar to option~\option{--gray-projector}.

  Append a \sample{\%}-sign to specify the cutoff relative to maximum pixel value in the source
  image (for example 255 or~65535).  Negative \metavar{UPPER-CUTOFF} values indicate the maximum
  minus the absolute \metavar{UPPER-CUTOFF} value; the same holds for negative percentages.

  \begin{geeknote}
    The impact of this option is similar, but not identical to transforming \emph{all} input
    images with \uref{\imagemagickorg}{ImageMagick's}\appidx{ImageMagick}
    \command{convert}\prgidx{convert} (see \appendixName~\fullref{sec:helpful-programs}) prior
    to fusing with the commands demonstrated in
    \exampleName~\ref{ex:imagemagick-convert-cutoff}.

    \begin{exemplar}
      \begin{terminal}
        \$ convert IMAGE \bslash \\
        ~~~~~~~~~~~\bslash( +clone -threshold LOWER-CUTOFF \bslash) \bslash \\
        ~~~~~~~~~~~-compose copy\_opacity -composite \bslash \\
        ~~~~~~~~~~~MASKED-IMAGE
      \end{terminal}

      \smallskip

      \begin{terminal}
        \$ convert IMAGE \bslash \\
        ~~~~~~~~~~~\bslash( \bslash \\
        ~~~~~~~~~~~~~~~\bslash( IMAGE -threshold LOWER-CUTOFF \bslash) \bslash \\
        ~~~~~~~~~~~~~~~\bslash( IMAGE -threshold UPPER-CUTOFF -negate \bslash) \bslash \\
        ~~~~~~~~~~~~~~~-compose multiply -composite \bslash \\
        ~~~~~~~~~~~\bslash) \bslash \\
        ~~~~~~~~~~~-compose copy\_opacity -composite \bslash \\
        ~~~~~~~~~~~MASKED-IMAGE
      \end{terminal}

      \caption[\application{ImageMagick} for exposure cutoff]%
              {\label{ex:imagemagick-convert-cutoff}%
                Using \application{ImageMagick} for exposure cutoff operations.  The first
                example only applies a lower cutoff, wherease the second one applies both a
                lower and an upper cutoff to the images.}
    \end{exemplar}

    Transforming some or all input images as shown in the example gives the user more
    flexibility because the thresholds can be chosen for each image individually.
  \end{geeknote}

  The option allows to specify projection operators as in option~\option{--gray-projector} for
  the \metavar{LOWER-CUTOFF} and \metavar{UPPER-CUTOFF} thresholds.

  This option can be helpful if the user wants to exclude underexposed or overexposed pixels
  from the fusing process in \emph{all} of the input images.  The values of
  \metavar{LOWER-CUTOFF} and \metavar{UPPER-CUTOFF} as well as the gray-scale projector
  determine which pixels are considered ``underexposed'' or ``overexposed''.  As any change of
  the exposure-weight curve this option changes the brightness of the resulting image:
  increasing \metavar{LOWER-CUTOFF} lightens the final image and lowering \metavar{UPPER-CUTOFF}
  darkens it.

  Defaults: \val{val:default-exposure-lower-cutoff} for \metavar{LOWER-CUTOFF} and
  \val{val:default-exposure-upper-cutoff} for \metavar{UPPER-CUTOFF}, that is, all pixels'
  values are weighted according to the ``uncut'' exposure-weight curve.

  \figureName~\ref{fig:exposure-cutoff} shows an example.

  The gray-scale projectors \metavar{LOWER-PROJECTOR} and \metavar{UPPER\hyp{}PROJECTOR} default
  to \sample{\val{val:default-exposure-lower-cutoff-projector}} and
  \sample{\val{val:default-exposure-upper-cutoff-projector}}, which are usually the best choices
  for effective cutoff operations on the respective ends.

  \begin{figure}
    \begin{maxipage}
      \centering
      \includeimage{exposure-cutoff}
    \end{maxipage}

    \caption[Exposure cutoff function]{\label{fig:exposure-cutoff}%
      Exposure weight~$w_Y$ after an exposure-cutoff of $\metavar{LOWER-CUTOFF} = 5\%$ and
      $\metavar{UPPER-CUTOFF} = 97\%$ was applied to a \propername{Gaussian} with the
      $\metavar{optimum} = 0.5$ and $\metavar{width} = 0.2$.}
  \end{figure}

  Note that the application of the respective cutoffs is completely independent of the actual
  shape of the exposure weight function.

  If a set of images stubbornly refuses to ``react'' to this option, look at their histograms to
  verify the cutoff actually falls into populated ranges of the histograms.  In the absence of
  an image manipulation program like \uref{\gimporg}{\application{The Gimp}}\appidx{Gimp},
  \uref{\imagemagickorg}{ImageMagick's}\appidx{ImageMagick} can be used to generate
  \uref{\imagemagickorgusagefileshistogram}{histograms}, like, for example,

  \begin{terminal}
    \$ convert -define histogram:unique-colors=false \bslash \\
    ~~~~~~~~~~~IMAGE histogram:- | \bslash \\
    ~~~~~~~display
  \end{terminal}

  The syntax of this option is flexible enough to combine ease of use and precision, as
  \tableName~\ref{tab:flexible-exposure-cutoff} demonstrates.

  \begin{table}
    \begin{maxipage}
      \begin{tabular}{p{.3\linewidth}lp{.3\linewidth}}
        \hline
        \multicolumn{1}{c|}{Task} &
        \multicolumn{1}{c|}{Cutoff Setting} &
        \multicolumn{1}{c}{Effect} \\
        \hline\extraheadingsep
        Suppress some noise. & \option{--exposure-cutoff=5\%} & The percentage makes the cutoff
        specification channel-width agnostic. \\

        Shave off pure white pixels. & \option{--exposure-cutoff=0:-1} & This cutoff
        specification only works for integral pixels, but it will set the weight of the very
        brightest pixels to zero. \\

        Shave off bright white pixels. & \option{--exposure-cutoff=0:-1\%} & Here we exclude the
        brightest 1\% of pixels form the exposure fusion no matter whether the image is encoded
        with integers or floating-point numbers. \\

        Suppress some noise and shave off pure white pixels. & \option{--exposure-cutoff=5\%:-1}
        & Combine the effects of lower and upper cutoff, while mixing relative and absolute
        specifications.
      \end{tabular}
    \end{maxipage}

    \caption[Flexible exposure cutoff]{\label{tab:flexible-exposure-cutoff}%
      Some possible exposure-cutoff settings and their effects on the exposure weights.}
  \end{table}


  \label{opt:exposure-weight-function}%
  \optidx[\defininglocation]{--exposure-weight-function}%
  \genidx{weight!function!exposure}%
  \gensee{exposure weight!function}{weight, function, exposure}%
\item[\itempar{--exposure-weight-function=\metavar{WEIGHT-FUNCTION}~\textrm{(\oldstylefirst~form)}
    \\ --exposure-weight-function=\metavar{SHARED-OBJECT}:\feasiblebreak
    \metavar{SYMBOL}\optional{:\feasiblebreak
      \metavar{ARGUMENT}\optional{:\dots}}~\textrm{(\oldstylesecond~form)}}]\itemend
  First form: override the default (\sample{\val{val:exposure-weight-function}}) exposure weight
  function and use one of the weight-functions in
  \tableName~\fullref{tab:exposure-weight-functions}.

  \restrictednote{Versions with dynamic-linking support only.}  See
  \sectionName~\fullref{sec:finding-out-details} on how to check for this feature.

  \genidx{shared object}%
  \gensee{object!shared}{shared object}%
  \noindent Second form: dynamically load \metavar{SHARED-OBJECT} and use \metavar{SYMBOL} as
  user-defined exposure weight function.  Optionally pass the user-defined function
  \metavar{ARGUMENT}s.

  \genidx{dynamic library}%
  \gensee{library!dynamic}{dynamic library}%
  \begin{geeknote}
    Depending on the operating system environment, a ``shared object'' is sometimes also called
    a ``dynamic library''.
  \end{geeknote}

  \genidx{exposure weight!linear transform}%

  In \tableName~\ref{tab:exposure-weight-functions} the variable~$w_{\mathrm{exp}}$ denotes the
  exposure weight and $z$ represents the normalized luminance~$Y$ linearly transformed by the
  exposure optimum~$Y_{\mathrm{opt}}$ (option~\option{--exposure-optimum}) and \metavar{width}
  (option~\option{--exposure-width}) according to the linear transform
  \begin{equation}\label{equ:linear-luminance-transform}
  z = \frac{Y - Y_{\mathrm{opt}}}{\mathit{width}}.
  \end{equation}

 \genidx{exposure weight function!full width half maximum}%
 \genidx{exposure weight function!\acronym{FWHM}}%
 \genidx{full width half maximum}%
 \gensee{FWHM@\acronym{FWHM}}{full width half maximum}%
 Internally \App{} uses a rescaled \metavar{width} that gives all weight functions the same full
 width at half of the maximum (\acronym{FWHM}), also see \figureName~\ref{fig:exposure-weights}.
 This means for the user that changing the exposure function neither changes the optimum
 exposure nor the width.

  \begin{table}
    \begin{codelist}
      \genidx{exposure weight function!gauss}%
      \genidx{exposure weight function!gaussian}%
    \item[\itempar{gauss \\ gaussian}]\itemend
      The original exposure weight function of \App{} and the only one up to version~4.1.
      \begin{equation}\label{equ:weight:gauss}
      w_{\mathrm{exp}} = \exp\left({-z^2 / 2}\right)
      \end{equation}

      \genidx{exposure weight function!lorentz}%
      \genidx{exposure weight function!lorentzian}%
    \item[\itempar{lorentz \\ lorentzian}]\itemend
      Lorentz curve.
      \begin{equation}\label{equ:weight:lorentz}
      w_{\mathrm{exp}} = \frac{1}{1 + z^2 / 2}
      \end{equation}

      \genidx{exposure weight function!halfsine}%
      \genidx{exposure weight function!half-sine}%
    \item[\itempar{halfsine \\ half-sine}]\itemend
      Upper half-wave of a sine; exactly zero outside.
      \begin{equation}\label{equ:weight:halfsine}
      w_{\mathrm{exp}} =
      \left\{\begin{array}{cl}
      \cos(z) & \mbox{if } |z| \leq \pi/2 \\
      0       & \mbox{otherwise.}
      \end{array}\right.
      \end{equation}

      \genidx{exposure weight function!fullsine}%
      \genidx{exposure weight function!full-sine}%
    \item[\itempar{fullsine \\ full-sine}]\itemend
      Full sine-wave shifted upwards by one to give all positive weights; exactly zero outside.
      \begin{equation}\label{equ:weight:fullsine}
      w_{\mathrm{exp}} =
      \left\{\begin{array}{cl}
      (1 + \cos(z)) / 2 & \mbox{if } |z| \leq \pi \\
      0                 & \mbox{otherwise.}
      \end{array} \right.
      \end{equation}

      \genidx{exposure weight function!bisquare}%
      \genidx{exposure weight function!bi-square}%
    \item[\itempar{bisquare \\ bi-square}]\itemend
      Quartic function.
      \begin{equation}\label{equ:weight:bisquare}
      w_{\mathrm{exp}} =
      \left\{
      \begin{array}{cl}
        1 - z^4 & \mbox{if } |z| \leq 1 \\
        0       & \mbox{otherwise.}
      \end{array}
      \right.
      \end{equation}
    \end{codelist}

    \caption[Exposure weight functions]{\label{tab:exposure-weight-functions}%
      \genidx[\summarylocation]{exposure weight function}%
      Predefined exposure weight functions.  For a graphical comparison see
      \figureName~\ref{fig:exposure-weights}.}
  \end{table}

  For a detailed explanation of all the weight functions
  \sectionName~\fullref{sec:exposure-weighting}.

  If this option is given more than once, the last instance wins.


  \label{opt:gray-projector}%
  \optidx[\defininglocation]{--gray-projector}%
  \genidx{gray projector}%
  \gensee{projector to grayscale}{gray projector}%
\item[--gray-projector=\metavar{PROJECTOR}]\itemend
  Use gray projector~\metavar{PROJECTOR} for conversion of \acronym{RGB} images to grayscale:
  \[
  (R, G, B) \rightarrow Y.
  \]

  In this version of \App{}, the option is effective for exposure weighting and local contrast
  weighting and \metavar{PROJECTOR} defaults to \sample{\val{val:default-grayscale-accessor}}.

  Valid values for \metavar{PROJECTOR} are:

  \begin{codelist}
    \genidx{gray projector!anti-value@\code{anti-value}}%
    \gensee{anti-value gray projector@\code{anti-value} gray projector}%
           {gray projector, \code{anti-value}}%
  \item[anti-value]\itemend
    Do the opposite of the \sample{value} projector: take the minimum of all color channels.
    \[
    Y = \min(R, G, B)
    \]
    This projector can be useful when exposure weighing while employing a lower cutoff (see
    option~\option{--exposure-cutoff}) to reduce the noise in the fused image.

    \genidx{gray projector!average@\code{average}}%
    \gensee{average gray projector@\code{average} gray projector}%
           {gray projector, \code{average}}%
  \item[average]\itemend
    Average red, green, and blue channel with equal weights.  This is the default, and it often
    is a good projector for $\mbox{gamma} = 1$ data.
    \[
    Y = \frac{R + G + B}{3}
    \]

    \genidx{gray projector!channel-mixer@\code{channel-mixer}}%
    \gensee{channel-mixer gray projector@\code{channel-mixer} gray projector}%
           {gray projector, \code{channel-mixer}}%
  \item[\itempar{channel-mixer:\metavar{RED-WEIGHT}:\feasiblebreak
    \metavar{GREEN-WEIGHT}:\feasiblebreak\metavar{BLUE-WEIGHT}}]\itemend
    Weight the channels as given.
    \[
    \begin{array}{r@{\hspace{1ex}}c@{\hspace{1ex}}l@{\hspace{1ex}}c@{\hspace{1ex}}cc}
      Y & = & \metavar{RED-WEIGHT}   & \times & R & + \\
        &   & \metavar{GREEN-WEIGHT} & \times & G & + \\
        &   & \metavar{BLUE-WEIGHT}  & \times & B &
    \end{array}
    \]

    The weights are automatically normalized to one, so

    \begin{literal}
      --gray-projector=channel-mixer:0.25:0.5:0.25 \\
      --gray-projector=channel-mixer:1:2:1 \\
      --gray-projector=channel-mixer:25:50:25
    \end{literal}

    all define the same mixer configuration.

    The three weights \metavar{RED-WEIGHT}, \metavar{GREEN\hyp{}WEIGHT}, and
    \metavar{BLUE\hyp{}WEIGHT} define the relative weight of the respective color channel.  The
    sum of all weights is normalized to one.

    \genidx{gray projector!l-star@\code{l-star}}%
    \gensee{l-star gray projector@\code{l-star} gray projector}%
           {gray projector, \code{l-star}}%
    \genidx{RGB-L*a*b* conversion@\acronym{RGB}-\acronym{L*a*b*} conversion}%
    \gensee{conversion!RGB-L*a*b*@\acronym{RGB}-\acronym{L*a*b*}}%
           {\acronym{RGB}-\acronym{L*a*b*} conversion}%
  \item[l-star]\itemend
    Use the L-channel of the L*a*b*-conversion of the image as its grayscale representation.
    This is a useful projector for $\mbox{gamma} = 1$ data.  It reveals minute contrast
    variations even in the shadows and the highlights.  This projector is computationally
    expensive.  Compare with \sample{pl-star}, which is intended for gamma-corrected images.

    See \uref{\wikipedialabcolorspace}{Wikipedia} for a detailed description of the
    \acronym{Lab}~color space.

    \genidx{gray projector!lightness@\code{lightness}}%
    \gensee{lightness gray projector@\code{lightness} gray projector}%
           {gray projector, \code{lightness}}%
  \item[lightness]\itemend
    Compute the lightness of each \acronym{RGB} pixel as in an
    Hue\hyp{}Saturation\hyp{}Lightness (\acronym{HSL}) conversion of the image.
    \[
    Y = \frac{\max(R, G, B) + \min(R, G, B)}{2}
    \]

    \genidx{gray projector!luminance@\code{luminance}}%
    \gensee{luminance gray projector@\code{luminance} gray projector}%
           {gray projector, \code{luminance}}%
  \item[luminance]\itemend
    Use the weighted average of the \acronym{RGB} pixel's channels as defined by \acronym{CIE}
    (``Commission Internationale de l'\'Eclairage'') and the \acronym{JPEG} standard.
    \[
    Y = 0.30 \times R + 0.59 \times G + 0.11 \times B
    \]

    \genidx{gray projector!pl-star@\code{pl-star}}%
    \gensee{pl-star gray projector@\code{pl-star} gray projector}%
           {gray projector, \code{pl-star}}%
    \genidx{RGB'-L*a*b* conversion@\acronym{RGB'}-\acronym{L*a*b*} conversion}%
  \item[pl-star]\itemend
    Use the L-channel of the L*a*b*-conversion of the image as its grayscale representation.
    This is a useful projector for gamma-corrected data.  It reveals minute contrast variations
    even in the shadows and the highlights.  This projector is computationally expensive.
    Compare with \sample{l-star}, which is intended for $\mbox{gamma} = 1$ images.

    See \uref{\wikipedialabcolorspace}{Wikipedia} for a detailed description of the
    \acronym{Lab}~color space.

    \genidx{gray projector!value@\code{value}}%
    \gensee{value gray projector@\code{value} gray projector}{gray projector, \code{value}}%
  \item[value]\itemend
    Take the Value-channel of the Hue-Saturation-Value (\acronym{HSV}) conversion of the image.
    \[
    Y = \max(R, G, B)
    \]
  \end{codelist}
\end{codelist}

\genidx[\rangeendlocation]{expert fusion options}


%%% Local Variables:
%%% fill-column: 96
%%% End:

%% This file is part of Enblend.
%% Licence details can be found in the file COPYING.


\subsection[Information Options\commonpart]{\label{sec:information-options}%
  \genidx[\rangebeginlocation]{information options}%
  \genidx{options!information}%
  Information Options\commonpart}

\begin{codelist}
  \label{opt:help}%
  \optidx[\defininglocation]{--help}%
  \shoptidx{-h}{--help}%
  \genidx{help}%
\item[\itempar{-h \\ --help}]\itemend
  Print information on the command-line syntax and all available options, giving a boiled-down
  version of this manual.


  \label{opt:show-globbing-algorithms}%
  \optidx[\defininglocation]{--show-globbing-algorithms}%
  \genidx{algorithm!globbing}%
  \gensee{globbing algorithms}{algorithm, globbing}%
\item[--show-globbing-algorithms]\itemend
  Show all globbing algorithms.

  Depending on the build-time configuration and the operating system the binary may support
  different globbing algorithms.  See \sectionName~\fullref{sec:globbing-algorithms}.


%% -- Commented out for Stable Branch 4.2
%%   \label{opt:show-gpu-info}%
%%   \optidx[\defininglocation]{--show-gpu-info}%
%%   \genidx{GPU@\acronym{GPU}}%
%%   \genidx{OpenCL@\acronym{OpenCL}}%
%%   \genidx{GPU@\acronym{GPU}!information on}%
%%   \gensee{information!on \acronym{GPU}}{\acronym{GPU}, information}%
%%   \genidx{OpenCL@\acronym{OpenCL}!information on configuration}%
%%   \gensee{information!on \acronym{OpenCL} configuration}{\acronym{OpenCL}, information}%
%% \item[--show-gpu-info \restrictednote{\acronym{OpenCL}-enabled versions only.}]\itemend
%%   Print a list of all available \acronym{GPU}~devices under \acronym{OpenCL}~control on all
%%   accessible platforms, the current preferences, and then exit; it is the same enumeration that
%%   \begin{literal}
%%     \app{} --verbose --version
%%   \end{literal}
%%   reveals.  \exampleName~\ref{ex:opencl-config} shows a complete output.
%%
%%   \begin{exemplar}
%%     \begin{maxipage}
%%       \begin{terminal}
%%         \$ \app{} --show-gpu-info \\
%%         Available, OpenCL-compatible platform(s) and their device(s) \\
%%         ~~- Platform \#1:~Advanced Micro Devices, Inc., \\
%%         ~~~~~~~~~~~~~~~~~AMD Accelerated Parallel Processing, \\
%%         ~~~~~~~~~~~~~~~~~OpenCL 1.2 AMD-APP (1526.3) \\
%%         ~~~~* no GPU devices found on this platform \\
%%         ~~- Platform \#2:~NVIDIA Corporation, \\
%%         ~~~~~~~~~~~~~~~~~NVIDIA CUDA, \\
%%         ~~~~~~~~~~~~~~~~~OpenCL 1.1 CUDA 6.5.51 \\
%%         ~~~~* Device \#1:~max.~1024 work-items \\
%%         ~~~~~~~~~~~~~~~~~1047872 KB global memory with 32 KB read/write cache \\
%%         ~~~~~~~~~~~~~~~~~48 KB dedicated local memory \\
%%         ~~~~~~~~~~~~~~~~~64 KB maximum constant memory \\
%%         ~~~~~~~~~~~~~~~~~Extensions: \\
%%         ~~~~~~~~~~~~~~~~~~~~~cl\_khr\_byte\_addressable\_store \\
%%         ~~~~~~~~~~~~~~~~~~~~~cl\_khr\_fp64 \\
%%         ~~~~~~~~~~~~~~~~~~~~~cl\_khr\_gl\_sharing \\
%%         ~~~~~~~~~~~~~~~~~~~~~cl\_khr\_global\_int32\_base\_atomics \\
%%         ~~~~~~~~~~~~~~~~~~~~~cl\_khr\_global\_int32\_extended\_atomics \\
%%         ~~~~~~~~~~~~~~~~~~~~~cl\_khr\_icd \\
%%         ~~~~~~~~~~~~~~~~~~~~~cl\_khr\_local\_int32\_base\_atomics \\
%%         ~~~~~~~~~~~~~~~~~~~~~cl\_khr\_local\_int32\_extended\_atomics \\
%%         ~~~~~~~~~~~~~~~~~~~~~cl\_nv\_compiler\_options \\
%%         ~~~~~~~~~~~~~~~~~~~~~cl\_nv\_copy\_opts \\
%%         ~~~~~~~~~~~~~~~~~~~~~cl\_nv\_device\_attribute\_query \\
%%         ~~~~~~~~~~~~~~~~~~~~~cl\_nv\_pragma\_unroll \\
%%         ~~Search path (expanding ENBLEND\_OPENCL\_PATH and appending built-in path) \\
%%         ~~~~/usr/local/share/enblend/kernels:/usr/share/enblend/kernels \\
%%         Currently preferred GPU is device \#1 on platform \#2 (auto-detected).
%%       \end{terminal}
%%     \end{maxipage}
%%
%%     \caption[Sample \acronym{OpenCL} configuration.]%
%%             {\label{ex:opencl-config}%
%%               A sample \acronym{OpenCL} configuration as detected by \App.}
%%   \end{exemplar}


  \label{opt:show-image-formats}%
  \optidx[\defininglocation]{--show-image-formats}%
  \genidx{format!image}%
  \gensee{image formats}{format, image}%
\item[--show-image-formats]\itemend
  Show all recognized image formats, their filename extensions and the supported per-channel
  depths.

  Depending on the build-time configuration and the operating system, the binary supports
  different image formats, typically: \acronym{BMP}, \acronym{EXR}, \acronym{GIF},
  \acronym{HDR}, \acronym{JPEG}, \acronym{PNG}, \acronym{PNM}, SUN, \acronym{TIFF},
  and~\acronym{VIFF} and recognizes different image-filename extensions, again typically:
  \filename{bmp}, \filename{exr}, \filename{gif}, \filename{hdr}, \filename{jpeg},
  \filename{jpg}, \filename{pbm}, \filename{pgm}, \filename{png}, \filename{pnm},
  \filename{ppm}, \filename{ras}, \filename{tif}, \filename{tiff}, and~\filename{xv}.

  The maximum number of different per-channel depths any \appcmd{} provides is seven:
  \begin{compactitemize}
  \item 8~bits unsigned integral, \sample{uint8}
  \item 16~bits unsigned or signed integral, \sample{uint16} or \sample{int16}
  \item 32~bits unsigned or signed integral, \sample{uint32} or \sample{int32}
  \item 32~bits floating-point, \sample{float}
  \item 64~bits floating-point, \sample{double}
  \end{compactitemize}


  \label{opt:show-signature}%
  \optidx[\defininglocation]{--show-signature}%
  \genidx{signature}%
\item[--show-signature]\itemend
  Show the user name of the person who compiled the binary, when the binary was compiled, and on
  which machine this was done.

  This information can be helpful to ensure the binary was created by a trustworthy builder.


  \label{opt:show-software-components}%
  \optidx[\defininglocation]{--show-software-components}%
  \genidx{information!on software components}%
  \gensee{software!components}{information, on software components}%
  \genidx{dynamic-library environment}%
  \genidx{header files}%
\item[--show-software-components]\itemend
  Show the name and version of the compiler that built \App{} followed by the versions of all
  important libraries against which \App{} was compiled and linked.

  Technically, the version information is taken from header files, thus it is independent of the
  dynamic-library environment the binary runs within.  The library versions printed here can
  help to reveal version mismatches with respect to the actual dynamic libraries available to
  the binary.


  \label{opt:version}%
  \optidx[\defininglocation]{--version}%
  \shoptidx{-V}{--version}%
  \genidx{software!version}%
  \gensee{version}{software, version}%
  \gensee{binary version}{software, version}%
\item[\itempar{-V \\ --version}]\itemend
  Output information on the binary's version.

  Team this option with \flexipageref{\option{--verbose}}{opt:verbose} to show configuration
  details, like the extra features that may have been compiled in.  For details consult
  \sectionName~\fullref{sec:exact-version}.
\end{codelist}

\genidx[\rangeendlocation]{information options}


%%% Local Variables:
%%% fill-column: 96
%%% End:


\genidx[\rangeendlocation]{options}


%%% Local Variables:
%%% fill-column: 96
%%% End:

%% This file is part of Enblend.
%% Licence details can be found in the file COPYING.


\section[Option Delimiters\commonpart]{Option Delimiters\commonpart
  \label{sec:option-delimiters}
  \genidx{option delimiters}
  \genidx{options!delimiters}}

\application{Enblend} and \application{Enfuse} allow the arguments
supplied to the programs' options to be separated by different
separators.  The online documentation and this manual, however,
exclusively use the colon \sample{:} in every syntax definition and in
all examples.


\subsection[Numeric Arguments]{Numeric Arguments
  \label{sec:option-delimiters-numeric-arguments}
  \genidx{options!delimiters!numeric arguments}}

Valid delimiters are the the semicolon \sample{;}, the colon
\sample{:}, and the slash \sample{/}.  All delimiters may be mixed
within any option that takes numeric arguments.

Examples using some \application{Enfuse} options:

\begin{codelist}
\item[--contrast-edge-scale=0.667:6.67:3.5]\itemend Separate all
  arguments with colons.

\item[--contrast-edge-scale=0.667;6.67;3.5]\itemend Use semi-colons.

\item[--contrast-edge-scale=0.667;6.67/3.5]\itemend Mix semicolon and
  slash in weird ways.

\item[--entropy-cutoff=3\%/99\%]\itemend All delimiters also work in
  conjunction with percentages.

\item[--gray-projector=channel-mixer:3/6/1]\itemend Separate arguments
  with a colon and two slashes.

\item[--gray-projector=channel-mixer/30;60:10]\itemend Go wild and
  Enfuse will understand.
\end{codelist}


\subsection[Filename Arguments]{Filename Arguments
  \label{sec:option-delimiters-filename-arguments}
  \genidx{options!delimiters!filename arguments}}

Here, the accepted delimiters are \sample{,}, \sample{;}, and
\sample{:}.  Again, all delimiters may be mixed within any option that
has filename arguments.

Examples:

\begin{codelist}
\item[--save-masks=soft-mask-\%03i.tif:hard-mask-03\%i.tif]\itemend Separate
  all arguments with colons.

\item[--save-masks=\%d/soft-\%n.tif,\%d/hard-\%n.tif]\itemend Use a
  comma.
\end{codelist}

%% This file is part of Enblend.
%% Licence details can be found in the file COPYING.


\section[Response Files\commonpart]{\label{sec:response-files}%
  \genidx[\rangebeginlocation]{response files}%
  \genidx{file!response}%
  Response Files\commonpart}

A response file contains names of images or other response filenames.
\genidx{\val*{val:response-file-prefix-char}\ (response file prefix)}%
\gensee{response file prefix!\sample{\val*{val:response-file-prefix-char}}}%
       {\sample{\val*{val:response-file-prefix-char}}}
Introduce response file names at the command line or in a response file with an
\code{\val{val:response-file-prefix-char}}~character.

\genidx{order!of image processing}\gensee{image processing order}{order of image processing}
\application{Enblend} and \application{Enfuse} process the list \metavar{INPUT} strictly from
left to right, expanding response files in depth-first order.  Multi-layer files are processed
from first layer to the last.  The following examples only show \application{Enblend}, but
\application{Enfuse} works exactly the same.

\begin{description}
\item[Solely image filenames.]\itemend
  Example:

  \begin{literal}
    enblend image-1.tif image-2.tif image-3.tif
  \end{literal}

  The ultimate order in which the images are processed is: \filename{image-1.tif},
  \filename{image-2.tif}, \filename{image-3.tif}.

\item[Single response file.]\itemend
  Example:

  \begin{literal}
    enblend \val*{val:response-file-prefix-char} list
  \end{literal}

  where file~\filename{list} contains

  \begin{literal}
    img1.exr \\
    img2.exr \\
    img3.exr \\
    img4.exr \\
  \end{literal}

  Ultimate order: \filename{img1.exr}, \filename{img2.exr}, \filename{img3.exr},
  \filename{img4.exr}.

\item[Mixed literal names and response files.]\itemend
  Example:

  \begin{literal}
    enblend \val*{val:response-file-prefix-char} master.list image-09.png image-10.png
  \end{literal}

  where file~\filename{master.list} comprises of

  \begin{literal}
    image-01.png \\
    \val*{val:response-file-prefix-char} first.list \\
    image-04.png \\
    \val*{val:response-file-prefix-char} second.list \\
    image-08.png \\
  \end{literal}

  \filename{first.list} is

  \begin{literal}
    image-02.png \\
    image-03.png \\
  \end{literal}

  and \filename{second.list} contains

  \begin{literal}
    image-05.png \\
    image-06.png \\
    image-07.png \\
  \end{literal}

  Ultimate order: \filename{image-01.png}, \filename{image-02.png}, \filename{image-03.png},
  \filename{image-04.png}, \filename{image-05.png}, \filename{image-06.png},
  \filename{image-07.png}, \filename{image-08.png}, \filename{image-09.png},
  \filename{image-10.png},
\end{description}


\subsection[Response File Format]{\label{sec:response-file-format}%
  \genidx{response file!format}%
  \gensee{format of response file}{response file format}%
  Response File Format}

\genidx{\val*{val:response-file-comment-char} (response file comment)}%
\gensee{response file!comment (\sample{\val*{val:response-file-comment-char}})}%
       {\sample{\val*{val:response-file-comment-char}}}%
Response files contain one filename per line.  Blank lines or lines beginning with a
\code{\val{val:response-file-comment-char}}~sign are ignored; the latter can serve as comments.
Filenames that begin with a \code{\val{val:response-file-prefix-char}}~character denote other
response files.  \tableName~\ref{tab:response-file-format} states a formal grammar of response
files in \uref{\wikipediaebnf}{\acronym{EBNF}}.

\begin{table}
  \begin{tabular}{l@{$\quad::=\quad$}l}
    \metavar{response-file} & \metavar{line}* \\
    \metavar{line} & (\metavar{comment} $|$ \metavar{file-spec}) [\sample{\bslash r}] \sample{\bslash n} \\
    \metavar{comment} & \metavar{space}* \sample{\val*{val:response-file-comment-char}} \metavar{text} \\
    \metavar{file-spec} & \metavar{space}* \sample{\val*{val:response-file-prefix-char} } \metavar{filename} \metavar{space}* \\
    \metavar{space} & \sample{\textvisiblespace} $|$ \sample{\bslash t} \\
  \end{tabular}

  \noindent where \metavar{text} is an arbitrary string and \metavar{filename} is any filename.

  \caption[Grammar of response files]{\label{tab:response-file-format}%
    \genidx{response file!grammar}%
    \gensee{grammar!response file}{response file, grammar}%
    \acronym{EBNF} definition of the grammar of response files.}
\end{table}

In a response file relative filenames are used relative the response file itself, not relative
to the current-working directory of the application.

The above grammar might surprise the user in the some ways.

\begin{description}
\item[Whitespace trimmed at both line ends]\itemend
  For convenience, whitespace at the beginning and at the end of each line is ignored.  However,
  this implies that response files cannot represent filenames that start or end with whitespace,
  as there is no quoting syntax.  Filenames with embedded whitespace cause no problems, though.

\item[Only whole-line comments]\itemend
  Comments in response files always occupy a complete line.  There are no ``line-ending
  comments''.  Thus, in

  \begin{literal}
    \val*{val:response-file-comment-char} exposure series \\
    img-0.33ev.tif \val*{val:response-file-comment-char} "middle" EV \\
    img-1.33ev.tif \\
    img+0.67ev.tif \\
  \end{literal}

  only the first line contains a comment, whereas the second line includes none.  Rather, it
  refers to a file called

  \begin{literal}
    img-0.33ev.tif \val*{val:response-file-comment-char} "middle" EV
  \end{literal}

\item[Image filenames cannot start with \code{\val{val:response-file-prefix-char}}]\itemend
  A \code{\val{val:response-file-prefix-char}}~sign invariably introduces a response file, even
  if the filename's extension hints towards an image.
\end{description}

\genidx{response file!force recognition of}If \application{Enblend} or \application{Enfuse} do
not recognize a response file, they will skip the file and issue a warning.  To force a file
being recognized as a response file add one of the following syntactic comments to the
\emph{first} line of the file.

\begin{literal}
  response-file: true\synidx{response-file} \\
  enblend-response-file: true\synidx{enblend-response-file} \\
  enfuse-response-file: true\synidx{enfuse-response-file} \\
\end{literal}

Finally, \exampleName~\ref{ex:response-file} shows a complete response file.

\begin{exemplar}
  \begin{literal}
    \val*{val:response-file-comment-char}~4\bslash pi panorama! \\
    \mbox{} \\
    \val*{val:response-file-comment-char}~These pictures were taken with the panorama head. \\
    \val*{val:response-file-prefix-char}~round-shots.list \\
    \mbox{} \\
    \val*{val:response-file-comment-char}~Freehand sky shot. \\
    zenith.tif \\
    \mbox{} \\
    \val*{val:response-file-comment-char}~"Legs, will you go away?" images. \\
    nadir-2.tif \\
    nadir-5.tif \\
    nadir.tif \\
  \end{literal}

  \caption[Complete response file]%
          {\label{ex:response-file}%
            Example of a complete response file.}
\end{exemplar}


\subsection[Syntactic Comments]{\label{sec:syntactic-comments}%
  \genidx{response file!syntactic comment}%
  \gensee{syntactic comment!response file}{response file, syntactic comment}%
  Syntactic Comments}

Comments that follow the format described in
\tableName~\ref{tab:response-file-syntactic-comment} are treated as instructions how to
interpret the rest of the response file.  A syntactic comment is effective immediately and its
effect persists to the end of the response file, unless another syntactic comment undoes it.

\begin{table}
  \begin{tabular}{l@{$\quad::=\quad$}l}
    \metavar{syntactic-comment} & \metavar{space}* \sample{\val*{val:response-file-comment-char}}
    \metavar{space}* \metavar{key}
    \metavar{space}* \sample{:}
    \metavar{space}* \metavar{value} \\

    \metavar{key} & (\sample{A}\dots \sample{Z} $|$ \sample{a}\dots \sample{z} $|$ \sample{-})+ \\
  \end{tabular}

  where \metavar{value} is an arbitrary string.

  \caption[Grammar of syntactic comments]{\label{tab:response-file-syntactic-comment}%
  \genidx{syntactic comment!grammar}%
  \genidx{grammar!syntactic comment}%
    \acronym{EBNF} definition of the grammar of syntactic comments in response files.}
\end{table}

Unknown syntactic comments are silently ignored.

A special index for \flexipageref{syntactic comments}{sec:syncomm-index} lists them in
alphabetic order.


\subsection[Globbing Algorithms]{\label{sec:globbing-algorithms}%
  \genidx{globbing algorithm}%
  \gensee{algorithm}{globbing algorithm}%
  Globbing Algorithms}

The three equivalent syntactic keys

\begin{itemize}
\item
  \code{glob},\synidx{glob}

\item
  \code{globbing},\synidx{globbing} or

\item
  \code{filename-globbing}\synidx{filename-globbing}
\end{itemize}

control the algorithm that \application{Enblend} or \application{Enfuse} use to glob filenames
in response files.

\genidx{globbing algorithm!\code{literal}}%
\genidx{globbing algorithm!\code{wildcard}}%
All versions of \application{Enblend} and \application{Enfuse} support at least two algorithms:
\code{literal}, which is the default, and \code{wildcard}.  See
\tableName~\ref{tab:globbing-algorithms} for a list of all possible globbing algorithms.  To
find out about the algorithms in your version of \application{Enblend} or \application{Enfuse}
use option~\option{--show-globbing-algorithms}.

\begin{table}
  \begin{minipage}{\linewidth}
    \begin{codelist}
      \genidx{globbing algorithm!\code{literal}}%
    \item[literal]\itemend
      Do not glob.  Interpret all filenames in response files as literals. This is the default.

      Please remember that whitespace at both ends of a line in a response file \emph{always}
      gets discarded.

      \genidx{globbing algorithm!\code{wildcard}}%
      \genidx{glob}%
    \item[wildcard]\itemend
      Glob using the wildcard characters~\sample{?}, \sample{*}, \sample{[}, and \sample{]}.

      The \propername{Win32} implementation only globs the filename part of a path, whereas all
      other implementations perform wildcard expansion in \emph{all} path components.  Also see
      \uref{\kernelorgglob}{\manpage{glob}{7}}.

      \genidx{globbing algorithm!\code{none}}
    \item[none]\itemend
      Alias for \code{literal}.

      \genidx{globbing algorithm!\code{shell}}
    \item[shell]\itemend
      The \code{shell} globbing algorithm works as \code{literal} does.  In addition, it
      interprets the wildcard characters~\sample{\{}, \sample{\atsign}, and \sample{\squiggle}.
      This makes the expansion process behave more like common \acronym{UN*X}
      shells.

      \genidx{globbing algorithm!\code{sh}}
    \item[sh]\itemend
      Alias for \code{shell}.
    \end{codelist}
  \end{minipage}

  \caption[Globbing algorithms]{\label{tab:globbing-algorithms}%
    \genidx{globbing algorithms}%
    \genidx{algorithms!globbing}%
    Globbing algorithms for the use in response files.}
\end{table}

\exampleName~\ref{ex:globbing-algorithm} gives an example of how to control filename-globbing in
a response file.

\begin{exemplar}
  \begin{literal}
    \val*{val:response-file-comment-char}~Horizontal panorama \\
    \val*{val:response-file-comment-char}~15 images \\
    \mbox{} \\
    \val*{val:response-file-comment-char}~filename-globbing: wildcard \\
    \mbox{} \\
    image\_000[0-9].tif \\
    image\_001[0-4].tif \\
  \end{literal}

  \caption[Filename-globbing syntactic comment]%
          {\label{ex:globbing-algorithm}%
            Control filename-globbing in a response file with a syntactic comment.}
\end{exemplar}


\subsection[Default Layer Selection]{\label{sec:default-layer-selection}%
  \genidx{default layer selection}%
  \genidx{layer selection!default}%
  Default Layer Selection}

The key~\code{layer-selector}\synidx{layer-selector} provides the same functionality as does the
command\hyp{}line option~\option{--layer-selector}, but on a per response\hyp{}file basis.  See
\sectionName~\ref{sec:common-options}.

This syntactic comment affects the layer selection of all images listed after it including those
in included response files until another \code{layer-selector} overrides it.

\genidx[\rangeendlocation]{response files}


%%% Local Variables:
%%% fill-column: 96
%%% End:

%% This file is part of Enblend.
%% Licence details can be found in the file COPYING.


\section[Layer Selection\commonpart]{\label{sec:layer-selection}%
  \genidx[\rangebeginlocation]{layer!selection}%
  Layer Selection\commonpart}

Some image formats, like for example \acronym{TIFF}, permit to store more than one image in a
single file, where all the contained images can have different sizes, number of channels,
resolutions, compression schemes, and so on.  The file there acts as a container for an
\emph{ordered} set of images.

\genidx[\defininglocation]{file!multi-page}%
\gensee{multi-page file}{file, multi-page}%
\gensea{multi-page file}{layer}%
\genidx[\defininglocation]{image!directory}%
\genidx[\defininglocation]{layer!image}%
In the \acronym{TIFF}-documentation these are known as ``multi-page'' files and because the
image data in a \acronym{TIFF}-file is associated with a ``directory'', the files sometimes are
also called ``multi-directory'' files.  In this manual, multiple images in a file are called
``layers''.

The main advantage of multi-layer files over a set of single-layer ones is a cleaner work area
with less image-files and thus an easier handling of the intermediate products which get created
when generating a panorama or fused image, and in particularly with regard to panoramas of fused
images.

The difficulty in working with layers is their lack of a possibly mnemonic naming scheme.  They
do not have telling names like \filename{taoth-vaclarush} or \filename{valos-cor}, but only
numbers.


\subsection[Layer Selection Syntax]{\label{sec:layer-selection-syntax}%
  \genidx{layer selection syntax}%
  \genidx{syntax!layer selection}%
  Layer Selection Syntax}

To give the user the same flexibility in specifying and ordering images as with single-layer
images, both \App{} and \OtherApp{} offer a special syntax to select layers in multi-page files
by appending a \metavar{layer-specification} to the image filename.
\tableName~\ref{tab:layer-selection-grammar} defines the grammar of
\metavar{layer-specification}\/s.

Selecting a tuple of layers with a \metavar{layer-specification} overrides the active layer
selection algorithm.  See also
option~\flexipageref{\option{--layer-selector}}{opt:layer-selector} and
\sectionName~\fullref{sec:response-files}.  Layer selection works at the command-line as well as
in Response Files; see \sectionName~\ref{sec:response-files}.

\begin{table}
  \begin{tabular}{l@{$\quad::=\quad$}l}
    \metavar{layer-specification} &
    \sample{\val*{val:LAYERSPEC_OPEN}} \metavar{selection-tuple} \sample{\val*{val:LAYERSPEC_CLOSE}} \\
    \metavar{selection-tuple} & \metavar{selection} [ \sample{:} \metavar{selection} ] \\
    \metavar{selection} & \{ \metavar{singleton} $|$ \metavar{range} \} \\
    \metavar{range} & [ \sample{\val*{val:layer-range-reverse-keyword}} ]
            [ \metavar{range-bound} ] \sample{\val*{val:layer-range-separator}} [ \metavar{range-bound} ] \\
    \metavar{range-bound} & \metavar{singleton} $|$ \sample{\val*{val:layer-range-empty-index-symbol}} \\
    \metavar{singleton} & \metavar{index} $|$ \sample{-} \metavar{index} \\
  \end{tabular}

  where \metavar{index} is an integral layer index starting at one.

  \caption[Grammar of layer specifications]{\label{tab:layer-selection-grammar}%
    \genidx{syntax!layer selection!grammar}%
    \acronym{EBNF} definition of the grammar of layer specifications.  In addition to the
    \metavar{selection} separator~\sample{:} shown all usual numeric-option delimiters
    (\sample{\val{val:numeric-option-delimiters}}) apply.  The keyword for \metavar{range}
    reversal, \sample{\val{val:layer-range-reverse-keyword}}, can be abbreviated to any length
    and is treated case-insensitively.}
\end{table}

The simplest \metavar{layer-specification} are the layer-\metavar{index}es.  The first layer
gets index~1, the second layer~2, and so on.  Zero never is a valid index!  For convenience
indexing backward\footnotemark{} is also possible.  This means by prefixing an index with a
minus-sign~(\sample{-}) counting will start with the last layer of the \emph{associated}
multi-page image, such that the last layer always has index~\code{-1}, the next to last
index~\code{-2} and so on.  Out-of-range indexes are silently ignored whether forward or
backward.

\footnotetext{\genidx{Carter, Samantha@\propername{Carter, Samantha}}\propername{Samantha
    Carter}: ``There has to be a way to reverse the process.  The answer has to be here.''}

The single layer of a single-layer file always can be accessed either with index~\sample{1} or
\sample{-1}.

Select a contiguous \metavar{range} of indexes with the range
operator~\sample{\val{val:layer-range-separator}}, where the \metavar{range-bound}\/s are
forward or backward indexes.  Leaving out a bound or substituting the open-range
indicator~\sample{\val{val:layer-range-empty-index-symbol}} means a maximal range into the
respective direction.

Layer specifications ignore white space, but usual shells do not.  This means that at the
command-line

\begin{terminal}
  \$ \app{} --output=out.tif --verbose multi-layer.tif[2:]
\end{terminal}

\noindent works, whereas spaced-out out phrase \sample{multi-layer.tif [2 : ]} must be quoted

\begin{terminal}
  \$ \app{} --output=out.tif --verbose 'multi-layer.tif[2 : ]'
\end{terminal}

Quoting will also be required if \App's delimiters have special meanings to the shell.

\smallskip

Examples for an image with 8~layers.

\begin{codelist}
  \newcommand*{\lspec}[1]{\mbox{\val*{val:LAYERSPEC_OPEN}{#1}\val*{val:LAYERSPEC_CLOSE}}}%
\item[\lspec{}] The empty selection selects nothing and in that way works like the
  layer-selector \sample{no-layer}.

\item[\lspec{2 :\ 4 :\ 5}] Select only layers~2, 4, and~5 in this order.

\item[\lspec{2 :\ -4 :\ -3}] Like before, but with some backward-counting indexes.

\item[\lspec{1 \val*{val:layer-range-separator}\ 4}] Layers 1~to 4, this is 1, 2, 3, and~4 in
  this order.

\item[\lspec{\val*{val:layer-range-empty-index-symbol}\ \val*{val:layer-range-separator}\ 4}]
  Same as above in open-range notation.

\item[\lspec{\val*{val:layer-range-separator}\ 4}] Same as above in abbreviated, open-range
  notation.

\item[\lspec{-2 \val*{val:layer-range-separator}\ \val*{val:layer-range-empty-index-symbol}}]
  The last two layers, which are~7 and~8 in our running example.

\item[\lspec{\val*{val:layer-range-empty-index-symbol}\ \val*{val:layer-range-separator}\ \val*{val:layer-range-empty-index-symbol}}]
  All layers in their natural order.

\item[\lspec{\val*{val:layer-range-separator}}] All layers in their natural order selected with
  the abbreviated notation.

\item[\lspec{reverse
    \val*{val:layer-range-empty-index-symbol}\ \val*{val:layer-range-separator}\ \val*{val:layer-range-empty-index-symbol}}]
  All layers in reverse order.  This yields 8, 7, 6, 5, 4, 3, 2, and~1.

\item[\lspec{rev \val*{val:layer-range-separator}}] All layers in reversed order as before
  selected with the abbreviated notation.

\item[\lspec{r -3 \val*{val:layer-range-separator}}] The last three layers in reverse order,
  this is 8, 7 and~6 in our running example.
\end{codelist}

\begin{geeknote}
  Shell expansion will not work anymore with a filename terminated by a layer specification
  expression (or anything else), because to the shell it is not a filename anymore.  Work around
  with, for example,

  \begin{terminal}
    \$ \app{} `for x in image-??.tif; do echo \$x[2]; done`
  \end{terminal}

  \noindent or

  \begin{terminal}
    \$ \app{} \$(ls -1 image-??.tif | sed -e 's/\$/[2]/')
  \end{terminal}

  The order of the indexes determines the order of the layers, this is, the images.  An index
  can occur multiple times, which causes layer to be considered \emph{again}.  Consequently,
  this will lead to an error with \application{Enblend}, but may be desired with
  \application{Enfuse} in \code{soft-mask}~mode to give the image more weight by mentioning it
  more than once.
\end{geeknote}


\subsection[Tools for Multi-Page Files]{\label{sec:tools-for-multi-page-files}%
  \genidx{multi-page file!tools}%
  Tools for Multi-Page Files}

Here are some tools that are particularly useful when working with multi-page files.  For more
helpful utilities check out \appendixName~\fullref{sec:helpful-programs}.

\begin{itemize}
\item
  \appidx{Hugin}\application{Hugin}'s stitcher, \prgidx{nona \textrm{(Hugin)}}\command{nona}
  produces multi-page \acronym{TIFF}~file, when called with the \sample{-m
    TIFF\_multilayer}-option.

\item
  The utility \prgidx{tiffcp \textrm{(LibTIFF)}}\command{tiffcp} of the
  \acronym{TIFF}-\uref{\remotesensingorglibtiff}{LibTIFF tool suite} merges several
  \acronym{TIFF}-images into a single multi-page file.

\item
  The sister program \prgidx{tiffsplit \textrm{(LibTIFF)}}\command{tiffsplit} splits a
  multi-page file into a set of single-page \acronym{TIFF}-images.

\item
  Another utility of the same origin, \prgidx{tiffinfo \textrm{(LibTIFF)}}\command{tiffinfo}, is
  very helpful when inquiring the contents of single- or multi-page file \acronym{TIFF}-files.

  \genidx{image!frame}
\item
  All tools of the \uref{\imagemagickorg}{\application{ImageMagick}}-suite, like, for example,
  \prgidx{convert \textrm{(ImageMagick)}}\command{convert} and \prgidx{display
    \textrm{(ImageMagick)}}\command{display} use a
  \uref{\imagemagickorgcommandlineprocessinginput}{similar syntax} as \App{} to select layers
  (which in \application{ImageMagick} parlance are called ``frames'') in multi-page files.
  Please note that \application{ImageMagick} tools start indexing at zero, whereas \App{} starts
  counting at one.

\item
  \App{} and \OtherApp{} by default apply the \sample{\val{val:layer-selector}} selector (see
  option~\flexipageref{\option{--layer-selector}}{opt:layer-selector}) to each of the input
  images.
\end{itemize}

Please bear in mind that some image-processing tools -- none of the above though -- do
\emph{not} handle multi-page files correctly, where the most unfruitful ones only take care of
the first layer and \emph{silently} ignore any further layers.

\genidx[\rangeendlocation]{layer!selection}


%%% Local Variables:
%%% fill-column: 96
%%% End:


\genidx[\rangeendlocation]{invocation}

%HEVEA\cutend


%%% Local Variables:
%%% fill-column: 96
%%% End:
