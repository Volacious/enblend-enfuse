%% This file is part of Enblend.
%% Licence details can be found in the file COPYING.


\subsection[User-Defined \acronym{OpenCL} Functions]{\label{sec:user-defined-opencl-functions}%
  \genidx{weighting!exposure!user-defined!\acronym{OpenCL}}%
  User-Defined \acronym{OpenCL} Exposure Weighting Functions}

See also \sectionName~\ref{sec:user-defined-dynlink-functions} above on defining C++-weighting
functions via dynamic linking.


\subsubsection[Introduction]{\label{sec:user-defined-opencl-functions-introduction}%
  \genidx{weighting!exposure!\acronym{OpenCL} introduction}%
  Introduction}

\genidx{\acronym{GPGPU}}%
\genidx{\acronym{OpenCL}}%

So what has \acronym{OpenCL} to do with user-defined exposure weighting functions?  Every
\acronym{OpenCL} system working on top of a \acronym{GPGPU} comes with a compiler and a linker
specifically tailored for the \acronym{GPGPU}.  Both tools are hidden in one or more libraries
of the respective graphics card.  \App{} reaps the benefits from these libraries already having
been installed and uses the \acronym{OpenCL}-host interface to compile, link and execute
\acronym{OpenCL}-programs on the \acronym{GPGPU}.

In brief, if \appcmd{} has been built with \acronym{OpenCL} support, the system has all
necessary libraries installed, the user can feed \acronym{OpenCL}-source code -- actually a
C-dialect -- to \App{} without the need to install and maintain all the tools to compile
\appcmd{} itself as in the previously described method of dynamic linking.

The ability of \acronym{GPGPU}s to crunch humongous amounts of data is unimportant here.

\noindent\restrictednote{\acronym{OpenCL}-enabled versions only.}

\optidx{--exposure-weight-function}%
\optidx{--gpu}%
For \App{} executables compiled with \acronym{OpenCL} support (see
Section~\fullref{sec:finding-out-details} on how to check this feature), \App{} can work with
user-defined exposure weighting functions, passed with the long form of
option~\option{--exposure-weight-function} after \acronym{GPU}-support has been requested with
option~\option{--gpu}.  \App{} compiles, links and evaluables the exposure weight function
identifed by \metavar{SYMBOL} from \metavar{OPENCL-SOURCE}:

\begin{literal}
  --gpu --exposure-weight-function=\metavar{OPENCL-SOURCE}:\feasiblebreak
  \metavar{SYMBOL}\optional{:\feasiblebreak
    \metavar{ARGUMENT}\optional{:\dots}}%
  \optidx{--exposure-weight-function}
\end{literal}

If \metavar{SYMBOL} is not found in \metavar{OPENCL-SOURCE} or does not define a suitable
function, \app{} aborts with an extensive error message.  In
\exampleName~\ref{ex:opencl-userweight-error} we supply a wrong \metavar{SYMBOL} called
\sample{foobar}, whereas \filename{variable\_power.cl} -- one of the supplied examples --
actually defines \sample{weight}.

\begin{exemplar}[htbp]
  \begin{maxipage}
    \centering
    \begin{terminal}
\$ enfuse --gpu --exposure-weight-function=variable\_power.cl:foobar image-[0-9].tif \\
enfuse: info: chose OpenCL platform \#2, NVIDIA Corporation, NVIDIA CUDA, device \#1 \\
enfuse: invalid binary \\
enfuse: note: build status \\
enfuse: note:~~~~~error (status-code = -2) \\
enfuse: note: build options \\
enfuse: note: ~~~~-cl-nv-verbose \bslash \\
~~~~~~~~~~~~~~~~~~-cl-single-precision-constant \bslash \\
~~~~~~~~~~~~~~~~~~-I/usr/local/share/enblend/kernels  \bslash \\
~~~~~~~~~~~~~~~~~~-I/usr/share/enblend/kernels \bslash \\
~~~~~~~~~~~~~~~~~~-DENFUSE\_FWHM\_GAUSSIAN=2.354820f \bslash \\
~~~~~~~~~~~~~~~~~~-DENFUSE\_OPTIMUM\_Y=0.5f \bslash \\
~~~~~~~~~~~~~~~~~~-DENFUSE\_USER\_WEIGHT\_FUNCTION=foobar \bslash \\
~~~~~~~~~~~~~~~~~~-DENFUSE\_WIDTH=0.2f \\
enfuse: note: build log \\
enfuse: note:~~~~~ptxas fatal: Unresolved extern function 'foobar'
    \end{terminal}
  \end{maxipage}

  \caption[\acronym{OpenCL} uer-weight error]{%
    \label{ex:opencl-userweight-error}%
    Error output of \App{} for an invalid (here: unknown) user-defined \acronym{OpenCL}
    exposure-weight function.}
\end{exemplar}


\subsubsection[\acronym{OpenCL} Coding Guidelines]{\label{sec:opencl-coding-guidelines}%
  \genidx{coding guidelines!\acronym{OpenCL}}%
  \acronym{OpenCL} Coding Guidelines}

\begin{enumerate}
\item
  Define a weight-function with following signature
  \begin{cxxlisting}
float weight(float y);
  \end{cxxlisting}
  in a text file.  It takes a luminance in the normalized interval~$(0, 1)$ and returns the
  weight.  This function name will become \metavar{SYMBOL} at the \App{} command line.

\item
  Pass filename~\metavar{OPENCL-SOURCE} of the text file and the function name~\metavar{SYMBOL}
  to \App{} as shown above.

  \genidx{environment variable!ENBLEND\_OPENCL\_PATH@\envvar{ENBLEND\_OPENCL\_PATH}}%
\item
  \metavar{OPENCL-SOURCE} can include (\code{\#include "\dots"}) other files.  The include path
  is \envvar{ENBLEND\_OPENCL\_PATH}.  Note that the environment variable starts with
  ``ENBLEND'', not ``ENFUSE''.

\item
  All \metavar{ARGUMENT}s are converted to preprocessor symbols (\code{\#define \dots}) and
  passed to the \acronym{OpenCL}-compiler when it processes \metavar{OPENCL-SOURCE}.  For
  example, adding an \metavar{ARGUMENT} like \sample{EXPONENT=1.25} will add
  \sample{-DEXPONENT=1.25} to the compiler options, which is equivalent to saying
  \begin{cxxlisting}
#define EXPONENT 1.25
  \end{cxxlisting}
  at the top of \metavar{OPENCL-SOURCE}.

  To make the \metavar{ARGUMENT}s optional at the \app{} command line use the
  \begin{cxxlisting}
#ifndef ARGUMENT
#define ARGUMENT ...
#endif
  \end{cxxlisting}
  idiom in the source code and supply a reasonable default.

  \genidx{OpenCL helpers@\acronym{OpenCL} helpers}
\item
  \App{} supplies a set of preprocessor symbols and functions to each user-defined function.
  They ease writing weight functions.  Everyone starts with \sample{ENFUSE\_} (preprocessor
  symbols) or \sample{enfuse\_} (functions) to avoid name collisions and shadowing.

  \begin{codelist}
    \genidx{OpenCL helpers@\acronym{OpenCL} helpers!ENFUSE-OPTIMUM-Y@\code{ENFUSE\_OPTIMUM\_Y}}%
    \gensee{ENFUSE-OPTIMUM-Y@\code{ENFUSE\_OPTIMUM\_Y}}{\acronym{OpenCL} helpers, \code{ENFUSE\_OPTIMUM\_Y}}%
    \genidx{exposure!optimum}%
    \optidx{--exposure-optimum}%
  \item[ENFUSE\_OPTIMUM\_Y]\itemend The value of the optimum luminance in the normalized
    luminance interval.  The user sets this value with option~\option{--exposure-optimum}
    (default:~\val{val:minimum-exposure-optimum}, see \fullref{opt:exposure-optimum}).

    \genidx{OpenCL helpers@\acronym{OpenCL} helpers!ENFUSE-WIDTH@\code{ENFUSE\_WIDTH}}%
    \gensee{ENFUSE-WIDTH@\code{ENFUSE\_WIDTH}}{\acronym{OpenCL} helpers, \code{ENFUSE\_WIDTH}}%
    \genidx{exposure!weight}%
    \optidx{--exposure-width}%
  \item[ENFUSE\_WIDTH]\itemend The characteristic width of the weight function.  Set this value
    at the command line with option~\option{--exposure-width}
    (default:~\val{val:default-exposure-width}, see \fullref{opt:exposure-width}).

    \genidx{OpenCL helpers@\acronym{OpenCL} helpers!ENFUSE-FWHM-GAUSSIAN@\code{ENFUSE\_FWHM\_GAUSSIAN}}%
    \gensee{ENFUSE-FWHM-GAUSSIAN@\code{ENFUSE\_FWHM\_GAUSSIAN}}%
           {\acronym{OpenCL} helpers, \code{ENFUSE\_FWHM\_GAUSSIAN}}%
  \item[ENFUSE\_FWHM\_GAUSSIAN]\itemend Shorthand for the width of \App's first ever weight
    function.  Use the value to rescale the width of the newly written user function to make it
    bahave like a predefined one with respect to the width.

    The actual value, $2 \sqrt{2 \log(2)}$, is approximately 2.35482.

    \genidx{OpenCL helpers@\acronym{OpenCL} helpers!enfuse-normalized-luminance@\code{enfuse\_normalized\_luminance}}%
    \gensee{enfuse-normalized-luminance@\code{enfuse\_normalized\_luminance}}%
           {\acronym{OpenCL} helpers, \code{enfuse\_normalized\_luminance}}%
  \item[enfuse\_normalized\_luminance]\itemend Helper function which implements
    \equationabbr~\ref{equ:linear-luminance-transform}.
    \begin{cxxlisting}
float enfuse_normalized_luminance(float y)
{
    return (y - ENFUSE_OPTIMUM_Y) / ENFUSE_WIDTH;
}
    \end{cxxlisting}
  \end{codelist}
\end{enumerate}

\begin{geeknote}
  The weight function's signature being so simple and weight functions usually also being simple
  none of the language-specific features like, for example, \code{local}-\slash\code{global}
  address spaces, synchronization or vector-data types are requied to write useful weight
  functions.
\end{geeknote}

Examples~\ref{ex:opencl-generalized-gaussian-weight} and
\ref{ex:variable-opencl-exposure-weight-function} show how weight functions can be constructed
using \acronym{OpenCL}.  Example~\ref{ex:opencl-generalized-gaussian-weight} implements
\[
    w(z) = \exp\left(-|z|^{\mathtt{EXPONENT}}\right),
\]
where $\code{EXPONENT} = 2$ obviously duplicates the pre-defined \propername{Gaussian} weight,
\equationabbr~\ref{equ:weight:gauss}.

\begin{exemplar}[htbp]
  \begin{cxxlisting}
#ifndef EXPONENT
#define EXPONENT 2.0f
#endif

float weight(float y)
{
    return exp(-pow(fabs(enfuse_normalized_luminance(y)),
                    EXPONENT));
}
  \end{cxxlisting}

  \caption[Generalized \propername{Gauss} weight function]{%
    \label{ex:opencl-generalized-gaussian-weight}%
    Generalized \propername{Gauss} weight function written in \acronym{OpenCL}.  Note the
    definition of a default parameter, \code{EXPONENT}.}
\end{exemplar}

Say \exampleName~\ref{ex:opencl-generalized-gaussian-weight} is stored in
\filename{generalized\_gaussian.cl} and reachable via an element in
\envvar{ENBLEND\_OPENCL\_PATH} then is can be used as
\begin{literal}
  \app{} --gpu --exposure-weight-function=\feasiblebreak
  generalized\_gaussian.cl:\feasiblebreak
  weight \dots
\end{literal}
or
\begin{literal}
  \app{} --gpu --exposure-weight-function=\feasiblebreak
  generalized\_gaussian.cl:\feasiblebreak
  weight:\feasiblebreak
  1.5 \dots
\end{literal}

\exampleName~\ref{ex:variable-opencl-exposure-weight-function} implements the same variable
weight as \exampleName~\ref{ex:variable-dynamic-exposure-weight-function} without the parameter
checks:
\[
    w(z) =
    \left\{
      \begin{array}{cl}
        1 - |z|^{\mathtt{EXPONENT}} & \mbox{if } |z| \leq 1 \\
        0                         & \mbox{otherwise.}
      \end{array}
    \right.
\]

\begin{exemplar}[htbp]
  \begin{cxxlisting}
#ifndef M_LN2
#define M_LN2 0.69314718f
#endif

#ifndef EXPONENT
#define EXPONENT 2.0f
#endif

float normalized_luminance(float y)
{
    const float fwhm = 2.0f / exp(M_LN2 / EXPONENT);

    return enfuse_normalized_luminance(y) *
           (fwhm / ENFUSE_FWHM_GAUSSIAN);
}

float variable_power_weight(float y)
{
    return max(1.0f - pow(fabs(normalized_luminance(y)),
                          EXPONENT),
               0.0f);
}
  \end{cxxlisting}

  \caption[\acronym{OpenCL} exposure weight function with an extra argument]{%
    \label{ex:variable-opencl-exposure-weight-function}%
    \acronym{OpenCL} exposure weight function with extra argument~\code{EXPONENT} that defaults
    to 2.  Also compare with \exampleName~\ref{ex:variable-dynamic-exposure-weight-function} for
    the implementation of almost the same functionality using dynamic linking.}
\end{exemplar}

Say \exampleName~\ref{ex:variable-opencl-exposure-weight-function} is stored in
\filename{variable\_power.cl} and reachable via an element in
\envvar{ENBLEND\_OPENCL\_PATH} then is can be used as
\begin{literal}
  \app{} --gpu --exposure-weight-function=\feasiblebreak
  variable\_power.cl:\feasiblebreak
  variable\_power\_weight \dots
\end{literal}
or
\begin{literal}
  \app{} --gpu --exposure-weight-function=\feasiblebreak
  variable\_power.cl:\feasiblebreak
  variable\_power\_weight:\feasiblebreak
  3 \dots
\end{literal}

\optidx{--exposure-weight}%
\optidx{--exposure-optimum}%
\optidx{--exposure-width}%
The options~\option{--exposure-weight}, \option{--exposure-optimum} and
\option{--exposure-width} work as expected for user-defined functions, too.

\begin{quote}
  All debts have now been paid.  --
  \genidx{Linea@\propername{Linea}}%
  \gensee{Ke'ra@\propername{Ke'ra}}{\propername{Linea}}%
  \propername{Linea}, Destroyer of Worlds (\propername{Ke'ra})
\end{quote}


%%% Local Variables:
%%% fill-column: 96
%%% End:
