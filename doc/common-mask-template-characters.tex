%% This file is part of Enblend.
%% Licence details can be found in the file COPYING.


\begin{table}[htbp]
  \centering
  \begin{tabular}{cp{.8\linewidth}}
    \hline
    \multicolumn{1}{c|}{Format} & \multicolumn{1}{c}{Interpretation} \\
    \hline\extraheadingsep
    \code{\%\%}\genidx{mask!template character!\%@\sample{\%}} &
    Produces a literal \sample{\%}-sign. \\


    \code{\%i}\genidx{mask!template character!i@\sample{i}} & Expands to
    the index of the mask file starting at zero.  \sample{\%i} allows
    for setting a pad character or a width specification:

    \begin{literal}
      \% \metavar{PAD} \metavar{WIDTH} i
    \end{literal}

    \metavar{PAD} is either \sample{0} or any punctuation character;
    the default pad character is \sample{0}.  \metavar{WIDTH} is an
    integer specifying the minimum width of the number.  The default
    is the smallest width given the number of input images, this is 1
    for 2--9~images, 2 for 10--99~images, 3 for 100--999~images, and
    so on.

    Examples: \sample{\%i}, \sample{\%02i}, or \sample{\%\_4i}. \\


    \code{\%n}\genidx{mask!template character!n@\sample{n}} & Expands to
    the number of the mask file starting at one. Otherwise it behaves
    identically to \sample{\%i}, including pad character and width
    specification. \\


    \code{\%p}\genidx{mask!template character!p@\sample{p}} & This is
    the full name (path, filename, and extension) of the input file
    associated with the mask.

    Example: If the input file is called
    \filename{/home/luser/snap/img.jpg}, \sample{\%p} expands to
    \filename{/home/luser/snap/img.jpg}, or shorter: \sample{\%p}
    \result{} \filename{/home/luser/snap/img.jpg}. \\


    \code{\%P}\genidx{mask!template character!P@\sample{P}} & This is
    the full name of the output file. \\


    \code{\%d}\genidx{mask!template character!d@\sample{d}} & Is
    replaced with the directory part of the associated input file.

    Example (cont.): \sample{\%d} \result{}
    \filename{/home/luser/snap}. \\


    \code{\%D}\genidx{mask!template character!D@\sample{D}} & Is
    replaced with the directory part of the output file. \\


    \code{\%b}\genidx{mask!template character!b@\sample{b}} & Is
    replaced with the non-directory part (often called ``basename'')
    of the associated input file.

    Example (cont.): \sample{\%b} \result{} \filename{img.jpg}. \\


    \code{\%B}\genidx{mask!template character!B@\sample{B}} & Is
    replaced with the non-directory part of the output file. \\


    \code{\%f}\genidx{mask!template character!f@\sample{f}} & Is
    replaced with the filename without path and extension of the
    associated input file.

    Example (cont.): \sample{\%f} \result{} \filename{img}. \\


    \code{\%F}\genidx{mask!template character!F@\sample{F}} & Is
    replaced with the filename without path and extension of the
    output file. \\


    \code{\%e}\genidx{mask!template character!e@\sample{e}} & Is
    replaced with the extension (including the leading dot) of the
    associated input file.

    Example (cont.): \sample{\%e} \result{} \filename{.jpg}. \\


    \code{\%E}\genidx{mask!template character!E@\sample{E}} & Is
    replaced with the extension of the output file.
  \end{tabular}

  \caption[Mask template characters]{Special format characters to
    control the generation of mask filenames.  Uppercase letters refer
    to the output filename and lowercase ones to the input
    files.\label{tab:mask-template-characters}}

  \genidx[\summarylocation]{mask!template character}
\end{table}
