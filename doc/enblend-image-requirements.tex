%% This file is part of Enblend.
%% Licence details can be found in the file COPYING.


\section[Image Requirements]{Input Image Requirements
  \label{sec:image-requirements}
  \genidx{image requirements}
  \genidx{requirements!image}}

All input images for \App{} must comply with the following
requirements.

\begin{itemize}
\item
  Parts of the images overlap.

\item
  Each image has an alpha channel also called ``mask''.

\item
  The images agree on their number of channels:
  \begin{itemize}
  \item
    one plus alpha or
  \item
    three plus alpha.
  \end{itemize}


  This is, either all images are black-and-white (one channel and alpha
  channel) or all are \acronym{RGB}-color images (three channels and
  alpha channel).

\item
  The images agree on their number of bits-per-channel, i.e., their
  ``depth'':
  \begin{itemize}
  \item
    \code{uint8},
  \item
    \code{uint16},
  \item
    \code{float},
  \item
    etc.
  \end{itemize}

  See option~\flexipageref{\option{--depth}}{opt:depth} for an
  explanation of different output depths.

\item
  \App{} understands the images' filename extensions as well as their
  file formats.

  You can check the supported extensions and formats by calling \App{}
  with
  option~\flexipageref*{\option{--show-image-formats}}{opt:show-image-formats}.
\end{itemize}

Moreover, there are some good practices, which are not enforced by the
application, but almost certainly deliver superior results.

\begin{itemize}
\item
  Either all files lack an \acronym{ICC} profile, or all images are
  supplied with the \emph{same} \acronym{ICC} profile.

\item
  If the images' meta-data contains resolution information
  (``\acronym{DPI}''), it is the same for all pictures.
\end{itemize}
