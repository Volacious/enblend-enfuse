%% This file is part of Enblend.
%% Licence details can be found in the file COPYING.

\ifhevea
  \let\thesectionoriginal\thesection
  \renewcommand{\thesection}{Frontmatter \Alph{section}}
  \section{\label{sec:notation}%
    \genidx{notation}%
    \gensee{typographic conventions}{notation}%
    \gensee{conventions!typographic}{notation}%
    Notation}
  \let\thesection\thesectionoriginal
\else
  \section*{\label{sec:notation}%
    \genidx{notation}%
    \gensee{typographic conventions}{notation}%
    \gensee{conventions!typographic}{notation}%
    Notation}
\fi

This manual uses some typographic conventions to clarify the subject.  The markup of the
ready-to-print version differs from the web markup.

\begin{center}
  \begin{tabular}{p{.18\linewidth}p{.39\linewidth}l}
    \hline
    \multicolumn{1}{c|}{Category} & \multicolumn{1}{c|}{Description} & \multicolumn{1}{c}{Examples} \\
    \hline\extraheadingsep
    Acronym & Common acronym & \acronym{sRGB}, \acronym{OpenMP} \\

    Application & \acronym{GUI} or \acronym{CLI} application & \application{Hugin}, \App \\

    Command & Name of a binary in the running text & \command{convert}, \appcmd \\

    Common part & Chapter, section, or any other part that appears in both manuals & Response
    Files\commonpart \\

    Default & Value as compiled into the \appcmd{} binary that belongs to this documentation &
    \indicatesourcevalue{1}, \indicatesourcevalue{\filename{a.tif}} \\

    Environment variable & Variable passed to \app{} by the operating system & \envvar{PATH},
    \envvar{TMPDIR} \\

    Filename & Name of a file in the filesystem & \filename{a.tif} \\

    Filename extension & Name of a filename extension with or without dots & \filename{.png},
    \filename{tiff}\\

    Fix me! & Section that needs extending or at least some improvement & \fixme{Explain} \\

    Literal text & Text that (only) makes sense when typed in exactly as shown & \code{uint16}
    \\

    Option & Command-line option given to \app & \option{--verbose} \\

    Optional part & Optional part of a syntax description in square brackets &
    \option{--verbose}~\optional{=\metavar{LEVEL}} \\

    Placehoder & Meta-syntactic variable that stands in for the actual text &
    \metavar{ICC-PROFILE} \\

    Proper name & Name of a person or algorithm & \propername{Dijkstra} \\

    Restricted note & Annotation that applies only to a particular program, configuration,~or
    operating system & \restrictednote{\App.} \\

    Sample & Literal text in quotes & \sample{\%} or \sample{--prefer-gpu} \\

    Side note & Non-essential or ``geeky'' material & {\geekytext Gory details} \\

    White space & Indispensable white space &
    \code{r}\textvisiblespace\code{g}\textvisiblespace\code{b} \\
  \end{tabular}
\end{center}

\ifhevea\relax\else
If identifiers like for example \option{--show-soft\shyp ware-com\shyp po\shyp nents}
or \envvar{EN\shyp BLEND\_OPEN\shyp CL\_PATH} must be broken at the end of a line, the break is
indicated with an additional character that does not occur if the identifier is written as one
word with a \sample{\signalinghyphenchar}-character.
\fi


%%% Local Variables:
%%% fill-column: 96
%%% End:
