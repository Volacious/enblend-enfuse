%% This file is part of Enblend.
%% Licence details can be found in the file COPYING.


\chapter[Applications of \App]{\label{sec:applications}%
  \genidx[\rangebeginlocation]{applications of \App}%
  Applications of \App}

This section describes some of the novel possibilities that \App{} offers the photographer.  In
contrast to the previous chapters, it centers around the effects on the final image.

\section[What Makes Images Fusable?]{\label{sec:what-makes-images-fusable}%
  \genidx{images!fusable}%
  What Makes Images Fusable?}

\begin{sloppypar}
  Images should align well to be suitable for fusion.  However, there is no hard mathematical
  rule what ``well'' means.  The alignment requirements for 16~MPixel images to yield a sharp
  $4" \times 6"$ print at 300\dmn{dpi} or even for web presentation are relatively low, whereas
  the alignment of 8~MPixel images for a $12" \times 18"$ print ought to be tight.
\end{sloppypar}

If the input images need to be aligned, \application{Hugin}\appidx{Hugin} (see also
Appendix~\fullref{app:hugin}) is the tool of choice.  It produces images exactly in the format
that \App{} expects.

Sometimes images naturally align extremely well so that no re-alignment is required.  An image
series with preprogrammed exposure steps taken in rapid succession where the camera is mounted
on a heavy tripod and a humongous ball head, mirror lockup, and a cable release are used, comes
to mind.

When in doubt about what will work, try it, and judge for yourself.

\noindent Useful ideas for a good alignment:
\begin{itemize}
\item
  Fix all camera parameters that are not explicitly varied.

  \begin{description}
  \item[Aperture:] Engage full manual or aperture-priority mode.

  \item[Auto-focus:] Disable ``Auto Focus''.  Be aware that the auto-focus function could be
    linked to shutter-release button position ``half pressed'' or to the shutter release in
    insidious ways.

  \item[Closed eyepiece:] Close the eyepiece when using a cable release to suppress variations
    in stray light.  (This tip applies only to single lens reflex cameras.)

  \item[Exposure time/Shutter speed:] Use the shortest possible exposure time or, in other
    words, use the fastest shutter speed to avoid blur caused by camera shake or motion blur.

  \item[Flash power:] Explicitly control the flash power of \emph{all} flashes.  This is
    sometimes called ``flash exposure lock''.

  \item[Sensitivity:] Disable ``Auto \acronym{ISO}''.

  \item[White balance:] Disable ``Auto White Balance''.  Instead, use the most suitable fixed
    white balance or take the white balance off a white card.  When in doubt, use the setting
    ``Daylight'' or equivalent.
  \end{description}

\item
  Steady the camera by any means.

  \begin{itemize}
  \item
    Apply your best camera bracing technique combined with controlled breathing.

  \item
    Prefer a monopod, or better, a rigid tripod with a heavy head.

  \item
    Use a cable release if possible.

  \item
    (This applies to cameras with a moving mirror only.)  Engage ``mirror lockup''.

  \item
    Consider automatic bracketing when applicable.

  \item
    Activate camera- or lens-based image stabilization if you are sure that it improves the
    image quality in your particular case; otherwise disengage the feature.

    For some lens-based image stabilization systems, it is known that they ``lock'' into
    different positions every time they are activated.  Moreover, some stabilization systems
    decrease the image quality if the lens is mounted on a tripod.
  \end{itemize}

\item
  Fire in rapid succession.
\end{itemize}


%% http://www.usa.canon.com/dlc/controller?act=GetArticleAct&articleID=1786


\section[Repetition -- Noise Reduction]{\label{sec:repetition}%
  \genidx{series!simple}%
  \gensee{simple series}{series, simple}%
  \genidx{noise reduction}%
  Repetition -- Noise Reduction}

\mainpurpose{Reduce noise}

With the default settings, \App{} computes a weighted average of the input pixels.  For a series
of images, repeated with identical camera settings, this results in a reduction of photon~shot
noise.  In other words, the dynamic range increases slightly, because the higher signal-to-noise
ratio (\acronym{SNR}) makes darker shades usable.  Furthermore, smooth or glossy surfaces get a
``cleaner'' look, and edges become visually sharper.  The nitty-gritty reportage look that
sometimes stems from a high sensitivity setting disappears.

\genidx{dark-frame subtraction}%
Averaged images, and therefore low-noise images, are the base for a multitude of techniques
like, for example, differences.  The most prominent one in this class is dark-frame subtraction.


\section[Exposure Series -- Dynamic Range Increase]{\label{sec:exposure-series--dynamic-range-increase}%
  \genidx{series!exposure}%
  \gensee{exposure series}{series, exposure}%
  \genidx{dynamic range increase}%
  Exposure Series -- Dynamic Range Increase}

\mainpurpose{Increase manageable dynamic range}

An exposure series is a set of images taken with identical parameters except for the exposure
time.  Some cameras even provide special functions to automate recording exposure series.  See
the instruction manual of your model for details.  Also check out the features that
\uref{\magiclanternfm}{Magic Lantern} offers to shoot \acronym{HDR}-series and to extend the
dynamic range right in the camera.

\App{}'s defaults for $\mbox{exposure weight} = \val*{val:default-weight-exposure}$ and sets all
other weights to zero which generally is well suited for fusion of \emph{color} and
black-and-white images.

Saturation weighting only works for \acronym{RGB}~data.  For color images,
option~\option{--saturation-weight} helps to control burnt-out highlights, if they are heavily
desaturated.  Alternatively, use option~\option{--exposure-cutoff} to suppress noise or
blown-out highlights without altering the overall brightness too much.  If no image suffers from
troublesome highlights, the relative saturation weight can be reduced and even be set to zero.

For black-and-white images \sample{--entropy-weight} can be an alternative to
\sample{--saturation-weight} because it suppresses overexposed pixels, as these contain little
information.  However, entropy weighting is not limited to grayscale data; it has been
successfully applied to \acronym{RGB} images, too.  Note that entropy weighting considers
\emph{each} color channel of an \acronym{RGB} image separately and chooses the channel with the
minimum entropy as representative for the whole pixel.

\App{} offers the photographer tremendous flexibility in fusing differently exposed images.
Whether you combine only two pictures or a series of 21, \App{} imposes no limits on you.
Accordingly, the photographic effects achieved range from subtle to surreal, like the late~1980s
``Max Headroom'' \acronym{TV}-Series, to really unreal.  Like some time ago in the chemical days
of photography, when a new developer opened unseen possibilities for artists, exposure fusion
extends a photographer's expressive space in the digital age.  Whether the results look good or
bad, whether the images are dull or exciting, is entirely up the artist.

In the next sections we give assistance to starters, and rectify several misconceptions about
\App{}.


\subsection[Tips For Beginners]{\label{sec:tips-for-beginners}%
  \genidx{exposure series!tips for beginners}%
  Tips For Beginners}

Here are some tips to get you in business quickly.

\begin{description}
\item[Include the best single exposure.]\itemend
  Include the exposure you would have taken if you did not use \App{} in your series.  It gives
  you a solid starting point.  Think of the other images as augmenting this best single exposure
  to bring out the light and dark features you would like to see.

\item[Begin with as small a number of images as possible.]\itemend
  Pre-visualizing the results of \App{} is difficult.  The more images put into the fusion
  process and the wider their \acronym{EV}-spacing is, the more challenging visualizing the
  output image becomes.  Therefore, start off with as few images as possible.

  You can take a larger series of images and only use part of it.

\item[Start with a moderate \acronym{EV}-spacing.]\itemend
  As has been pointed out in the previous item, a wide \acronym{EV}-spacing makes
  pre-visualization harder.  So start out with a spacing of $\slfrac{2}{3}$\dmn{EV} to
  $\slfrac{4}{3}$\dmn{EV}.

\item[Use Magic Lantern's dual-\acronym{ISO} mode (and avoid \App).]\itemend
  Magic Lantern can reprogram a camera to take a \emph{single} short with \emph{two different}
  sensor speeds.  They call it ``dual-\acronym{ISO}'' mode.  The non-standard RAW~image created
  that way must be processed with \command{cr2hdr}\footnote{\command{cr2hdr} is part of Magic
    Lantern's source distribution.}, which produces a \acronym{DNG}~file readable with the usual
  RAW~converters.
\end{description}


\subsection[Common Misconceptions]{\label{sec:common-misconceptions}%
  \genidx{exposure series!common misconceptions}%
  Common Misconceptions}

Here are some surprisingly common misconceptions about exposure series.

\begin{description}
  \genidx{digital blending}%
  \genidx{blending exposures}%
\item[A single image cannot be the source of an exposure series.]\itemend
  Raw-files in particular lend themselves to be converted multiple times and the results being
  fused together.  The technique is simpler, faster, and usually even looks better than
  \uref{\luminouslandscapedigitalblending}{digital blending} (as opposed to using a graduated
  neutral density filter) or \uref{\gimpguruorgblendingexposures}{blending exposures} in an
  image manipulation program.  Moreover, perfect alignment comes free of charge!

\item[An exposure series must feature symmetrically-spaced exposures.]\itemend
  Twice wrong!  Neither do the exposures have to be ``symmetric'' like \{0\dmn{EV},
  $-\slfrac{2}{3}$\dmn{EV}, $+\slfrac{2}{3}$\dmn{EV}\}, nor does the number of exposures have to
  be odd.  Series like \{$-\slfrac{4}{3}$\dmn{EV}, $-\slfrac{1}{3}$\dmn{EV},
  $+\slfrac{1}{3}$\dmn{EV}\} or \{$-1$\dmn{EV}, 1\dmn{EV}\} might be just right.  By the way,
  the order in which the images were taken does not matter either.

  \genidx{light probe}%
\item[An exposure series must cover the whole dynamic range of the scene.]\itemend
  If you do not want to cover the whole range, you do not have to.  Some \acronym{HDR} programs
  require the user to take a light probe, whereas \App{} offers the user complete freedom of
  exposure.

  \begin{geeknote}
    \uref{\debevecorg}{Paul E.~Debevec} defines: ``A light probe image is an omnidirectional,
    high dynamic range image that records the incident illumination conditions at a particular
    point in space.''
  \end{geeknote}

\item[All exposure values must be different.]\itemend
  You can repeat any exposure as often as you like.  That way you combine an exposure series
  with parts of \sectionName~\ref{sec:repetition}, emphasizing the multiply occurring exposures
  and reducing noise.
\end{description}


\section[Flash Exposure Series -- Directed Lighting]{%
  \label{sec:flash-exposure-series--directedlighting}%
  \genidx{series!flash exposure}%
  \gensee{flash exposure series}{series, flash exposure}%
  \genidx{dynamic range increase}%
  Flash Exposure Series -- Directed Lighting}

\mainpurpose{???}

\fixme{Text?}


\section[Polarization Series -- Saturation Enhancement]{%
  \label{sec:polarization-series--saturation-enhancement}%
  \genidx{series!polarization}%
  \gensee{polarization series}{series, polarization}%
  \genidx{saturation enhancement}%
  Polarization Series -- Saturation Enhancement}

\mainpurpose{Reflection suppression, saturation enhancement}

In the current implementation of \App{}, it is not possible in general to fuse a polarization
series.  Na\"ively (ab)using \sample{--saturation-weight} will not work.


\section[Focus Stacks -- Depth-of-Field Increase]{%
  \label{sec:focus-stacks--depth-of-field-increase}%
  \genidx{focus stacks}%
  \genidx{depth-of-focus increase}%
  Focus Stacks -- Depth-of-Field Increase}

\mainpurpose{Synthetic Depth-of-Field Increase}

\genidx{depth-of-field}%
\gensee{DoF@\acronym{DoF}}{depth-of-field}%
A focus stack is a series of images where the distance of the focal plane from the sensor
varies.  Sloppily speaking, the images were focused at different distances.  Fusing such a stack
increases the depth-of-field (\acronym{DoF}) beyond the physical limits of diffraction.


\subsection[Why create focus stacks?]{\label{sec:why-create-focus-stacks}%
  \genidx{focus stacks!why to create them?}%
  Why create focus stacks?}

Given

\begin{itemize}
\item
  a fixed sensor or film size,

\item
  a lens' particular focal length, and

  \genidx{circle-of-confusion}%
  \gensee{CoC@\acronym{CoC}}{circle-of-confusion}%
\item
  a notion about ``sharpness'', technically speaking the size of the circle-of-confusion
  (\acronym{CoC})
\end{itemize}

\genidx{depth-of-field}%
\noindent the photographer controls the depth-of-field with the aperture.  Smaller apertures --
this is larger aperture numbers -- increase the \acronym{DoF} and vice versa.  However, smaller
apertures increase diffraction which in turn renders the image unsharp.  So, there is an optimum
aperture where the photographer gets maximum \acronym{DoF}.  Sadly, for some purposes like macro
shots it is not enough.  One way out is to combine the sharp parts of images focused at
different distances, thereby artificially increasing the total \acronym{DoF}.  This is exactly
what \App{} can do.

\genidx{aperture!sweet spot}%
\gensee{sweet spot aperture}{aperture, sweet spot}%
All lenses have a so called ``sweet spot'' aperture, where their resolution is best.  Taking
pictures at this aperture, the photographer squeezes the maximum quality out of the lens.  But:
the ``sweet spot'' aperture often is only one or two stops away from wide open.  Wouldn't it be
great to be able combine these best-possible images to form one high-quality,
sufficient-\acronym{DoF} image?  Welcome to \App{}'s local-contrast selection abilities.


\subsection[Preparing Focus Stacks]{\label{sec:preparing-focus-stacks}%
  \genidx{focus stacks!preparation}%
  Preparing Focus Stacks}

We are going to combine images with limited \acronym{DoF} to increase their in-focus parts.  The
whole process is about image sharpness.  Therefore, the input images must align very well, not
just well, but very well.  For optimum results the maximum control point distance in
\application{Hugin} should not exceed 0.3--0.5~pixels to ensure perfect blending.

\genidx{signal-to-noise ratio}%
\gensee{SNR@\acronym{SNR}}{signal-to-noise ratio}%
As in all image fusion operations it is preferable to use 16~bit linear, this is $\mathrm{gamma}
= 1$ images throughout, but 8~bit gamma-encoded images will do.  Naturally, high signal-to-noise
(\acronym{SNR}) ratio input data always is welcome.


\subsection[Local Contrast Based Fusing]{\label{sec:local-contrast-based-fusing}%
  \genidx{local-contrast-based fusing}%
  \genidx{fusing!local-contrast-based}%
  \genidx{focus stacks!fusing}%
  Local Contrast Based Fusing}

A bare bones call to \App{} for focus stacking could look like this.

\begin{terminal}
  \$ enfuse \bslash \\
  ~~~~--exposure-weight=0 \bslash \\
  ~~~~--contrast-weight=1 \bslash \\
  ~~~~--hard-mask \bslash \\
  ~~~~\dots{} \bslash \\
  ~~~~--output=output.tif \bslash \\
  ~~~~input-<0000-9999>.tif
\end{terminal}

Here is what each option causes:

\begin{codelist}
\item[--exposure-weight=0]\itemend
  Switch \strong{off} exposure based pixel selection.  The default weight is
  \val{val:default-weight-exposure}.

\item[--contrast-weight=1]\itemend
  Switch \strong{on} pixel selection based on local contrast.

\item[--hard-mask]\itemend
  Select the best pixel from the image stack and ignore all others.  Without this option, \App{}
  uses all pixels in the stack and weights them according to their respective quality, which in
  our case is local contrast.  Without \sample{--hard-mask}, the result will always look a bit
  soft.  See also \sectionName~\fullref{sec:local-contrast-weighting}.
\end{codelist}

If you want to see some entertaining progress messages -- local-contrast weighting takes a while
--, also pass the \option{--verbose}~option for an entertaining progress report.


\subsection[Basic Focus Stacking]{\label{sec:basic-focus-stacking}%
  \genidx{focus stacking!basic}%
  \gensee{basic focus stacking}{focus stacking, basic}%
  Basic Focus Stacking}

For a large class of image stacks \App{}'s default algorithm, as selected in
\ref{sec:local-contrast-based-fusing}, to determine the sharpness produces nice results.  The
algorithm uses a moving square window, the so-called contrast window.  It computes the standard
deviation of the pixels inside of the window.  The program then selects the window's center
pixel of the image in the stack where the standard deviation is largest, that is, the local
contrast reaches the maximum.

However, the algorithm fails to deliver good masks for images which exhibit high contrast edges
on the scale of the contrast window size.  The typical artifacts that show up are

\begin{itemize}
\item
  faint dark seams on the light side of the high contrast edges and

\item
  extremely soft, slightly lighter seams on the dark side of the high contrast edges,
\end{itemize}

\noindent where the distance of the seams from the middle of the edge is comparable to the
contrast window size.

If your results do not show any of these artifacts, stick with the basic algorithm.  Advanced
focus stacking, as described in the next sections, delivers superior results in case of
artifacts, though requires manually tuning several parameters.


\subsection[Advanced Focus Stacking]{\label{sec:advanced-focus-stacking}%
  \genidx{focus stacking!advanced}%
  \gensee{advanced focus stacking}{focus stacking, advanced}%
  Advanced Focus Stacking}

If your fused image shows any of the defects described in the previous section, you can try a
more difficult-to-use algorithm that effectively works around the seam artifacts.  It is
described in the next section.


\subsubsection[Detailed Look at the Problem]{\label{sec:detailed-look-at-the-problem}%
  \genidx{local contrast problem}%
  Detailed Look at the Problem}

Let us use an example to illustrate the problem of relating the sharpness with the local
contrast variations.  Say we use a $5 \times 5$ contrast window.  Moreover, let
\code{sharp\_edge} and \code{smooth\_edge} be two specific configurations:

\begin{literal}
  sharp\_edge~=~~[~~~~0,  ~~0,  200,  ~~0,  ~~0; \\
    ~~~~~~~~~~~~~~~~~~~0,  225,  ~~0,  ~~0,  ~~0; \\
    ~~~~~~~~~~~~~~~~~~~0,  255,  ~~0,  ~~0,  ~~0; \\
    ~~~~~~~~~~~~~~~~~215,  ~~0,  ~~0,  ~~0,  ~~0; \\
    ~~~~~~~~~~~~~~~~~200,  ~~0,  ~~0,  ~~0,  ~~0]
\end{literal}

\begin{literal}
  smooth\_edge~=~[~~~~0,  ~62,  125,  187,  250; \\
    ~~~~~~~~~~~~~~~~~~~1,  ~63,  126,  188,  251; \\
    ~~~~~~~~~~~~~~~~~~~2,  ~65,  127,  190,  252; \\
    ~~~~~~~~~~~~~~~~~~~3,  ~66,  128,  191,  253; \\
    ~~~~~~~~~~~~~~~~~~~5,  ~67,  130,  192,  255]
\end{literal}

\noindent where \sample{;} separates the rows and \sample{,} separates the columns.  This is in
fact \uref{\gnuorgoctave}{Octave} syntax.  \figureName~\ref{fig:sharp-edge}
and~\ref{fig:smooth-edge} show plots of the matrices \code{sharp\_edge} and~\code{smooth\_edge}.


\begin{figure}
  \begin{maxipage}
    \centering
    \includeimage{sharp-edge}
  \end{maxipage}

  \caption[Sharp edge]{\label{fig:sharp-edge}%
    3D-plot augmented by contour plot of the matrix~\code{sharp\_edge}.}
\end{figure}


\begin{figure}
  \begin{maxipage}
    \centering
    \includeimage{smooth-edge}
  \end{maxipage}

  \caption[Smooth edge]{\label{fig:smooth-edge}%
    3D-plot augmented by contour plot of the matrix~\code{smooth\_edge}.}
\end{figure}


Our intuition lets us ``see'' an extremely sharp edge in the first matrix, whereas the second
one describes an extraordinarily smooth diagonal intensity ramp.  Which one will be selected?
Well, \code{sharp\_edge} has a standard deviation of 88.07 and \code{smooth\_edge} has 88.41.
Thus, \code{smooth\_edge} wins, contradicting our intuition, and even worse, our intention!
Sadly, configurations like \code{smooth\_edge} occur more often with high-quality, good
\uref{\luminouslandscapebokeh}{bokeh} lenses.  In fact, they are the very manifestation of
``good bokeh''.  Therefore, Laplacian edge detection plays an important role when working with
high-quality lenses.


\subsubsection[\propername{Laplacian} Edge Detection]{\label{sec:laplacian-edge-detection}%
  \genidx{edge detection!Laplacian@\propername{Laplacian}}%
  \gensee{Laplacian edge detection@\propername{Laplacian} edge detection}%
         {edge detection, \propername{Laplacian}}%
  \propername{Laplacian} Edge Detection}

\App{} provides a Laplacian-based algorithm that can help in situations where weighting based on
the standard deviation fails.  It is activated with a positive value for \metavar{SCALE} in
\option{--contrast-edge-scale}=\metavar{SCALE}.  The Laplacian will detect two-dimensional
\emph{curvature} on the scale of \metavar{SCALE}.  Here and in the following we simply speak of
``curvature'' where we mean ``magnitude of curvature''.  That is, we shall not distinguish
between convex and concave edges.  \App{} always use the magnitude of curvature for weighting.

Typically, \metavar{SCALE} ranges between 0.1~pixels and 0.5~pixels, where 0.3~pixels are a
reasonable starting point.  To find the best value for \metavar{SCALE} though, usually some
experimentation will be necessary.  Use \sample{--save-masks} to get all soft-mask (default:
\mbox{\code{\val{val:default-soft-mask-template}}}) and hard-mask files (default:
\mbox{\code{\val{val:default-hard-mask-template}}}).  Check how different scales affect the
artifacts.  Also see Chapter~\fullref{sec:understanding-masks}.


\subsubsection[Local Contrast Enhancement]{\label{sec:local-contrast-enhancement}%
  \genidx{contrast enhancement!local}%
  \gensee{local contrast enhancement}{contrast enhancement, local}%
  Local Contrast Enhancement}

Sometimes \App{} misses smoother edges with \sample{--contrast-edge-scale} and a little local
contrast enhancement (\acronym{LCE}) helps.  Set \option{--contrast\hyp edge\hyp
  scale}=\feasiblebreak \metavar{SCALE}:\feasiblebreak \metavar{LCE\hyp SCALE}:\feasiblebreak
\metavar{LCE\hyp FACTOR}, where \metavar{LCE\hyp SCALE} and \metavar{LCE\hyp FACTOR} work like
the \uref{\cambridgeincolourcomunsharpmask}{unsharp mask} filters in various image manipulation
programs.  Start with \metavar{LCE\hyp SCALE} ten times the value of \metavar{SCALE} and a
\metavar{LCE\hyp FACTOR} of 2--5.

\metavar{LCE\hyp SCALE} can be specified as a percentage of \metavar{SCALE}.  \metavar{LCE\hyp
  FACTOR} also can be specified as a percentage.  Examples:

\begin{literal}
--contrast-edge-scale=0.3:3.0:3 \\
--contrast-edge-scale=0.3:1000\%:3.0 \\
--contrast-edge-scale=0.3:3:300\% \\
--contrast-edge-scale=0.3:1000\%:300\%
\end{literal}

By default \acronym{LCE} is turned off.


\subsubsection[Suppressing Noise or Recognizing Faint Edges]{%
  \label{sec:suppressing-noise-or-recognizing-faint-edges}%
  \genidx{advanced focus stacking!suppressing noise}%
  \genidx{advanced focus stacking!recognizing faint edges}%
  Suppressing Noise or Recognizing Faint Edges}

The \propername{Laplacian}-based algorithm is much better at resisting the seam problem than the
local-contrast algorithm, but it has two shortcomings:

\begin{compactenumerate}
\item
  The \propername{Laplacian} is very susceptible to noise and

\item
  it fails to recognize faint edges.
\end{compactenumerate}

\noindent The option~\option{--contrast-min-curvature} option helps to mitigate both flaws.

The argument to \option{--contrast-min-curvature}=\metavar{CURVATURE} either is an absolute
lightness value, for example 0\dots255 for 8~bit data and 0\dots65535 for 16~bit data, or, when
given with a \sample{\%}-sign it is a relative lightness value ranging from 0\% to~100\%.

To suppress unreal edges or counter excessive noise, use the \option{--contrast\hyp min\hyp
  curvature}~option with a \emph{negative} curvature measure~\metavar{CURVATURE}.  This forces
all curvatures less than $-\metavar{CURVATURE}$ to zero.

A \emph{positive} curvature measure~\metavar{CURVATURE} makes \App{} merge the \acronym{LoG}
data with the local-contrast data.  Every curvature larger than or equal to \metavar{CURVATURE}
is left unchanged, and every curvature less than \metavar{CURVATURE} gets replaced with the
rescaled local-contrast data, such that the largest local contrast is just below
\metavar{CURVATURE}.  This combines the best parts of both techniques and ensures a precise edge
detection over the whole range of edge curvatures.

\noindent\strong{Summary}

\begin{codelist}
\item[\option{--contrast-edge-scale=0.3}]\itemend
  Use \acronym{LoG} to detect edges on a scale of 0.3~pixels.  Apply the default grayscale
  projector: \code{average}.

\item[\itempar{\option{--contrast-edge-scale=0.3} \option{--gray-projector=l-star}}]\itemend
  Use \acronym{LoG} to detect edges on a scale of 0.3~pixels.  Apply the L*-grayscale projector.

\item[\option{--contrast-edge-scale=0.3:3:300\%}]\itemend
  Use \acronym{LoG} to detect edges on a scale of 0.3~pixels, pre-sharpen the input images by
  300\% on a scale of 3~pixels.  Apply the default grayscale projector: \code{average}.

\item[\itempar{\option{--contrast-edge-scale=0.3}
    \mbox{\option{--contrast-min-curvature=-0.5\%}}}]\itemend
  Use \acronym{LoG} to detect edges on a scale of 0.3~pixels.  Apply the default grayscale
  projector: \code{average} and throw away all edges with a curvature of less than 0.5\%.

\item[\itempar{\mbox{\option{--contrast-edge-scale=0.3}}
    \mbox{\option{--contrast-min-curvature=0.5\%}}
    \mbox{\option{--contrast-window-size=7}}}]\itemend
  Use \acronym{LoG} to detect edges on a scale of 0.3~pixels.  Apply the default grayscale
  projector: \code{average} and throw away all edges with a curvature of less than 0.5\% and
  replace the \acronym{LoG} data between 0\% and 0.5\% with \acronym{SDev} data.  Use a window
  of $7 \times 7$~pixel window to compute the \acronym{SDev}.
\end{codelist}


\subsubsection[Focus Stacking Decision Tree]{%
  \label{sec:focus-stacking-decision-tree}%
  \genidx{decision tree!focus stacking}%
  \gensee{focus stacking decision tree}{decision tree, focus stacking}%
  Focus Stacking Decision Tree}

\figureName~\ref{fig:focus-stacking-decision-tree} helps the user to arrive at a well-fused
focus stack with as few steps as possible.

\begin{figure}
  \begin{maxipage}
    \centering
    \includeimage{focus-stack-decision-tree}
  \end{maxipage}

  \caption[Focus stacking decision tree.]{\label{fig:focus-stacking-decision-tree}%
    Focus stacking decision tree.}
\end{figure}

Always start with the default, contrast weighting with a local contrast window.  Only if seams
appear as described in \sectionName~\ref{sec:advanced-focus-stacking} switch to
\propername{Laplacian}-of-\propername{Gaussian} contrast detection.

If some seams remain even in \acronym{LoG}-mode, decrease the sensitivity of the edge detection
with a positive \code{--contrast\hyp min\hyp curvature}.  A too high value of
\code{--contrast\hyp min\hyp curvature} suppresses fine detail though.  Part of the detail can
be brought back with pre-sharpening, that is, \ref{sec:local-contrast-enhancement} or combining
\acronym{LoG} with local-contrast-window mode by using a negative \code{--contrast\hyp min\hyp
  curvature}.

Carefully examining the masks (option~\option{--save-masks}) that \App{} uses helps to judge the
effects of the parameters.


\subsection[Tips For Focus Stacking Experts]{\label{sec:tips-for-focus-stacking-experts}%
  \genidx{tips!focus stacking experts}%
  \gensee{expert focus stacking tips}{tips, focus stacking experts}%
  Tips For Focus Stacking Experts}

We have collected some advice with which even focus-stacking adepts can benefit.

\begin{itemize}
  \genidx{sensor!use clean}%
\item
  Ensure that the sensor is clean.

  Aligning focus stacks requires varying the viewing angle, which corresponds to a changing
  focal length.  Hence, the same pixel on the sensor gets mapped onto different positions in the
  final image.  Dirt spots will occur not only once but as many times as there are images in the
  stack -- something that is no fun to correct in post-processing.

  \genidx{dark frame}%
  \genidx{subtraction of dark frame}%
  \genidx{hot pixels}%
  \genidx{pixels!hot}%
  Along the same lines, the photographer may want to consider to prepare dark frames before, and
  possibly also after, the shoot of the focus stack, to subtract hot pixels before fusion.

\item
  Prefer a low-sensitivity setting (``\acronym{ISO}'') on the camera to get low-noise images.

  Fusing with option~\option{--hard-mask}\optidx{--hard-mask} does not average, and thus does
  not suppress any noise in the input images.

\item
  If the transition of in-focus to out-of-focus areas is too abrupt, record the images with
  closest and farthest focusing distances twice: first with the intended working aperture, and a
  second time with a small aperture (large aperture number).

  \genidx{natural sharp-unsharp transition}%
  \genidx{transition!natural sharp-unsharp}%
  The small aperture will give the fused image a more natural in-focus to out-of-focus
  transition and the working-aperture shots supply the detail in the in-focus regions.

\item
  Consider the use of Magic Lantern (\appendixName~\fullref{app:magiclantern}) to automate the
  creation of focus stacks.
\end{itemize}

\genidx[\rangeendlocation]{applications of \App}


%%% Local Variables:
%%% fill-column: 96
%%% End:
