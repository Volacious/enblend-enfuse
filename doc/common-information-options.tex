%% This file is part of Enblend.
%% Licence details can be found in the file COPYING.


\subsection[Information Options\commonpart]{\label{sec:information-options}%
  \genidx[\rangebeginlocation]{information options}%
  \genidx{options!information}%
  Information Options\commonpart}

\begin{codelist}
  \label{opt:help}%
  \optidx[\defininglocation]{--help}%
  \shoptidx{-h}{--help}%
\item[\itempar{-h \\ --help}]\itemend
  \genidx{help}Print information on the command-line syntax and all available options, giving a
  boiled-down version of this manual.


  \label{opt:show-globbing-algorithms}%
  \optidx[\defininglocation]{--show-globbing-algorithms}%
\item[--show-globbing-algorithms]\itemend
  \genidx{algorithm!globbing}\gensee{globbing algorithms}{algorithm, globbing}Show all globbing
  algorithms.

  Depending on the build-time configuration and the operating system the binary may support
  different globbing algorithms.  See \sectionName~\fullref{sec:globbing-algorithms}.


  \label{opt:show-gpu-info}%
  \optidx[\defininglocation]{--show-gpu-info}%
\item[--show-gpu-info \restrictednote{\acronym{OpenCL}-enabled versions only.}]\itemend
  Print a list of all available \genidx{GPU@\acronym{GPU}}\acronym{GPU}~devices under
  \genidx{OpenCL@\acronym{OpenCL}}\acronym{OpenCL}~control on all accessible platforms, the
  current preferences, and then exit; it is the same enumeration that
  \begin{literal}
    \app{} --verbose --version
  \end{literal}
  reveals.%
  \genidx{GPU@\acronym{GPU}!information on}%
  \gensee{information!on \acronym{GPU}}{\acronym{GPU}, information}%
  \genidx{OpenCL@\acronym{OpenCL}!information on configuration}%
  \gensee{information!on \acronym{OpenCL} configuration}{\acronym{OpenCL}, information}
  \exampleName~\ref{ex:opencl-config} shows a complete output.

  \begin{exemplar}[htbp]
    \begin{maxipage}
      \centering
      \begin{terminal}
        \$ \app{} --show-gpu-info \\
        Available, OpenCL-compatible platform(s) and their device(s) \\
        ~~- Platform \#1:~Advanced Micro Devices, Inc., AMD, OpenCL 1.2 AMD-APP (938.2) \\
        ~~~~* no GPU devices found on this platform \\
        ~~- Platform \#2:~NVIDIA Corporation, NVIDIA CUDA, OpenCL 1.1 CUDA 4.2.1 \\
        ~~~~* Device \#1:~max.~1024 work-items \\
        ~~~~~~~~~~~~~~~~~1047872 KB global memory with 32 KB read/write cache \\
        ~~~~~~~~~~~~~~~~~48 KB dedicated local memory \\
        ~~~~~~~~~~~~~~~~~64 KB maximum constant memory \\
        ~~Search path (expanding ENBLEND\_OPENCL\_PATH and appending built-in path) \\
        ~~~~/usr/local/share/enblend/kernels:/usr/share/enblend/kernels \\
        Currently preferred GPU is device \#1 on platform \#2 (autodetected).
      \end{terminal}
    \end{maxipage}

    \caption[Sample \acronym{OpenCL} configuration.]{\label{ex:opencl-config}A sample
      \acronym{OpenCL} configuration as detected by \App.}
  \end{exemplar}


  \label{opt:show-image-formats}%
  \optidx[\defininglocation]{--show-image-formats}%
\item[--show-image-formats]\itemend
  \genidx{format!image}\gensee{image formats}{format, image}Show all recognized image formats,
  their filename extensions and the supported per-channel depths.

  Depending on the build-time configuration and the operating system, the binary supports
  different image formats, typically: \acronym{BMP}, \acronym{EXR}, \acronym{GIF},
  \acronym{HDR}, \acronym{JPEG}, \acronym{PNG}, \acronym{PNM}, SUN, \acronym{TIFF},
  and~\acronym{VIFF} and recognizes different image-filename extensions, again typically:
  \filename{bmp}, \filename{exr}, \filename{gif}, \filename{hdr}, \filename{jpeg},
  \filename{jpg}, \filename{pbm}, \filename{pgm}, \filename{png}, \filename{pnm},
  \filename{ppm}, \filename{ras}, \filename{tif}, \filename{tiff}, and~\filename{xv}.

  The maximum number of different per-channel depths any \appcmd{} provides is seven:
  \begin{itemize}
  \item 8~bits unsigned integral -- \sample{uint8}
  \item 16~bits unsigned or signed integral -- \sample{uint16} or \sample{int16}
  \item 32~bits unsigned or signed integral -- \sample{uint32} or \sample{int32}
  \item 32~bits floating-point -- \sample{float}
  \item 64~bits floating-point -- \sample{double}
  \end{itemize}


  \label{opt:show-signature}%
  \optidx[\defininglocation]{--show-signature}%
  \genidx{signature}%
\item[--show-signature]\itemend
  Show the user name of the person who compiled the binary, when the binary was compiled, and on
  which machine this was done.

  This information can be helpful to ensure the binary was created by a trustworthy builder.


  \label{opt:show-software-components}%
  \optidx[\defininglocation]{--show-software-components}%
\item[--show-software-components]\itemend
  \genidx{information!on software components}%
  \gensee{software!components}{information, on software components}%
  Show the name and version of the compiler that built \App{} followed by the versions of all
  important libraries against which \App{} was compiled and linked.

  \genidx{dynamic-library environment}Technically, the version information is taken from
  \genidx{header files}header files, thus it is independent of the dynamic-library environment
  the binary runs within.  The library versions printed here can help to reveal version
  mismatches with respect to the actual dynamic libraries available to the binary.


  \label{opt:version}%
  \optidx[\defininglocation]{--version}%
  \shoptidx{-V}{--version}%
\item[\itempar{-V \\ --version}]\itemend
  \genidx{software!version}%
  \gensee{version}{software, version}%
  \gensee{binary version}{software, version}%
  Output information on the binary's version.

  Team this option with \option{--verbose} to show configuration details, like the extra
  features that may have been compiled in.  For details consult
  \sectionName~\fullref{sec:exact-version}.
\end{codelist}

\genidx[\rangeendlocation]{information options}


%%% Local Variables:
%%% fill-column: 96
%%% End:
