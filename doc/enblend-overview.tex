%% This file is part of Enblend.
%% Licence details can be found in the file COPYING.


\chapter[Overview]{\label{sec:overview}%
  \genidx[\rangebeginlocation]{overview}%
  Overview}

\genidx{Burt@\propername{Burt, Peter J.}}%
\genidx{Adelson@\propername{Adelson, Edward H.}}%
\genidx{Burt-Adelson@\propername{Burt-Adelson}}%
\genidx{multiresolution spline}%
\gensee{spline}{multiresolution spline}%
\begin{sloppypar}
  \App{} overlays multiple images using the \propername{Burt}-\propername{Adelson}
  multiresolution spline multiresolution spline algorithm.\footnotemark{} This technique tries
  to make the seams between the input images invisible.  The basic idea is that image features
  should be blended across a transition zone proportional in size to the spatial frequency of
  the features.  For example, objects like trees and windowpanes have rapid changes in color.
  By blending these features in a narrow zone, you will not be able to see the seam because the
  eye already expects to see color changes at the edge of these features.  Clouds and sky are
  the opposite.  These features have to be blended across a wide transition zone because any
  sudden change in color will be immediately noticeable.%
  %
  \footnotetext{\propername{Peter J.~Burt} and \propername{Edward H.~Adelson}, ``A
    Multiresolution Spline With Application to Image Mosaics'', \acronym{ACM} Transactions on
    Graphics, \abbreviation{Vol}.~2, \abbreviation{No}.~4, October 1983, pages~217--236.}
\end{sloppypar}

\genidx{channel!alpha}%
\gensee{alpha channel}{channel, alpha}%
\App{} expects each input file to have an alpha channel.  The alpha channel should indicate the
region of the file that has valid image data.  \App{} compares the alpha regions in the input
files to find the areas where images overlap.  Alpha channels can be used to indicate to \App{}
that certain portions of an input image should not contribute to the final image.

\genidx{feathering}%
\appidx{Hugin}%
\appidx{PanoTools}%
\App{} does \emph{not} align images.  Use a tool such as \application{Hugin} or PanoTools to do
this.  The \acronym{TIFF}~files produced by these programs are exactly what \App{} is designed
to work with.  Sometimes these \acronym{GUI}s allow to select feathering for the edges the
images.  This treatment is detrimental to \App{}.  Turn off feathering by deselecting it or
setting the ``feather width'' to zero.

\App{} blends the images in the order they are specified on the command line.  You should order
your images according to the way that they overlap, for example from left-to-right across the
panorama.  If you are making a multi-row panorama, we recommend blending each horizontal row
individually, and then running \App{} a last time to blend all of the rows together vertically.
The input images are processed in the order they appear on the command line.  Multi\hyp{}layer
images are processed from the first layer to the last before \App{} considers the next image on
the command line.  Consult \sectionName~\ref{sec:layer-selection} on how to change the images'
order within multi\hyp{}layer image files.

\genidx{SourceForge}Find out more about \App{} on its \uref{\sourceforgenet}{SourceForge}
\uref{\enblendsourceforgenet}{web page}.

\genidx[\rangeendlocation]{overview}


%%% Local Variables:
%%% fill-column: 96
%%% End:
