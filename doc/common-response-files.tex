%% This file is part of Enblend.
%% Licence details can be found in the file COPYING.


\section[Response Files\commonpart]{\label{sec:response-files}%
  \genidx[\rangebeginlocation]{response files}%
  \genidx{file!response}%
  Response Files\commonpart}

A response file contains names of images or other response filenames.
\genidx{\val*{val:response-file-prefix-char}\ (response file prefix)}%
\gensee{response file prefix!\sample{\val*{val:response-file-prefix-char}}}%
       {\sample{\val*{val:response-file-prefix-char}}}
Introduce response file names at the command line or in a response file with an
\code{\val{val:response-file-prefix-char}}~character.

\genidx{order!of image processing}%
\gensee{image processing order}{order of image processing}%
\App{} and \OtherApp{} process the list \metavar{INPUT} strictly from left to right, expanding
response files in depth-first order.  Multi-layer files are processed from first layer to the
last.  The following examples only show \application{Enblend}, but \application{Enfuse} works
exactly the same.

\begin{description}
\item[Solely image filenames.]\itemend
  Example:

  \begin{literal}
    enblend image-1.tif image-2.tif image-3.tif
  \end{literal}

  The ultimate order in which the images are processed is: \filename{image-1.tif},
  \filename{image-2.tif}, \filename{image-3.tif}.

\item[Single response file.]\itemend
  Example:

  \begin{literal}
    enblend \val*{val:response-file-prefix-char} list
  \end{literal}

  where file~\filename{list} contains

  \begin{literal}
    img1.exr \\
    img2.exr \\
    img3.exr \\
    img4.exr \\
  \end{literal}

  Ultimate order: \filename{img1.exr}, \filename{img2.exr}, \filename{img3.exr},
  \filename{img4.exr}.

\item[Mixed literal names and response files.]\itemend
  Example:

  \begin{literal}
    enblend \val*{val:response-file-prefix-char} master.list image-09.png image-10.png
  \end{literal}

  where file~\filename{master.list} comprises of

  \begin{literal}
    image-01.png \\
    \val*{val:response-file-prefix-char} first.list \\
    image-04.png \\
    \val*{val:response-file-prefix-char} second.list \\
    image-08.png \\
  \end{literal}

  \filename{first.list} is

  \begin{literal}
    image-02.png \\
    image-03.png \\
  \end{literal}

  and \filename{second.list} contains

  \begin{literal}
    image-05.png \\
    image-06.png \\
    image-07.png \\
  \end{literal}

  Ultimate order: \filename{image-01.png}, \filename{image-02.png}, \filename{image-03.png},
  \filename{image-04.png}, \filename{image-05.png}, \filename{image-06.png},
  \filename{image-07.png}, \filename{image-08.png}, \filename{image-09.png},
  \filename{image-10.png},
\end{description}


\subsection[Response File Format]{\label{sec:response-file-format}%
  \genidx{response file!format}%
  \gensee{format of response file}{response file format}%
  Response File Format}

\genidx{\val*{val:response-file-comment-char} (response file comment)}%
\gensee{response file!comment (\sample{\val*{val:response-file-comment-char}})}%
       {\sample{\val*{val:response-file-comment-char}}}%
Response files contain one filename per line.  Blank lines or lines beginning with a
\code{\val{val:response-file-comment-char}}~sign are ignored; the latter can serve as comments.
Filenames that begin with a \code{\val{val:response-file-prefix-char}}~character denote other
response files.  \tableName~\ref{tab:response-file-format} states a formal grammar of response
files in \uref{\wikipediaebnf}{\acronym{EBNF}}.

\begin{table}
  \begin{tabular}{l@{$\quad::=\quad$}l}
    \metavar{response-file} & \metavar{line}* \\
    \metavar{line} & (\metavar{comment} $|$ \metavar{file-spec}) [\sample{\bslash r}] \sample{\bslash n} \\
    \metavar{comment} & \metavar{space}* \sample{\val*{val:response-file-comment-char}} \metavar{text} \\
    \metavar{file-spec} & \metavar{space}* \sample{\val*{val:response-file-prefix-char} } \metavar{filename} \metavar{space}* \\
    \metavar{space} & \sample{\textvisiblespace} $|$ \sample{\bslash t} \\
  \end{tabular}

  \noindent where \metavar{text} is an arbitrary string and \metavar{filename} is any filename.

  \caption[Grammar of response files]{\label{tab:response-file-format}%
    \genidx{response file!grammar}%
    \gensee{grammar!response file}{response file, grammar}%
    \acronym{EBNF} definition of the grammar of response files.}
\end{table}

In a response file relative filenames are used relative the response file itself, not relative
to the current-working directory of the application.

The above grammar might surprise the user in the some ways.

\begin{description}
\item[Whitespace trimmed at both line ends]\itemend
  For convenience, whitespace at the beginning and at the end of each line is ignored.  However,
  this implies that response files cannot represent filenames that start or end with whitespace,
  as there is no quoting syntax.  Filenames with embedded whitespace cause no problems, though.

\item[Only whole-line comments]\itemend
  Comments in response files always occupy a complete line.  There are no ``line-ending
  comments''.  Thus, in

  \begin{literal}
    \val*{val:response-file-comment-char} exposure series \\
    img-0.33ev.tif \val*{val:response-file-comment-char} "middle" EV \\
    img-1.33ev.tif \\
    img+0.67ev.tif \\
  \end{literal}

  only the first line contains a comment, whereas the second line includes none.  Rather, it
  refers to a file called

  \begin{literal}
    img-0.33ev.tif \val*{val:response-file-comment-char} "middle" EV
  \end{literal}

\item[Image filenames cannot start with \code{\val{val:response-file-prefix-char}}]\itemend
  A \code{\val{val:response-file-prefix-char}}~sign invariably introduces a response file, even
  if the filename's extension hints towards an image.
\end{description}

\genidx{response file!force recognition of}%
If \App{} or \OtherApp{} do not recognize a response file, they will skip the file and issue a
warning.  To force a file being recognized as a response file add one of the following syntactic
comments to the \emph{first} line of the file.

\begin{literal}
  response-file: true\synidx{response-file} \\
  enblend-response-file: true\synidx{enblend-response-file} \\
  enfuse-response-file: true\synidx{enfuse-response-file} \\
\end{literal}

Finally, \exampleName~\ref{ex:response-file} shows a complete response file.

\begin{exemplar}
  \begin{literal}
    \val*{val:response-file-comment-char}~4\bslash pi panorama! \\
    \mbox{} \\
    \val*{val:response-file-comment-char}~These pictures were taken with the panorama head. \\
    \val*{val:response-file-prefix-char}~round-shots.list \\
    \mbox{} \\
    \val*{val:response-file-comment-char}~Freehand sky shot. \\
    zenith.tif \\
    \mbox{} \\
    \val*{val:response-file-comment-char}~"Legs, will you go away?" images. \\
    nadir-2.tif \\
    nadir-5.tif \\
    nadir.tif \\
  \end{literal}

  \caption[Complete response file]%
          {\label{ex:response-file}%
            Example of a complete response file.}
\end{exemplar}


\subsection[Syntactic Comments]{\label{sec:syntactic-comments}%
  \genidx{response file!syntactic comment}%
  \gensee{syntactic comment!response file}{response file, syntactic comment}%
  Syntactic Comments}

Comments that follow the format described in
\tableName~\ref{tab:response-file-syntactic-comment} are treated as instructions how to
interpret the rest of the response file.  A syntactic comment is effective immediately and its
effect persists to the end of the response file, unless another syntactic comment undoes it.

\begin{table}
  \begin{tabular}{l@{$\quad::=\quad$}l}
    \metavar{syntactic-comment} & \metavar{space}* \sample{\val*{val:response-file-comment-char}}
    \metavar{space}* \metavar{key}
    \metavar{space}* \sample{:}
    \metavar{space}* \metavar{value} \\

    \metavar{key} & (\sample{A}\dots \sample{Z} $|$ \sample{a}\dots \sample{z} $|$ \sample{-})+ \\
  \end{tabular}

  where \metavar{value} is an arbitrary string.

  \caption[Grammar of syntactic comments]{\label{tab:response-file-syntactic-comment}%
  \genidx{syntactic comment!grammar}%
  \genidx{grammar!syntactic comment}%
    \acronym{EBNF} definition of the grammar of syntactic comments in response files.}
\end{table}

Unknown syntactic comments are silently ignored.

A special index for \flexipageref{syntactic comments}{sec:syncomm-index} lists them in
alphabetic order.


\subsection[Globbing Algorithms]{\label{sec:globbing-algorithms}%
  \genidx{globbing algorithm}%
  \gensee{algorithm}{globbing algorithm}%
  Globbing Algorithms}

The three equivalent syntactic keys

\begin{itemize}
\item
  \code{glob},\synidx{glob}

\item
  \code{globbing},\synidx{globbing} or

\item
  \code{filename-globbing}\synidx{filename-globbing}
\end{itemize}

control the algorithm that \App{} or \OtherApp{} use to glob filenames in response files.

\genidx{globbing algorithm!\code{literal}}%
\genidx{globbing algorithm!\code{wildcard}}%
All versions of \App{} and \OtherApp{} support at least two algorithms: \code{literal}, which is
the default, and \code{wildcard}.  See \tableName~\ref{tab:globbing-algorithms} for a list of
all possible globbing algorithms.  To find out about the algorithms in your version of \App{} or
\OtherApp{} use option~\option{--show-globbing-algorithms}.

\begin{table}
  \begin{minipage}{\linewidth}
    \begin{codelist}
      \genidx{globbing algorithm!\code{literal}}%
    \item[literal]\itemend
      Do not glob.  Interpret all filenames in response files as literals. This is the default.

      Please remember that whitespace at both ends of a line in a response file \emph{always}
      gets discarded.

      \genidx{globbing algorithm!\code{wildcard}}%
      \genidx{glob}%
    \item[wildcard]\itemend
      Glob using the wildcard characters~\sample{?}, \sample{*}, \sample{[}, and \sample{]}.

      The \propername{Win32} implementation only globs the filename part of a path, whereas all
      other implementations perform wildcard expansion in \emph{all} path components.  Also see
      \uref{\kernelorgglob}{\manpage{glob}{7}}.

      \genidx{globbing algorithm!\code{none}}
    \item[none]\itemend
      Alias for \code{literal}.

      \genidx{globbing algorithm!\code{shell}}
    \item[shell]\itemend
      The \code{shell} globbing algorithm works as \code{literal} does.  In addition, it
      interprets the wildcard characters~\sample{\{}, \sample{\atsign}, and \sample{\squiggle}.
      This makes the expansion process behave more like common \acronym{UN*X}
      shells.

      \genidx{globbing algorithm!\code{sh}}
    \item[sh]\itemend
      Alias for \code{shell}.
    \end{codelist}
  \end{minipage}

  \caption[Globbing algorithms]{\label{tab:globbing-algorithms}%
    \genidx{globbing algorithms}%
    \genidx{algorithms!globbing}%
    Globbing algorithms for the use in response files.}
\end{table}

\exampleName~\ref{ex:globbing-algorithm} gives an example of how to control filename-globbing in
a response file.

\begin{exemplar}
  \begin{literal}
    \val*{val:response-file-comment-char}~Horizontal panorama \\
    \val*{val:response-file-comment-char}~15 images \\
    \mbox{} \\
    \val*{val:response-file-comment-char}~filename-globbing: wildcard \\
    \mbox{} \\
    image\_000[0-9].tif \\
    image\_001[0-4].tif \\
  \end{literal}

  \caption[Filename-globbing syntactic comment]%
          {\label{ex:globbing-algorithm}%
            Control filename-globbing in a response file with a syntactic comment.}
\end{exemplar}


\subsection[Default Layer Selection]{\label{sec:default-layer-selection}%
  \genidx{default layer selection}%
  \genidx{layer selection!default}%
  Default Layer Selection}

The key~\code{layer-selector}\synidx{layer-selector} provides the same functionality as does the
command\hyp{}line option~\option{--layer-selector}, but on a per response\hyp{}file basis.  See
\sectionName~\ref{sec:common-options}.

This syntactic comment affects the layer selection of all images listed after it including those
in included response files until another \code{layer-selector} overrides it.

\genidx[\rangeendlocation]{response files}


%%% Local Variables:
%%% fill-column: 96
%%% End:
