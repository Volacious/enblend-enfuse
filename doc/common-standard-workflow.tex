%% This file is part of Enblend.
%% Licence details can be found in the file COPYING.


\section[Standard Workflow]{\label{sec:standard-workflow}%
  \genidx{workflow!standard}%
  \gensee{standard workflow}{workflow, standard}%
  Standard, All-In-One Workflow}

\figureName~\ref{fig:photographic-workflow} shows where \application{Enblend} and
\application{Enfuse} sit in the tool chain of the standard workflow.

\begin{figure}
  \begin{maxipage}
    \centering
    \includeimage{photographic-workflow}
  \end{maxipage}

  \caption[Photographic workflow]{\label{fig:photographic-workflow}%
    \genidx{workflow!\application{Enblend}}%
    \genidx{workflow!\application{Enfuse}}%
    \prgidx{dcraw}%
    \prgidx{ufraw}%
    \appidx{Hugin}%
    \appidx{PanoTools}%
    \appidx{Gimp}%
    Photographic workflow with \application{Enblend} and \application{Enfuse}.}
\end{figure}

\begin{description}
\item[Take Images]\itemend
  Take \emph{multiple} images to form a panorama, an exposure series, a focus stack, etc.\dots

  There is one exception with \application{Enfuse} when a single raw image is converted multiple
  times to get several -- typically differently ``exposed'' -- images.

  \noindent\emph{Exemplary Benefits:}

  \begin{itemize}
  \item
    Many pictures taken from the same vantage point but showing different viewing directions.
    -- Panorama

  \item
    Pictures of the same subject exposed with different shutter speeds.  -- Exposure series

  \item
    Images of the same subject focussed at differing distances.  -- Focus stack
  \end{itemize}

  \noindent\emph{Remaining Problem:} The ``overlayed'' images may not fit together, that is the
  overlay regions may not match exactly.

  \genidx{conversion!raw}%
  \gensee{raw conversion}{conversion, raw}%
\item[Convert Images]\itemend
  Convert the \uref{\luminouslandscaperawfiles}{raw data} exploiting the full dynamic range of
  the camera and capitalize on a high-quality conversion.

\item[Align Images]\itemend
  Align the images so as to make them match as well as possible.

  Again there is one exception and this is when images naturally align.  For example, a series
  of images taken from a rock solid tripod with a cable release without touching the camera, or
  images taken with a shift lens, can align without further user intervention.

  \genidx{transformation!affine}%
  \gensee{affine transformation}{transformation, affine}%
  This step submits the images to affine transformations.

  \genidx{lens distortion!correction of}%
  If necessary, it rectifies the lens' distortions (e.g.\ barrel or pincushion), too.

  \genidx{alignment!photometric}%
  \gensee{photometric alignment}{alignment, photometric}%
  Sometimes even luminance or color differences between pairs of overlaying images are corrected
  (``photometric alignment'').

  \noindent\emph{Benefit:} The overlay areas of images match as closely as possible given the
  quality if the input images and the lens model used in the transformation.

  \genidx{parallax error}%
  \noindent\emph{Remaining Problem:} The images may still not align perfectly, for example,
  because of \uref{\wikipediaparallax}{parallax} errors, or blur produced by camera shake.

\item[Combine Images]\itemend
  \application{Enblend} and \application{Enfuse} Enfuse combine the aligned images into one.

  \noindent\emph{Benefit:} The overlay areas become imperceptible for
  all but the most mal-aligned images.

  \genidx{channel!alpha}%
  \gensee{alpha channel}{channel, alpha}%
  \noindent\emph{Remaining Problem:} Enblend and Enfuse write images with an alpha channel; for
  more information on alpha channels see \chapterName~\fullref{sec:understanding-masks}.
  Furthermore, the final image rarely is rectangular.

\item[Postprocess]\itemend
  Postprocess the combined image with your favorite tool.  Often the user will want to crop the
  image and simultaneously throw away the alpha channel.

\item[View]

\item[Print]

\item[Enjoy]
\end{description}


%%% Local Variables:
%%% fill-column: 96
%%% End:
