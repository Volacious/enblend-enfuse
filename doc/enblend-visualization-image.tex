%% This file is part of Enblend.
%% Licence details can be found in the file COPYING.


\chapter[Visualization Image]{Visualization Image
  \label{sec:visualization-image}
  \genidx{seam-line!visualization}
  \gensee{visualization image}{seam-line, visualization}
  \gensee{image!seam-line}{seam-line, visualization}}

The visualization image shows the symmetric difference of the pixels
in the rectangular region where two images overlap.  The larger the
difference the lighter shade of gray it appears in the visualization
image.  \App{} paints the non-overlapping parts of the image pair --
these are the regions where \emph{no} blending occurs -- in
\val{val:visualize-no-overlap-value-color}.
Table~\ref{tab:visualization-colors} shows the meanings of all the
colors that are used in seam-line visualization images.

\begin{table}[htbp]
  \begin{description}
  \item[\val{val:visualize-no-overlap-value-color} areas]\itemend
    Non-overlapping parts of image pair.

  \item[various shades of gray]\itemend Difference of the pixel values
    in the overlap region.

  \item[\val{val:visualize-state-space-color} dot]\itemend Location of
    an optimizer sample.

  \item[\val{val:visualize-first-vertex-value-color} dot]\itemend
    First sample of a line segment.

  \item[\val{val:visualize-next-vertex-value-color} dot]\itemend Any
    other but first sample of a line segment.

    \genidx{radius!Dijkstra@\propername{Dijkstra}}%
    \gensee{Dijkstra@\propername{Dijkstra} radius}{radius, \propername{Dijkstra}}%
  \item[\val{val:visualize-state-space-inside-color} dot]\itemend
    State space sample inside the \propername{Dijkstra} radius.

  \item[\val{val:visualize-state-space-unconverged-color} dot]\itemend
    Non-converged point.

  \item[\val{val:visualize-initial-path-color} line]\itemend Initial
    seam line as generated by the primary seam generator.

  \item[\val{val:visualize-short-path-value-color} line]\itemend Final
    seam line.

    \genidx{seam-line!endpoint!frozen}%
    \gensee{frozen seam-line endpoint}{seam-line, endpoint, frozen}%
  \item[\val{val:visualize-frozen-point}
    \val{val:mark-frozen-point}]\itemend Non-movable, or ``frozen''
    endpoint of a seam-line segment that no optimizer is allowed to
    move around.

    \genidx{seam-line!endpoint!movable}%
    \gensee{movable seam-line endpoint}{seam-line, endpoint, movable}%
  \item[\val{val:visualize-movable-point}
    \val{val:mark-movable-point}]\itemend Movable endpoint of a
    seam-line segment, which seam-line optimizers can move.
  \end{description}

  \caption[Visualization colors an symbols]{Colors and symbols used in
    seam-line visualization images.\label{tab:visualization-colors}}

  \genidx{seam-line!visualization image colors}
  \gensee{colors!visualization image}{seam-line, visualization image colors}
  \gensee{visualization image colors}{seam-line, visualization image colors}
\end{table}


Figure~\ref{fig:seam-line-visualization} shows an
example\genidx{seam-line!visualization example}%
\gensee{visualization image example}{seam-line, visualization example}%
of a seam-line visualization.  It was produced with an \App{} run at
all defaults, but \sample{--fine-mask} and \sample{--visualize}
enabled.

\begin{figure}[htbp]
  \begin{maxipage}
    \centering
    \includeimage[width=1\linewidth]{seam-line-visualization}
  \end{maxipage}

  \caption[Seam-line visualization]{Seam-line visualization of a
    simple overlap.  \ifhevea \relax\else The
    853\classictimes238~pixel image has been rescaled to fit the width
    of the current page.\fi\label{fig:seam-line-visualization}}
\end{figure}

The large \val{val:visualize-no-overlap-value-color} border is
``off-limits'' for \App, for the images do not overlap there.  The
dark wedge inside the \val{val:visualize-no-overlap-value-color}
frame is where the images share a common region.

The initial seam-line (\val{val:visualize-initial-path-color}) is almost
straight with the exception of a single bend on the left side of the
image and the final seam-line
(\val{val:visualize-short-path-value-color}) meanders around it.
